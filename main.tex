\documentclass[conference]{IEEEtran}
\IEEEoverridecommandlockouts

\usepackage{cite}
\usepackage{amsmath,amssymb,amsfonts}
\usepackage{algorithmic}
\usepackage{graphicx}
\usepackage{float}
\usepackage{textcomp}
\usepackage{xcolor}

\def\BibTeX{{\rm B\kern-.05em{\sc i\kern-.025em b}\kern-.08em
    T\kern-.1667em\lower.7ex\hbox{E}\kern-.125emX}}

\usepackage[utf8]{inputenc}
\usepackage[T1, T5]{fontenc}
\usepackage{babel}
\usepackage{booktabs}

\usepackage{enumitem}

\babelprovide[import,main]{vietnamese}

\renewcommand{\IEEEkeywordsname}{Từ khóa}

\newcommand{\pl}[1]{{\fontencoding{T1}\selectfont #1}}

\begin{document}

\title{Phân loại bánh xe và các cấu trúc của robot di động}

\author{\IEEEauthorblockN{Nguyễn Văn Diễn}
%\IEEEauthorblockA{\textit{22027541} \\
%\textit{QH-2022-I/CQ-E-RE}}
\and
\IEEEauthorblockN{Hoàng Văn Cường}
%\IEEEauthorblockA{\textit{22027549} \\
%\textit{QH-2022-I/CQ-E-RE}}
\and
\IEEEauthorblockN{Vũ Đức Hiếu}
%\IEEEauthorblockA{\textit{22027527} \\
%\textit{QH-2022-I/CQ-E-RE}}
}

\maketitle

\thispagestyle{plain}
\pagestyle{plain}

\begin{abstract}

\end{abstract}

\begin{IEEEkeywords}
Robot di động, Phân loại, Động học robot, Cơ cấu bánh xe, Robot đa hướng
\end{IEEEkeywords}

\section{Giới thiệu}
Trong thập kỷ qua, lĩnh vực robot di động đã chứng kiến sự phát triển vượt bậc, 
trở thành một thành phần không thể thiếu trong kỷ nguyên Công nghiệp 4.0.
Từ các robot tự hành (AGV) vận chuyển hàng hóa trong kho bãi, robot dịch vụ trong bệnh viện,
đến các phương tiện thám hiểm tự hành trên các hành tinh khác, robot di động đang giải phóng 
con người khỏi các công việc lặp lại, nặng nhọc và nguy hiểm \cite{siegwart2011, fragapane2021}.

Một trong những yếu tố nền tảng quyết định khả năng vận hành của robot di động là 
cơ chế di chuyển. Mặc dù có nhiều phương thức di chuyển khác nhau như sử dụng chân, 
bay, hoặc bơi, cơ cấu sử dụng bánh xe vẫn là giải pháp phổ biến và chiếm ưu thế nhất 
trong hầu hết các ứng dụng trên mặt đất \cite{chung2008}. Ưu điểm vượt trội của robot bánh xe 
nằm ở hiệu suất năng lượng cao, khả năng chịu tải lớn, cơ khí đơn giản và 
thuật toán điều khiển tương đối hoàn thiện so với các cơ chế khác \cite{todd2012}.

Tuy nhiên, sự đa dạng về yêu cầu ứng dụng — từ di chuyển tốc độ cao trên đường nhựa 
đến khả năng xoay sở linh hoạt trong không gian chật hẹp — đã dẫn đến sự ra đời 
của hàng loạt thiết kế bánh xe và cấu trúc khung khác nhau. 
Việc lựa chọn giữa bánh xe tiêu chuẩn (standard wheel), bánh xe đa hướng (omnidirectional wheel), 
hay các cấu trúc dẫn động vi sai (differential drive), cấu trúc lái Ackermann... 
không chỉ ảnh hưởng đến khả năng cơ động mà còn quyết định mô hình động học và độ phức tạp của hệ thống điều khiển \cite{campion1993, lynch2017}.

Mặc dù đã có nhiều nghiên cứu riêng lẻ về từng loại robot, 
nhưng việc tổng hợp và hệ thống hóa mối liên hệ giữa đặc tính linh kiện và cấu trúc động học trong một tài liệu duy nhất là rất cần thiết để có cái nhìn toàn diện.
Bài báo này nhằm mục đích cung cấp một cái nhìn toàn diện và phân loại chi tiết về các hệ thống bánh xe và cấu trúc robot di động hiện có.

Cấu trúc của bài báo được tổ chức như sau: 
Phần II trình bày phân loại chi tiết các loại bánh xe, từ bánh xe tiêu chuẩn đến các thiết kế bánh xe đa hướng và đặc biệt. 
Phần III phân tích các cấu trúc robot di động phổ biến dựa trên mô hình động học của chúng. 
Phần IV tổng hợp và so sánh các đặc tính kỹ thuật. 
Cuối cùng, Phần V và Phần VI thảo luận về các ứng dụng thực tế, xu hướng nghiên cứu tương lai và kết luận.

%---------------------------------

\section{Phân loại bánh xe}
Bánh xe là cơ chế di chuyển được sử dụng nhiều nhất trong robot di động và trong các phương tiện do con người tạo ra \cite{siegwart2011}.
Việc sử dụng bánh xe đơn giản và tiết kiệm chi phí hơn so với cơ chế di chuyển bằng chân, đồng thời cũng dễ dàng hơn trong việc thiết kế, chế tạo và lập trình khi di chuyển trên bề mặt phẳng, không gồ ghề \cite{rubio2019}.
Nó có hiệu suất làm việc cao và sử dụng ít năng lượng hơn so với các cơ chế di chuyển khác (\textit{Hình \ref{fig:fig1}}).
Bánh xe có thể chia thành 3 loại chính gồm \textit{bánh xe tiêu chuẩn}, \textit{bánh xe đa hướng} và \textit{bánh xe đặc biệt} \cite{chung2008}.

\begin{figure}[!htbp]
    \centering
    \includegraphics[width=0.3\textwidth]{figures/fig1.pdf}
    \caption{Biểu đồ giữa tốc độ và năng lượng của các cơ chế di chuyển của robot \cite{todd2012}}
    \label{fig:fig1}
\end{figure}

\subsection{Bánh xe tiêu chuẩn (standard wheel)}
Bánh xe tiêu chuẩn là nhóm bánh xe phổ biến và cơ bản nhất trong robot di động. 
Đặc điểm chung của nhóm này là chúng tạo ra các ràng buộc phi đa hướng (non-holonomic constraints) \cite{rasam2016, siegwart2011}, 
nghĩa là tại một thời điểm, 
bánh xe chỉ có thể lăn hiệu quả theo một hướng và không thể trượt theo phương của trục bánh xe trong điều kiện lý tưởng.

Ba điều kiện cần được xác định cho một thiết kế bánh xe tiêu chuẩn \cite{chung2008}:

\begin{itemize}
    \item Xác định hai độ lệch $d$ và $b$
    \item Thiết kế cơ khí cho phép chuyển động lái hoặc không (tức là, có cố định hướng bánh xe hay không)
    \item Xác định cơ cấu dẫn động cho việc lái và di chuyển (tức là, dẫn động chủ động hay bị động)
\end{itemize}

Bánh xe tiêu chuẩn gồm ba loại: Bánh xe cố định, bánh xe lái và bánh xe xoay tự do.

\begin{figure}[!htbp]
    \centering
    \includegraphics[width=0.3\textwidth]{figures/fig000.pdf}
    \caption{Thiết kế tổng quát của bánh xe tiêu chuẩn. (a) Hình chiếu cạnh, (b) Hình chiếu đứng, và (c) Hình chiếu bằng \cite{chung2008}}
    \label{fig:fig000}
\end{figure}

\subsubsection{Bánh xe cố định (fixed wheel)}
Bánh xe cố định là loại bánh xe đơn giản nhất, được gắn chặt vào khung robot với một hướng duy nhất không thể thay đổi \cite{mikova2016, leong2022}.
Nó chỉ có một bậc tự do (degree of freedom) là quay quanh trục chính của nó \cite{chung2008, mikova2013}.
Nó có cấu trúc đơn giản, nhiều kích thước và có độ tin cậy cao \cite{shabalina2018}.

\begin{figure}[!htbp]
    \centering
    \includegraphics[width=0.4\textwidth]{figures/fig2.pdf}
    \caption{Mô hình động học tổng quát mô tả một bánh xe tiêu chuẩn được gắn trên khung robot \cite{chung2008}}
    \label{fig:fig2}
\end{figure}

Hình \ref{fig:fig2} trình bày sơ đồ động học tổng quát của một bánh xe tiêu chuẩn.
Trong sơ đồ, vị trí của tâm bánh xe $A$ được xác định trong hệ toạ độ cực
bởi các thông số khoảng cách $l$ và góc $\alpha$ so với hệ toạ độ gắn trên khung robot $\{X_R, Y_R\}$.
Hướng của bánh xe được xác định bởi góc $\beta$, là góc hợp bởi đường thẳng
nối tâm khung robot $P$ với tâm bánh xe $A$ và trục của bánh xe.

Đối với \textit{bánh xe cố định}, cả ba thông số ($l$, $\alpha$, và $\beta$)
đều là các hằng số thiết kế và không thay đổi trong quá trình robot di chuyển.
Bánh xe có bán kính $r$ và chuyển động của nó được định nghĩa bằng góc thay đổi theo thời gian $\varphi (t)$.

Về mặt toán học, \textit{bánh xe cố định} áp đặt hai ràng buốc động học chính 
lên robot trong điều kiện lý tưởng \cite{siegwart2011,mikova2016}:
\begin{itemize}
    \item[--] Ràng buộc lăn: Vận tốc tại điểm tiếp xúc mặt đất phải bằng không
    (lăn không trượt), điều này liên quan đến vận tốc quay của bánh xe:
    \begin{equation} \label{eq1}
        \begin{bmatrix}
            \sin(\alpha + \beta) & -cos(\alpha + \beta) & -l\cos{\beta}
        \end{bmatrix}
        R(\theta) \dot{\xi}_I - r\dot{\varphi} = 0
    \end{equation}
    \item[--] Ràng buộc trượt ngang: Vận tốc của tâm bánh xe (điểm $A$) theo phương
    vuông góc với trục bánh xe phải bằng không:
    \begin{equation} \label{eq2}
        \begin{bmatrix}
            \cos(\alpha + \beta) & \sin(\alpha + \beta) & l \sin{\beta}
        \end{bmatrix}
        R(\theta) \dot{\xi}_I = 0
    \end{equation}

    Với $\dot{\varphi}$ là vận tốc góc của bánh và  $\dot{\xi_I} = \begin{bmatrix} \dot{x} \\ \dot{y} \\ \dot{\psi} \end{bmatrix}$ là vectơ biểu diễn chuyển động của robot
    ($x$, $y$ và $\psi$ lần lượt là vị trí và hướng của robot). 
\end{itemize}

Từ phương trình ràng buộc (\ref{eq2}), ta có thể thấy rằng bánh xe cố định triệt tiêu hoàn toàn 
khả năng di chuyển tức thời theo phương ngang trục bánh.
Tính chất này biến robot sử dụng bánh xe cố định thành hệ thống phi đa hướng.
Về mặt điều khiển, điều này có nghĩa là không gian điều khiển (số lượng động cơ) nhỏ hơn không gian cấu hình (số bậc tự do vị trí),
buộc robot phải thực hiện các quỹ đạo phức tạp (như đỗ xe song song) để đến được vị trí
mong muốn thay vì di chuyển thẳng []. Tuy nhiên, chính ràng buộc này lại mang lại sự ổn định hướng
vượt trội khi di chuyển ở tốc độ cao, giúp robot giữ lộ trình tốt hơn so với các loại bánh đa hướng.

Các phương trình (\ref{eq1}) và (\ref{eq2}) dựa trên giả thuyết lý tưởng về "lăn không trượt" (pure rolling)
và tiếp xúc điểm cứng tuyệt đối (rigid point contact). 
Tuy nhiên, trong các ứng dụng thực tế, đặc biệt là với các robot vận tải (AGV) mang tải trọng lớn 
hoặc robot nông nghiệp di chuyển trên địa hình mềm, giả thuyết này thường bị vi phạm:

\begin{itemize}
    \item Hiện tượng trượt: Khi lực kéo vượt quá giới hạn ma sát nghỉ, hiện tượng trượt dọc sẽ xảy ra, dẫn đến sai số tích luỹ trong bài toán định vị. Ngoài ra, khi robot quay vòng, hiện tượng trượt ngang cũng xuất hiện do lực ly tâm, tạo ra góc trượt làm lệch hướng di chuyển thực tế so với hướng bánh xe.
    \item Biến dạng lốp: Diện tích tiếp xúc thực tế là một vùng thay vì một điểm, tạo ra mô-men cản lăn làm tiêu tốn năng lượng \cite{wong2022, iagnemma2000}.
\end{itemize}

Do đó, để giải quyết các thách thức này, kỹ thuật robot hiện đại thường tiếp cận theo hai hướng song song:
\begin{enumerate}
    \item Các mô hình động học được bổ sung thêm các tham số về độ trượt và độ cứng lốp để phản ánh chính xác hơn tương tác vật lý \cite{ward2008, yi2009, mandow2007}.
    \item Vì mô hình hóa không thể loại bỏ hoàn toàn sai số tích lũy, dữ liệu từ bánh xe (Odometry) bắt buộc phải được kết hợp với các cảm biến ngoại vi (như IMU, LiDAR) thông qua các bộ lọc ước lượng trạng thái (như EKF) để đảm bảo độ chính xác vị trí trong thời gian dài \cite{thrun2010, moore2014}.
\end{enumerate}

Nhờ sự kết hợp giữa độ tin cậy cơ khí cao và các giải pháp điều khiển/định vị tiên tiến này, bánh xe cố định vẫn giữ vững vai trò là nền tảng cốt lõi cho các cấu trúc robot phổ biến như dẫn động vi sai và dẫn động trượt \cite{parhi2011}.

\subsubsection{Bánh xe lái (steerable standard wheel)}
Bánh xe lái phức tạp hơn bánh cố định. Ngoài bậc tự do quay quanh trục của bánh xe, 
nó còn có thêm một bậc tự do thứ hai là khả năng quay quanh trục thẳng đứng (trục lái) đi qua tâm của bánh xe và điểm tiếp xúc với mặt đất \cite{parhi2011}.
Một động cơ lái thường được sử dụng để chủ động thay đổi góc của bánh xe so với khung của robot \cite{ueno2017}.
Mặc dù có thể thay đổi hướng, tại bất kỳ thời điểm nào, 
nó vẫn là một bánh xe tiêu chuẩn và phải tuân thủ ràng buộc lăn không trượt theo phương mà nó đang hướng theo.
Đây là thành phần cốt lõi trong cấu trúc lái Ackermann \cite{qiu2018, gautam2021} (như bánh trước xe ô tô) hoặc trong cấu trúc 3 bánh \cite{patel2021}.

\begin{figure}[!htbp]
    \centering
    \includegraphics[width=0.4\textwidth]{figures/fig3.pdf}
    \caption{Mô hình động học tổng quát mô tả một bánh xe lái được gắn trên khung robot \cite{chung2008}}
    \label{fig:fig3}
\end{figure}

Ràng buộc lăn của bánh (yêu cầu điều khiển vận tốc) \cite{low2005}:
\begin{equation} \label{eq:eq3}
    \begin{split}
        & \begin{bmatrix}
            \sin(\alpha + \beta(t)) & -\cos(\alpha + \beta(t)) & -l \cos(\beta(t))
        \end{bmatrix} R(\theta)\dot{\xi}_I \\
        & \hspace{6.5cm} - r\dot{\varphi} = 0
    \end{split}
\end{equation}

Ràng buộc trượt của bánh (yêu cầu điều khiển vị trí):
\begin{equation} \label{eq:eq4}
    \begin{split}
        \begin{bmatrix}
            \cos(\alpha + \beta(t)) & \sin(\alpha + \beta(t)) & l \sin(\beta(t))
        \end{bmatrix} R(\theta)\dot{\xi}_I = 0
    \end{split}
\end{equation}

So với bánh xe cố định, việc biến góc $\beta$ thành một biến số theo thời gian $\beta(t)$ mang lại sự linh hoạt vượt trội trong việc định hình quỹ đạo. Tuy nhiên, điều này dẫn đến những thách thức kỹ thuật đặc thù mà các hệ thống robot hiện đại phải giải quyết:

\begin{enumerate}
    \item Ma sát xoay và Hiện tượng "Cày lốp" (Scrubbing): Một vấn đề vật lý nghiêm trọng của bánh xe lái là hiện tượng ma sát xoay khi robot cần đổi hướng mà vận tốc lăn thấp hoặc bằng không. Để thay đổi góc $\beta$, động cơ lái phải thắng được mô-men ma sát trượt xoay (twisting friction) rất lớn giữa bề mặt lốp và mặt đường \cite{gillespie1992}. Điều này không chỉ gây mài mòn lốp nhanh chóng (tire wear) mà còn đòi hỏi động cơ lái phải có mô-men xoắn cực đại lớn, làm tăng kích thước và tiêu thụ năng lượng của hệ thống.
    \item Giới hạn tốc độ lái và Phối hợp điều khiển: Trong lý thuyết, góc lái $\beta$ có thể thay đổi tức thời. Tuy nhiên, trong thực tế, tốc độ thay đổi góc lái ($\beta$) bị giới hạn bởi động lực học của cơ cấu chấp hành. Nếu bộ điều khiển yêu cầu thay đổi hướng quá nhanh so với vận tốc lăn của robot, bánh xe sẽ bị trượt ngang (lateral slip), vi phạm ràng buộc (4). Do đó, các thuật toán điều khiển hiện đại cho robot dùng bánh lái (như Ackermann hoặc Tricycle) đòi hỏi sự phối hợp đồng bộ (coordinated control) chặt chẽ giữa vận tốc lăn $\varphi$ và vận tốc lái $\beta$ để đảm bảo quỹ đạo mượt mà \cite{deluca1998}.
    \item Sai số cơ khí (Backlash): Hệ thống truyền động lái (thường dùng bánh răng hoặc trục vít) luôn tồn tại độ rơ (backlash). Sai số này làm cho góc lái thực tế lệch so với góc lái đo được từ encoder, dẫn đến sai số trong bài toán định vị (Odometry) lớn hơn so với bánh xe cố định.
\end{enumerate}

Ứng dụng điển hình của bánh xe lái là trong các robot nông nghiệp cỡ lớn, xe tự hành ngoài trời hoặc các robot giao hàng tốc độ cao, nơi cần kết hợp giữa lực kéo mạnh và khả năng điều hướng ổn định theo kiểu ô tô.

%------- BÁNH XE XOAY TỰ DO -------------------

\subsubsection{Bánh xe xoay tự do (caster wheel)}
Bánh xe xoay tự do (bánh xe con lăn) cũng có hai bậc tự do tương tự như bánh lái (lăn và xoay quanh trục đứng).
Tuy nhiên, điểm khác biệt cơ bản là nó là một cơ cấu bị động.
Khi robot di chuyển, lực ma sát từ mặt đất sẽ tạo ra một mô-men xoắn, 
khiến bánh xe tự động xoay và căn chỉnh theo hướng di chuyển của robot để giảm thiểu lực cản.
Chúng không cung cấp bất kỳ lực đẩy chủ động nào.
Chúng không dùng để lái hay đẩy, mà chỉ dùng để hỗ trợ và giữ thăng bằng cho robot.
Chúng cực kỳ phổ biến và thường được dùng làm bánh xe phụ trong cấu trúc dẫn động vi sai \cite{arrizabalaga2021}, 
hoặc có thể giúp robot di chuyển lên bậc hoặc bề mặt không bằng phẳng \cite{lee2024, garcia2016}.

\begin{figure}[!htbp]
    \centering
    \includegraphics[width=0.4\textwidth]{figures/fig4.pdf}
    \caption{Mô hình động học tổng quát mô tả một bánh xe xoay tự do được gắn trên khung robot \cite{chung2008}}
    \label{fig:fig4}
\end{figure}

Ràng buộc lăn của bánh xoay tự do:
\begin{equation}
    \begin{split}
        & \begin{bmatrix}
            \sin(\alpha + \beta(t))  & -\cos(\alpha + \beta(t)) & -l \cos{\beta(t)}
        \end{bmatrix}
        R(\theta) \dot{\xi}_I \\
        & \hspace{6.5cm} - r\dot{\varphi} = 0
    \end{split}
\end{equation}

Ràng buộc trượt của bánh xoay tự do:
\begin{equation}
    \begin{split}
        \begin{bmatrix}
            \cos(\alpha + \beta) & \sin(\alpha + \beta) & d + l\sin{\beta}
        \end{bmatrix}
        R(\theta)\dot{\xi}_I + d\dot{\beta} = 0
    \end{split}
\end{equation}

Sự khác biệt cốt lõi giữa bánh Caster và bánh lái tiêu chuẩn nằm ở tham số $d$ (khoảng cách lệch tâm giữa trục lái và điểm tiếp xúc) trong phương trình (6). Sự xuất hiện của thành phần $d\dot{\beta}$ mang ý nghĩa vật lý quan trọng: mọi chuyển động ngang (trượt) của khung robot tại điểm gắn kết đều có thể được triệt tiêu bằng một tốc độ xoay $\dot{\beta}$ phù hợp của bánh xe.
Điều này dẫn đến một kết luận quan trọng: Nếu khớp xoay là tự do (ma sát thấp), bánh xe Caster không áp đặt bất kỳ ràng buộc động học nào lên chuyển động tổng thể của robot [17]. Robot có thể di chuyển theo bất kỳ quỹ đạo nào, và bánh Caster sẽ tự động xoay để "đi theo" (follow) chuyển động đó.

Các vấn đề Hiện đại và Thách thức Động lực học: Mặc dù về mặt hình học (kinematics), bánh Caster rất hoàn hảo để làm bánh phụ trợ, nhưng về mặt động lực học (dynamics), chúng gây ra nhiều vấn đề phức tạp cho các robot hiện đại:

\begin{enumerate}
    \item Hiện tượng Rung lắc (Caster Flutter/Shimmy):Đây là một vấn đề kinh điển nhưng vẫn là thách thức trong thiết kế robot tốc độ cao. Khi di chuyển nhanh, bánh Caster thường xảy ra hiện tượng dao động tự kích (self-excited oscillation) quanh trục lái, gây rung lắc dữ dội cho thân robot. Điều này xuất phát từ sự tương tác phức tạp giữa độ đàn hồi của lốp, độ lệch tâm $d$ và ma sát, điều mà các mô hình động học (5) và (6) không mô tả được [18]. Các robot hiện đại thường phải sử dụng bộ giảm chấn (damper) hoặc thiết kế tối ưu hóa độ cứng lốp để triệt tiêu hiện tượng này.
    \item Hiệu ứng "Đảo chiều" (Turn-around Effect): Khi robot đổi hướng di chuyển đột ngột (ví dụ: từ tiến sang lùi), bánh Caster phải thực hiện một hành trình xoay 180 độ quanh trục đứng để căn chỉnh lại. Quá trình này tạo ra một khoảng "trễ" và một lực phản hồi bất định lên khung robot, gây ra sai số vị trí lớn (Odometry error) và làm quỹ đạo thực tế bị giật cục. Trong các thuật toán định vị hiện đại (SLAM), hành vi khó đoán này của bánh Caster thường được coi là nhiễu (noise).
    \item Xu hướng mới: Active Caster (Bánh Caster chủ động): Để khắc phục tính bị động, một xu hướng nghiên cứu hiện đại là Active Caster (hay Powered Caster). Thay vì để bánh tự xoay do ma sát, người ta gắn động cơ vào cả trục lăn và trục lái. Khi đó, hệ thống trở thành một mô đun truyền động đa hướng (omnidirectional drive) mạnh mẽ, cho phép robot di chuyển holonomic mà không cần dùng bánh Mecanum đắt tiền [19].
\end{enumerate}

Tóm lại, bánh Caster là giải pháp rẻ tiền và hiệu quả để duy trì cân bằng tĩnh, nhưng sự phức tạp về động lực học của nó đòi hỏi sự cân nhắc kỹ lưỡng trong thiết kế robot chính xác cao.

%-----


% -----------------------

\subsection{Bánh xe đa hướng (omnidirectional wheel)}
Bánh xe đa hướng, hay còn gọi là bánh xe toàn hướng, là một bước tiến quan trọng so với bánh xe tiêu chuẩn.
Chúng được thiết kế để khắc phục các ràng buộc phi đa hướng của bánh xe tiêu chuẩn nên chúng ngày càng phổ biến
trong robot di động vì robot có thể đi thẳng từ điểm này đến điểm khác \cite{ignatiev2016}.
Một robot sử dụng các bánh xe này có thể di chuyển theo bất kỳ hướng nào (tiến/lùi, sang ngang, chéo) 
và xoay tại chỗ một cách đồng thời mà không cần phải xoay định hướng thân robot trước.
Khả năng này đạt được bằng cách gắn các con lăn bị động vào chu vi của bánh xe chính.

% ======================================================================

% ----- BÁNH XE OMNI -----

\subsubsection{Bánh xe omni (omni wheel)}

\begin{figure}[!htbp]
    \centering
    \includegraphics[width=0.35\textwidth]{figures/fig1500.pdf}
    \caption{Bánh xe omni với các con lăn nhỏ gắn dọc vành bánh \cite{komori2016}}
    \label{fig:fig1500}
\end{figure}

Cấu tạo và Nguyên lý: Bánh xe Omni (hay còn gọi là Universal Wheel) bao gồm một vành bánh xe chính và các con lăn nhỏ (passive rollers) được gắn dọc theo chu vi. Đặc điểm hình học cốt lõi là trục quay của các con lăn này vuông góc (90 độ) với trục quay chính của bánh xe \cite{patil2024, kanjanawanishkul2015}. Nguyên lý hoạt động dựa trên sự tách biệt chuyển động:

\begin{itemize}
    \item Động cơ dẫn động trục chính tạo ra lực đẩy chủ động (active traction) theo hướng dọc.
    \item Các con lăn tự do cho phép bánh xe trượt thụ động (passive motion) theo phương ngang. Sự kết hợp của nhiều bánh xe Omni (thường là 3 hoặc 4) cho phép tổng hợp véc-tơ vận tốc theo bất kỳ hướng nào \cite{kanjanawanishkul2015, komori2016}.
\end{itemize}

Các Biến thể Cấu trúc và Giảm chấn (Structural Variations): Một nhược điểm cố hữu của thiết kế Omni cổ điển là sự không liên tục của điểm tiếp xúc (discontinuous contact points). Giữa các con lăn luôn tồn tại khoảng hở, gây ra rung động dọc (vertical vibration) khi bánh xe quay. Ferrière và các cộng sự \cite{ferriere1996} đã phân tích rằng biên độ rung động này phụ thuộc trực tiếp vào kích thước khe hở và bán kính con lăn. Để khắc phục, các biến thể cấu trúc hiện đại đã được phát triển:

\begin{itemize}
    \item Bánh Omni Kép (Dual-row / Double Wheel): Sử dụng hai đĩa bánh xe ghép lại, lệch pha nhau để đảm bảo biên dạng tiếp xúc liên tục. Tuy nhiên, thiết kế này làm tăng bề rộng và khối lượng bánh xe.
    \item Thiết kế Con lăn Xen kẽ (Alternate Concept): Các con lăn được thiết kế đan xen (interpenetrated) để giảm thiểu khe hở mà không cần dùng bánh kép, giúp giảm rung động hiệu quả.
\end{itemize}

\textbf{Phân tích Động lực học Trượt (Slip Dynamics - Modern Issue)}: Trong các nghiên cứu hiện đại, mô hình động học lý tưởng (lăn không trượt) đã được chứng minh là không đủ chính xác. Williams và các cộng sự \cite{williams2002} đã phát hiện ra rằng phần vật liệu cứng nằm giữa các con lăn (discontinuities) đóng vai trò quan trọng trong việc gây ra hiện tượng trượt (slip), ngay cả ở tốc độ thấp. Nghiên cứu thực nghiệm chỉ ra rằng hệ số ma sát trong hướng lăn ($\mu_W$) và hướng trượt ngang ($\mu_T$) là khác nhau đáng kể (ví dụ: $\mu_W \approx 0.26$ so với $\mu_T \approx 0.09$ trên bề mặt giấy). Do đó, các bộ điều khiển robot hiện đại cần tích hợp mô hình ma sát phi tuyến để bù trừ sai số trượt này, thay vì chỉ dựa vào mô hình hình học thuần túy.

Đột phá về Vật liệu (Material Innovation - 2024): Xu hướng "nhẹ hóa" robot đang thúc đẩy việc thay thế các vật liệu kim loại truyền thống. Nghiên cứu mới nhất của Patil và các cộng sự \cite{patil2024} đã đề xuất sử dụng vật liệu Composite sợi Carbon gia cường (Carbon Fiber Reinforced Polymer) để chế tạo đĩa bánh xe omni thay cho nhôm.

\begin{itemize}
    \item Sử dụng phương pháp đắp tay (hand lay-up) với nhựa epoxy, đĩa bánh xe sợi carbon đạt được độ bền uốn (bending strength) lên tới 2.2 KN.
    \item Mô phỏng ANSYS cho thấy biến dạng tối đa chỉ tập trung ở vùng cổ răng bánh xe và nằm trong giới hạn an toàn. Đây là bước tiến quan trọng giúp giảm quán tính quay của bánh xe, từ đó nâng cao khả năng tăng tốc và tiết kiệm năng lượng cho robot.
\end{itemize}

Hướng nghiên cứu mới: Bánh Omni Chủ động (Active Omni Wheel): Một hạn chế lớn của bánh Omni thường là khó kiểm soát chính xác chuyển động thụ động của con lăn do lực cản lăn không xác định. Để giải quyết vấn đề này, Komori và các cộng sự \cite{komori2016} đã đề xuất cơ cấu bánh omni chủ động.

\begin{itemize}
    \item Thiết kế này sử dụng cơ cấu vi sai (differential gear) để dẫn động chủ động cả trục bánh xe chính và trục của các con lăn con.
    \item Kết quả là một bánh xe duy nhất có thể chủ động di chuyển theo hướng bất kỳ (active motion in arbitrary direction) mà không cần phụ thuộc vào lực kéo của các bánh xe khác, mở ra khả năng thiết kế các loại xe vận chuyển đa hướng với cấu hình linh hoạt hơn.
\end{itemize}

% ======================================================================

% ----- BÁNH XE MECANUM -----

\subsubsection{Bánh xe Mecanum (Mecanum wheel)}

Bánh xe Mecanum đại diện cho giải pháp kỹ thuật phổ biến nhất hiện nay để đạt được khả năng di chuyển toàn hướng mà không cần cơ cấu lái phức tạp.

\begin{figure}[!htbp]
    \centering
    \includegraphics[width=0.3\textwidth]{figures/fig_ilon1975.pdf}
    \caption{Sáng chế bánh xe Mecanum của B. E. Ilon \cite{ilon1975}}
    \label{fig:fig_ilon1975}
\end{figure}

\begin{figure}[!htbp]
    \centering
    \includegraphics[width=0.3\textwidth]{figures/fig_barnett2003.pdf}
    \caption{Sáng chế bánh xe Mecanum của D. H. Barnett \cite{barnett2003}}
    \label{fig:fig_barnett2003}
\end{figure}

\begin{figure}[!htbp]
    \centering
    \includegraphics[width=0.3\textwidth]{figures/fig_potter2010.pdf}
    \caption{Sáng chế bánh xe Mecanum của S. D. Potter \cite{potter2010}}
    \label{fig:fig_potter2010}
\end{figure}

\begin{figure}[!htbp]
    \centering
    \includegraphics[width=0.3\textwidth]{figures/fig_baker2010.pdf}
    \caption{Sáng chế bánh xe Mecanum của A. R. Baker \cite{baker2010}}
    \label{fig:fig_baker2010}
\end{figure}

\begin{figure}[!htbp]
    \centering
    \includegraphics[width=0.3\textwidth]{figures/fig_scogna2011.pdf}
    \caption{Sáng chế bánh xe Mecanum của J. Scogna \cite{scogna2011}}
    \label{fig:fig_scogna2011}
\end{figure}

\begin{figure}[!htbp]
    \centering
    \includegraphics[width=0.3\textwidth]{figures/fig_schlee2013.pdf}
    \caption{Sáng chế bánh xe Mecanum của K. L. Schlee \cite{schlee2013}}
    \label{fig:fig_schlee2013}
\end{figure}

\begin{figure}[!htbp]
    \centering
    \includegraphics[width=0.3\textwidth]{figures/fig_gunther2014.pdf}
    \caption{Sáng chế bánh xe Mecanum của W. G\"unther \cite{gunther2014}}
    \label{fig:fig_gunther2014}
\end{figure}

\textbf{Lịch sử và Cấu hình Hình học:} Được phát minh vào năm 1973 bởi Bengt Ilon tại Thụy Điển \cite{ilon1975, wikipedia_mecanum_wheel}, bánh xe Mecanum có cấu tạo gồm một trục chính và các con lăn thụ động (rollers) bao quanh. Đặc điểm nhận dạng là trục của các con lăn tạo một góc $\alpha = 45^{\circ}$ so với trục bánh xe.

\begin{itemize}
    \item \textbf{Chế tạo:} Salih và các cộng sự \cite{salih2006} đã mô tả chi tiết quy trình chế tạo một bánh Mecanum tùy chỉnh gồm 9 con lăn làm từ vật liệu Delrin (một loại nhựa kỹ thuật có độ ma sát thấp và độ cứng cao), giúp giảm thiểu mài mòn khi trượt ngang.
    \item \textbf{Bố trí:} Để robot hoạt động đúng, 4 bánh xe phải được lắp đặt theo quy tắc hình chữ nhật hoặc hình vuông, sao cho các trục con lăn tạo thành hình chữ X hướng vào tâm robot khi nhìn từ trên xuống \cite{hasana2021, salih2006}.
\end{itemize}

\textbf{Cơ chế Tạo lực và Động học:} Nguyên lý vận hành dựa trên việc phân tách véc-tơ lực ma sát. Khi bánh xe quay, lực tác dụng lên mặt đất nghiêng $45^{\circ}$.

\begin{itemize}
    \item \textbf{Phối hợp véc-tơ:} Bằng cách điều khiển độc lập chiều quay và tốc độ của 4 bánh, các thành phần lực dọc và ngang có thể cộng hưởng hoặc triệt tiêu. Ví dụ, Salih và các cộng sự \cite{salih2006} đã thực nghiệm và chỉ ra rằng để di chuyển ngang (sideways), các bánh chéo nhau phải quay cùng chiều, trong khi hai bánh cùng bên quay ngược chiều.
    \item \textbf{Vấn đề Trượt (Slippage):} Mặc dù mô hình động học lý thuyết (dựa trên ma trận Jacobian) rất hoàn hảo \cite{dyakov2024}, Ignatiev và các cộng sự \cite{ignatiev2016} và Salih và các cộng sự \cite{salih2006} đều nhấn mạnh rằng trong thực tế, hiện tượng trượt là không thể tránh khỏi do diện tích tiếp xúc nhỏ (điểm) của con lăn. Điều này làm giảm độ chính xác của phương pháp định vị bằng Odometry (đếm xung encoder), đòi hỏi các giải pháp bù trừ sai số.
\end{itemize}

\textbf{Vấn đề Hiện đại: Rung động và Tối ưu hóa Thiết kế:} Nhược điểm lớn nhất của bánh Mecanum là rung động do sự không liên tục của điểm tiếp xúc khi chuyển giao giữa các con lăn.

\begin{itemize}
    \item \textbf{Phân tích Rung động:} Bae và Kang \cite{bae2016} đã đo đạc thực nghiệm và phát hiện ra gia tốc rung động dọc trục có tính bất đối xứng (asymmetric acceleration), tăng mạnh ở tốc độ cao.
    \item \textbf{Tối ưu hóa Topo (Topology Optimization):} Để khắc phục, Doroliat và các cộng sự \cite{doroliat2022} đề xuất sử dụng thuật toán tối ưu hóa topo để thiết kế lại cấu trúc bên trong của lớp cao su bọc con lăn. Bằng cách thay đổi độ dày vật liệu tại các vị trí khác nhau, họ tạo ra độ cứng tiếp xúc đồng nhất (uniform stiffness), giúp giảm rung động tới 30\%.
\end{itemize}

\textbf{Hệ thống Điều khiển và Tích hợp (System Integration):} Sự phát triển của bánh Mecanum gắn liền với sự tiến bộ của hệ thống nhúng.

\begin{itemize}
    \item \textbf{Điều khiển cơ bản:} Salih và các cộng sự \cite{salih2006} trình bày kiến trúc điều khiển kinh điển sử dụng vi điều khiển (như BasicStamp) kết hợp với mạch cầu H để điều xung PWM độc lập cho 4 động cơ DC, cho phép thực hiện các chuyển động cơ bản.
    \item \textbf{Điều khiển Tự hành Nâng cao:} Ignatiev và các cộng sự \cite{ignatiev2016} đã nâng cấp hệ thống lên mức tự hành bằng cách tích hợp Thị giác máy tính. Robot được trang bị camera và máy tính nhúng để nhận diện mẫu và bám theo đối tượng, khắc phục nhược điểm sai số vị trí của bánh Mecanum.
    \item \textbf{Mô phỏng Động lực học:} Để kiểm chứng thuật toán trước khi chế tạo, Dosoftei và các cộng sự \cite{dosoftei2020} và Zimmermann và các cộng sự \cite{zimmermann2014} đề xuất các phương pháp mô phỏng động lực học đa vật thể và mô hình hóa giản lược, giúp tính toán chính xác lực va chạm và mô-men xoắn yêu cầu.
\end{itemize}

\textbf{Ứng dụng Mở rộng:} Không chỉ giới hạn ở xe vận chuyển (AGV) trong nhà kho, bánh Mecanum đang được ứng dụng trong các hệ thống phức tạp hơn:

\begin{itemize}
    \item \textbf{Mobile Manipulator:} Ignatiev và các cộng sự \cite{ignatiev2016} giới thiệu robot tích hợp cánh tay máy công nghiệp (như PUMA 560) đặt trên nền tảng Mecanum, cho phép thực hiện các thao tác lắp ráp linh hoạt trong không gian hẹp.
    \item \textbf{Y tế và Vũ trụ:} Các ứng dụng xe lăn thông minh và robot thám hiểm không gian (NASA OmniBot) tận dụng khả năng xoay sở trong môi trường chật hẹp \cite{giurgiu2022, yadav2022}.
\end{itemize}

\begin{figure}[!htbp]
    \centering
    \includegraphics[width=0.35\textwidth]{figures/fig99nasa.pdf}
    \caption{NASA OmniBot mobile base \cite{doroftei2011}}
    \label{fig:fig99nasa}
\end{figure}

% ======================================================================

% ----- BÁNH XE ĐẶC BIỆT -----

\subsection{Bánh xe đặc biệt}
Nhóm này bao gồm các thiết kế bánh xe mang tính thử nghiệm, đột phá hoặc mới lạ, 
thường tập trung vào các lĩnh vực nghiên cứu cụ thể như robot tự cân bằng hoặc robot có khả năng thích ứng địa hình.

% ======================================================================

% ----- BÁNH HÌNH CẦU -----

\subsubsection{Bánh hình cầu}
Bánh hình cầu là một cơ cấu bánh xe sử dụng một quả cầu để tiếp xúc với mặt đất. 
Thiết kế này về cơ bản cho phép chuyển động đa hướng.
Chúng ta có thể phân biệt hai loại chính: bánh cầu bị động và bánh cầu chủ động.

\begin{itemize}
    \item Bánh cầu bị động: Nó hoạt động như một bánh xe xoay tự do đa hướng, bị động.
          Thường được dùng làm bánh xe hỗ trợ (thứ ba hoặc thứ tư) để giữ thăng bằng cho các robot có cấu trúc dẫn động vi sai.
    \item Bánh cầu chủ động: Đây là một cơ cấu cơ khí cực kỳ phức tạp, trong đó quả cầu được chủ động điều khiển để tạo lực đẩy.
          Quả cầu được giữ trong một hốc (socket) và tiếp xúc trực tiếp với các con lăn (rollers) 
          hoặc động cơ được đặt bên trong cơ cấu. Bằng cách điều khiển các con lăn này 
          (ví dụ, hai con lăn đặt vuông góc với nhau), cơ cấu có thể làm quả cầu quay theo bất kỳ trục nào trên mặt phẳng \cite{Lauwers2006, mukherjee1999}.
          Một bánh xe cầu chủ động là một cơ cấu truyền động đa hướng (holonomic) hoàn chỉnh. 
          Một robot được trang bị 3 hoặc 4 bánh xe này có thể di chuyển theo mọi hướng và xoay đồng thời, 
          tương tự như bánh Mecanum nhưng tiềm năng về độ mượt mà và khả năng vượt địa hình gồ ghề (nhẹ) tốt hơn \cite{nagarajan2009}.
          Ưu điểm là khả năng cơ động đa hướng tuyệt đối. 
          Nhược điểm là độ phức tạp cơ khí cực cao, chi phí lớn, và khó khăn trong việc duy trì lực bám và chống mài mòn \cite{Kumagai2008}.
\end{itemize}

\begin{figure}[!htbp]
    \centering
    \includegraphics[width=0.4\textwidth]{figures/fig5.pdf}
    \caption{Mô hình động học tổng quát mô tả một bánh xe hình cầu được gắn trên khung robot \cite{chung2008}}
    \label{fig:fig5}
\end{figure}

Quan sát mô hình động học (\textit{Hình \ref{fig:fig5}}), ta thấy các tham số hình học ($l$, $\alpha$, $\beta$) tương tự như \textit{bánh xe cố định}. Tuy nhiên, ý nghĩa vật lý của các phương trình ràng buộc lại hoàn toàn khác biệt \cite{siegwart2011}:

Phương trình lăn:
    \begin{equation} \label{eq7}
        \begin{bmatrix}
            \sin(\alpha + \beta) & -cos(\alpha + \beta) & -l\cos{\beta}
        \end{bmatrix}
        R(\theta) \dot{\xi}_I - r\dot{\varphi} = 0
    \end{equation}

Phương trình trượt:
    \begin{equation} \label{eq8}
        \begin{bmatrix}
            \cos(\alpha + \beta) & \sin(\alpha + \beta) & l \sin{\beta}
        \end{bmatrix}
        R(\theta) \dot{\xi}_I = 0
    \end{equation}

Đối với \textit{bánh xe cố định}, góc $\beta$ là hằng số, nên phương trình (\ref{eq8}) là một ràng buộc ngăn cản chuyển động ngang. Ngược lại, đối với \textit{bánh hình cầu}, góc $\beta$ là một biến tự do. Do đó, phương trình (\ref{eq8}) không hạn chế chuyển động của robot ($\dot{\xi}_I$), mà nó dùng để xác định hướng quay tức thời của quả cầu sao cho phù hợp với hướng di chuyển của robot. Nói cách khác, \textit{bánh hình cầu} không áp đặt bất kỳ ràng buộc động học nào lên khung robot, cho phép robot di chuyển toàn hướng.

Thách thức hiện đại: Mặc dù lý thuyết động học rất lý tưởng, việc hiện thực hóa \textit{bánh hình cầu} chủ động gặp thách thức lớn về hiệu suất truyền động. Lực đẩy phụ thuộc hoàn toàn vào ma sát tiếp xúc điểm giữa con lăn dẫn động và quả cầu. Nếu ma sát không đủ sẽ gây trượt, nếu quá lớn sẽ gây mài mòn và tổn hao năng lượng. Đây là bài toán khó về vật liệu và cơ khí chính xác mà các nghiên cứu hiện nay đang tập trung giải quyết \cite{Kumagai2008}.

% ======================================================================

% ----- BÁNH ORIGAMI -----

\subsubsection{Bánh Origami}
Đây là một hướng nghiên cứu mới, ứng dụng các nguyên lý của nghệ thuật gấp giấy Nhật Bản (Origami) để tạo ra các bánh xe có thể thay đổi hình dạng.
Ưu điểm lớn nhất là khả năng thích ứng địa hình (terrain adaptability).
Nhược điểm là độ phức tạp về cơ khí, độ bền của các khớp gấp, và sự phức tạp trong hệ thống điều khiển để quyết định khi nào cần thay đổi hình dạng [..].

Cơ chế và Nguyên lý Biến hình: Hầu hết các bánh xe Origami hiện đại dựa trên các mẫu gấp (tessellation patterns) nổi tiếng như mẫu "Bóng ma thuật" (Magic Ball pattern) hoặc "Waterbomb base" [].

\begin{itemize}
    \item Cấu trúc: Bánh xe được cấu tạo từ một màng mỏng (thường là composite hoặc polymer) được gấp nếp theo quy luật hình học xác định.
    
    \item Hoạt động: Một cơ cấu chấp hành tuyến tính (linear actuator) ở trục bánh xe sẽ đẩy hai đĩa ốp hai bên lại gần hoặc ra xa nhau. Nhờ cấu trúc gấp nếp, chuyển động tuyến tính này sẽ được chuyển đổi thành chuyển động hướng kính, làm thay đổi đường kính của bánh xe.
    \begin{itemize}
        \item Trạng thái thu gọn: Đường kính nhỏ, bề mặt trơn nhẵn → Di chuyển nhanh, hiệu suất cao trên mặt phẳng.
        \item Trạng thái mở rộng: Đường kính lớn, bề mặt xuất hiện các "gai" hoặc mấu bám → Tăng mô-men vượt chướng ngại vật, leo cầu thang hoặc đi trên đất mềm.
    \end{itemize}
\end{itemize}

Đặc tính Động học Biến thiên (Variable Kinematics): Điểm khác biệt "hiện đại" nhất so với bánh xe truyền thống là bán kính bánh xe r không còn là hằng số mà là một hàm số phụ thuộc vào trạng thái biến hình u(t). Phương trình ràng buộc lăn (1) trở thành phi tuyến:

Điều này đặt ra thách thức lớn cho bài toán Odometry và điều khiển PID, vì hệ số khuếch đại của hệ thống thay đổi liên tục khi bánh xe biến hình. Bộ điều khiển cần phải thích nghi (Adaptive Control) để cập nhật lại mô hình động học theo thời gian thực.

Các vấn đề Hiện đại và Thách thức Kỹ thuật: Nghiên cứu về bánh xe Origami đang đối mặt với 3 bài toán lớn:

\begin{itemize}
    \item Mâu thuẫn giữa Độ cứng và Độ linh hoạt (Stiffness vs. Flexibility Trade-off): Để biến hình, vật liệu cần mềm dẻo (flexible). Nhưng để chịu tải trọng của robot (Payload), bánh xe cần phải cứng (stiff).
    \begin{itemize}
        \item Giải pháp hiện đại: Các nghiên cứu mới nhất (như của ĐH Quốc gia Seoul) đề xuất cơ chế khóa cấu trúc (structural locking). Khi ở trạng thái mong muốn, bánh xe tự khóa cứng lại để chịu tải lên tới hàng trăm kg, nhưng khi cần biến hình thì mở khóa để trở nên mềm dẻo [Ref].
    \end{itemize}

    \item Hiệu ứng Đa giác (Polygon Effect): Khi mở rộng, bánh xe Origami thường không giữ được hình tròn hoàn hảo mà tạo thành hình đa giác. Điều này gây ra rung động chu kỳ (vibration) khi robot di chuyển, ảnh hưởng đến độ ổn định của camera hoặc cảm biến trên thân robot.
    
    \item Độ bền mỏi (Fatigue Durability): Các nếp gấp (hinges) là nơi tập trung ứng suất. Sau hàng nghìn chu kỳ gấp/mở, vật liệu rất dễ bị nứt gãy. Công nghệ vật liệu hiện đại đang hướng tới các loại vải gia cường (fabric-reinforced composites) để tăng tuổi thọ cho bánh xe.
\end{itemize}

Ứng dụng tiềm năng nhất của công nghệ này là các robot giao hàng (last-mile delivery) có khả năng tự leo vỉa hè hoặc robot thám hiểm bề mặt hành tinh với địa hình không xác định.

% ======================================================================

% ----- TỔNG HỢP, SO SÁNH CÁC LOẠI BÁNH XE -----

\subsection{Tổng hợp, so sánh các loại bánh xe}

Sau khi phân tích chi tiết đặc tính động học và cơ khí của từng nhóm bánh xe, phần này tổng hợp và so sánh chúng để cung cấp cái nhìn toàn diện cho việc lựa chọn thiết kế robot. Trong kỹ thuật robot hiện đại, không có loại bánh xe nào là ưu việt tuyệt đối; sự lựa chọn luôn là kết quả của bài toán tối ưu hóa đa mục tiêu giữa: Khả năng cơ động (Mobility), Độ phức tạp (Complexity) và Khả năng thích ứng địa hình (Traversability) [].

Bảng \ref{tab:table1} dưới đây tóm tắt các đặc tính kỹ thuật cốt lõi của các loại bánh xe đã khảo sát.

\begin{table*}[t]
  \centering
  \caption{TỔNG HỢP VÀ SO SÁNH CÁC LOẠI BÁNH XE}
  \label{tab:table1}
  \begin{tabular}{llllll} 
    \toprule
    \textbf{Loại bánh xe} & \textbf{Loại ràng buộc} & \textbf{Độ phức tạp cơ khí} & \textbf{Độ phức tạp điều khiển} & \textbf{Thích ứng địa hình} & \textbf{Hiệu suất năng lượng} \\
    \midrule
    Bánh xe cố định & Phi đa hướng & Thấp & Thấp & Cao & Cao \\
    Bánh xe lái & Phi đa hướng & Trung bình & Trung bình & Cao & Cao \\
    Bánh xe xoay tự do & Bị động & Thấp & Trung bình & Trung bình & Trung bình \\
    Bánh xe omni & Đa hướng & Trung bình & Cao & Thấp & Trung bình \\
    Bánh xe Mecanum & Đa hướng & Cao & Rất cao & Thấp & Thấp \\
    Bánh hình cầu & Đa hướng & Rất cao & Rất cao & Thấp & Thấp \\
    Bánh Origami & Biến thiên & Rất cao & Rất cao & Rất cao & Trung bình \\
    \bottomrule
  \end{tabular}
\end{table*}

Từ bảng so sánh, ta có thể rút ra ba xu hướng thiết kế chính trong robot hiện đại:

\begin{enumerate}
    \item Cơ động vs. Địa hình (Maneuverability vs. Traversability): Có một sự nghịch đảo rõ rệt giữa khả năng di chuyển linh hoạt và khả năng vượt địa hình.
    \begin{itemize}
        \item Các bánh xe toàn hướng (Omni, Mecanum, Hình cầu) cho phép robot di chuyển ngang và xoay tại chỗ trong không gian hẹp, nhưng cấu tạo con lăn nhỏ hoặc bề mặt cầu khiến chúng cực kỳ nhạy cảm với bụi bẩn, gờ giảm tốc và nền đất mềm [].
        \item Ngược lại, bánh xe tiêu chuẩn (Cố định, Lái) chịu tải tốt và bám đường tốt trên địa hình gồ ghề, nhưng lại bị hạn chế bởi bán kính quay vòng (đối với bánh lái) hoặc ma sát trượt khi quay (đối với bánh cố định).
        \item Xu hướng hiện đại: Bánh xe Origami nổi lên như một giải pháp lai, hy sinh sự đơn giản để đạt được cả hai yếu tố trên nhờ khả năng biến hình.
    \end{itemize}

    \item Độ phức tạp Điều khiển và Định vị (Control \& Localization):
    \begin{itemize}
        \item Bánh xe tiêu chuẩn có mô hình động học ổn định, ít bị trượt ngang, giúp thuật toán Odometry đạt độ chính xác cao.
        \item Các bánh xe Holonomic (đặc biệt là Mecanum và Ballbot) chịu ảnh hưởng lớn bởi sự trượt (slippage) ngẫu nhiên. Điều này đặt ra thách thức lớn cho hệ thống điều khiển: bắt buộc phải sử dụng các bộ lọc ước lượng trạng thái (như EKF/Particle Filter) và hợp nhất cảm biến (Sensor Fusion) đắt tiền để bù đắp sai số vị trí tích lũy [].
    \end{itemize}

    \item Hiệu suất Năng lượng (Energy Efficiency):
    \begin{itemize}
        \item Bánh xe cố định và bánh lái tận dụng ma sát lăn thuần túy, tiêu tốn ít năng lượng nhất.
        \item Bánh Omni và Mecanum tiêu tốn năng lượng đáng kể do lực ma sát trượt nội tại giữa các con lăn và trục, cũng như việc các véc-tơ lực của 4 bánh thường triệt tiêu lẫn nhau khi tạo chuyển động ngang []. Đây là lý do chúng hiếm khi được dùng cho các robot chạy pin hoạt động ngoài trời trong thời gian dài.
    \end{itemize}
\end{enumerate}

% ======================================================================

% ----- PHÂN LOẠI CẤU TRÚC ROBOT DI ĐỘNG -----

\section{Phân loại cấu trúc robot di động}

Trong thiết kế robot di động sử dụng bánh xe, các phương án cấu trúc rất đa dạng. Đối với robot di động, bài toán thiết kế tập trung vào ba yếu tố chính: lựa chọn loại bánh xe, bố trí vị trí bánh xe và xác định các tham số động học. Các mục tiêu thiết kế này cần được xác định cụ thể dựa trên môi trường hoạt động, nhiệm vụ được giao, cũng như cân nhắc về chi phí đầu tư và vận hành. Phần này sẽ phân loại các cấu trúc robot dựa trên số lượng bánh xe, đồng thời giới thiệu các đặc điểm của những thiết kế phổ biến hiện nay.

% ======================================================================

\subsection{Robot một bánh}

\subsubsection{Robot xe đạp một bánh (unicycle robot)}

Đây là cấu trúc robot di động có đặc điểm chỉ tiếp xúc với mặt đất tại một điểm duy nhất thông qua bánh xe. Về mặt động lực học, hệ thống này vốn không ổn định ở cả hai phương dọc và phương ngang, đồng thời là hệ thống thiếu cơ cấu chấp hành. Do đó, các nghiên cứu tập trung chủ yếu vào việc thiết kế cơ cấu phụ trợ để duy trì cân bằng phương ngang trong khi bánh xe chính duy trì cân bằng phương dọc. Dựa trên các tài liệu khảo sát, các phương pháp tiếp cận hiện đại có thể được phân loại như sau:

\begin{figure}[!htbp]
    \centering
    \includegraphics[width=0.4\textwidth]{figures/fig_Lee2013.pdf}
    \caption{Mô hình đơn giản của robot xe đạp một bánh sử dụng bánh phản lực, gồm 1 đĩa quay (Disk), thân (Body) và bánh xe di chuyển (Wheel) \cite{Lee2013}}
    \label{fig:fig_Lee2013}
\end{figure}

\begin{enumerate}[label=\alph*.]
    \item Nhóm sử dụng bánh phản lực và bàn xoay: Đây là phương pháp phổ biến nhất, sử dụng mô-men quán tính sinh ra từ việc tăng/giảm tốc độ của đĩa quay (hoặc bánh xe phụ) để triệt tiêu mô-men đổ.
    \begin{itemize}
        \item Thuật toán điều khiển: Để xử lý tính phi tuyến mạnh của hệ thống, các nghiên cứu thường áp dụng mô hình động lực học tách biệt. Han \cite{Han2015} và J. Lee \cite{Lee2013} đã đề xuất sử dụng bộ điều khiển trượt kết hợp với logic mờ để bù trừ sai số mô hình và giảm hiện tượng rung. Trong khi đó, Rosyidi \cite{Rosyidi2016} chứng minh rằng cấu trúc điều khiển PID đa tầng vẫn đảm bảo khả năng cân bằng ở góc nghiêng lớn lên tới $\pm 23 ^ {\circ}$.
        \item Cải tiến cơ khí: Để khắc phục hạn chế về khả năng điều hướng, Rizal et al. đã tích hợp thêm một bàn xoay phía trên thân robot để tạo mô-men xoắn trục đứng, giúp robot di chuyển chính xác từ điểm này sang điểm khác. Đáng chú ý, Geist \cite{Geist2022} đã giới thiệu robot "Wheelbot" với thiết kế đối xứng gồm hai bánh xe (một tiếp đất, một làm bánh phản lực), cho phép robot tự bật dậy từ trạng thái nằm trên sàn, khắc phục nhược điểm không thể tự khởi động của các robot truyền thống.
    \end{itemize}

\begin{figure}[!htbp]
    \centering
    \includegraphics[width=0.2\textwidth]{figures/fig_Shen2020.pdf}
    \caption{Mô hình robot OmBURo sử dụng bánh xe đa hướng chủ động \cite{Shen2020}}
    \label{fig:fig_Shen2020}
\end{figure}

    \item Nhóm sử dụng bánh xe đa hướng chủ động: Khác với phương pháp dùng bánh phản lực bên trong thân, nhóm này thay đổi cấu trúc của bánh xe tiếp đất để tạo lực cân bằng trực tiếp.
    \begin{itemize}
        \item Điển hình là robot OmBURo do Shen và Hong \cite{Shen2020} phát triển. Robot sử dụng bánh xe omni với các con lăn trên vành được dẫn động chủ động thông qua hệ thống trục mềm và bánh răng. Thiết kế này cho phép robot di chuyển ngang tức thời để "đỡ" lấy trọng tâm khi sắp đổ, biến bài toán cân bằng 3D phức tạp thành mô hình hai con lắc ngược 2D tách biệt, giúp robot đứng yên tại chỗ hiệu quả hơn so với cơ chế bánh phản lực.
    \end{itemize}

\begin{figure}[!htbp]
    \centering
    \includegraphics[width=0.3\textwidth]{figures/fig_JHLee2010.pdf}
    \caption{Robot xe đạp một bánh sử dụng khí động học để cân bằng \cite{JHLee2010}}
    \label{fig:fig_JHLee2010}
\end{figure}

    \item Nhóm sử dụng cơ chế Khí động học và Mô phỏng sinh học: Ngoài các phương pháp cơ khí truyền thống, một số nghiên cứu đề xuất các giải pháp sáng tạo khác:
    \begin{itemize}
        \item Khí động học: J.H. Lee \cite{JHLee2010} đã phát triển robot sử dụng hai quạt ducted fan gắn trên thân. Lực đẩy từ luồng khí được sử dụng để tạo mô-men khôi phục cân bằng phương ngang, thay thế cho khối lượng nặng nề của bánh phản lực.
        \item Mô phỏng sinh học: Suzuki \cite{Suzuki2014} thiết kế robot mô phỏng người lái xe đạp một bánh, sử dụng cơ cấu liên kết kín (closed linkage) cho phần chân và một rotor phía trên đầu để mô phỏng cử động vặn người nhằm hỗ trợ cân bằng và điều hướng.
    \end{itemize}
\end{enumerate}

\subsubsection{Robot cân bằng trên bóng (ballbot)}

\begin{figure}[!htbp]
    \centering
    \includegraphics[width=0.3\textwidth]{figures/fig_Abdelrahim2025.pdf}
    \caption{Mô hình robot cân bằng trên bóng trong mặt phẳng 2D (\textit{bên trái}) và không gian 3D (\textit{bên phải}) \cite{Abdelrahim2025}}
    \label{fig:fig_Abdelrahim2025}
\end{figure}

Ballbot là một dạng robot di động cân bằng động trên một bánh xe hình cầu duy nhất \cite{Cai2019}. Khác với các robot di động ổn định tĩnh (nhiều bánh xe) truyền thống vốn cần trọng tâm thấp và chân đế rộng, Ballbot có đặc điểm cao, hẹp và trọng tâm cao \cite{Nagarajan2013}. Nhờ chỉ có một điểm tiếp xúc duy nhất với mặt đất, Ballbot sở hữu khả năng di chuyển đa hướng linh hoạt, cho phép nó di chuyển trực tiếp theo bất kỳ hướng nào mà không cần phải quay đầu trước như các robot hai bánh tự cân bằng (dạng Segway) \cite{Cai2019}. Với diện tích mặt chân đế nhỏ, Ballbot đặc biệt phù hợp để hoạt động trong các môi trường chật hẹp của con người. Tuy nhiên, đây là một hệ thống thiếu cơ cấu chấp hành và không ổn định, đòi hỏi các chiến lược điều khiển tích cực liên tục để duy trì thăng bằng \cite{HYHYHan2014, Pellegrini2011}.

Cơ cấu truyền động là yếu tố cốt lõi để Ballbot giữ thăng bằng và di chuyển. Có hai phương pháp truyền động chính:

\begin{itemize}
    \item Cơ cấu lái chuột bi ngược (Inverse Mouse-Ball Drive): Được tiên phong bởi nhóm nghiên cứu tại Đại học Carnegie Mellon (CMU) với robot "Ballbot" ban đầu. Cơ cấu này sử dụng các con lăn ma sát ép vào quả bóng để truyền mô-men xoắn. Phiên bản đầu tiên gặp vấn đề dao động "nhảy" (hopping motion), nhưng phiên bản cải tiến sử dụng bốn động cơ DC servo điều khiển bốn con lăn chủ động đã khắc phục được điều này \cite{Nagarajan2013}. Bóng được sử dụng có thể là lõi nhôm phủ urethane để tăng độ bền và ma sát \cite{Lauwers2006}.
    
\begin{figure}[!htbp]
    \centering
    \includegraphics[width=0.4\textwidth]{figures/fig_Lauwers2006.pdf}
    \caption{Mô hình thiết kế (a, b) và thực tế (c) của CMU Ballbot \cite{Lauwers2006}}
    \label{fig:fig_Lauwers2006}
\end{figure}

    \item Cơ cấu Bánh xe đa hướng: Đây là thiết kế phổ biến hơn trong các nghiên cứu sau này nhờ khả năng kiểm soát tốt hơn.
    \begin{itemize}
        \item Cấu hình: Thông thường sử dụng 3 bánh xe đa hướng bố trí lệch nhau $120^{\circ}$ \cite{HYHYHan2014}. Góc tiếp xúc của bánh xe với mặt phẳng ngang thường được đặt ở mức $45^{\circ}$ hoặc $50^{\circ}$ để tối ưu hóa việc truyền lực \cite{Kumagai2008, Lima2018}.
        \item Biến thể: Một số thiết kế tập trung vào chi phí thấp và đơn giản hóa bằng cách sử dụng động cơ bước truyền động trực tiếp không hộp số để giảm độ rơ \cite{Kumagai2008}. Các thiết kế hiện đại hơn như robot MiaPURE tập trung vào khả năng chịu tải lớn (lên đến 60-80kg) để chở người hoặc hàng hóa, sử dụng bánh xe omni chịu lực cao và động cơ Quasi-Direct-Drive (QDD) \cite{Xiao2023}.
        \item Cải tiến: Để robot gọn nhẹ hơn, Park đã thiết kế một thân robot mỏng với hộp số hành tinh bậc thang tùy chỉnh, giúp đạt tỷ số truyền cao trong không gian nhỏ hẹp \cite{Park2024}.
    \end{itemize}
\end{itemize}

\begin{figure}[!htbp]
    \centering
    \includegraphics[width=0.2\textwidth]{figures/fig_Kumagai2008.pdf}
    \caption{Robot sử dụng cơ cấu bánh xe đa hướng \cite{Kumagai2008}}
    \label{fig:fig_Kumagai2008}
\end{figure}

Việc mô hình hóa chính xác là bước quan trọng để thiết kế bộ điều khiển.

\begin{itemize}
    \item Phương pháp Lagrange: Hầu hết các nghiên cứu đều sử dụng phương trình Euler-Lagrange để xây dựng mô hình động lực học \cite{Cai2019, Pellegrini2011, Rizal2015, Park2024}.
    \item Mô hình phẳng: Để đơn giản hóa, chuyển động của Ballbot thường được tách thành hai mô hình con lắc ngược bánh xe (WIP) độc lập trên hai mặt phẳng trực giao. Giả định này hợp lý khi góc nghiêng nhỏ và sự ghép nối giữa các mặt phẳng là không đáng kể \cite{Xiao2023, Nagarajan2013, HYHYHan2014}.
    \item Mô hình 3D toàn diện: Pellegrini và các cộng sự \cite{Pellegrini2011} đã phát triển một mô hình 3D chi tiết cho robot "Balancino", bao gồm cả động lực học của động cơ và ma sát xoay. Phân tích tính điều khiển được trên mô hình 3D chỉ ra rằng góc xoay (yaw) của bóng và thân robot không thể điều khiển đồng thời về 0 từ mọi trạng thái, và ma sát xoay là yếu tố cần thiết để đảm bảo tính điều khiển được.
\end{itemize}

Các phương pháp điều khiển đa dạng từ tuyến tính cổ điển đến phi tuyến hiện đại đã được áp dụng:

\begin{itemize}
    \item Điều khiển PID: Là phương pháp đơn giản và phổ biến cho các nguyên mẫu giá rẻ. Han et al. sử dụng các vòng lặp PID độc lập cho trục pitch và roll, với cấu trúc điều khiển vận tốc để tránh hiện tượng bão hòa tích phân \cite{HYHYHan2014}. Kumagai và Ochiai sử dụng lệnh gia tốc thay vì mô-men xoắn cho động cơ bước để tăng tính bền vững \cite{Kumagai2008}.

    \item Điều khiển LQR (Linear Quadratic Regulator): Được sử dụng rộng rãi để tối ưu hóa hiệu suất cân bằng quanh điểm làm việc thẳng đứng. Lauwers và các cộng sự \cite{Lauwers2006} sử dụng LQR với vòng lặp PI bên trong để ổn định vận tốc. Lima và các cộng sự \cite{Lima2018} thiết kế bộ điều khiển LQR rời rạc, sử dụng quy tắc Bryson để tinh chỉnh ma trận trọng số Q và R.

    \item Điều khiển Nâng cao \& Kết hợp:
    \begin{itemize}
        \item LQR-PI: Xiao và các cộng sự \cite{Xiao2023} đề xuất bộ điều khiển tầng LQR-PI, trong đó vòng ngoài LQR tính toán mô-men xoắn tối ưu, còn vòng trong PI xử lý các ma sát phi tuyến và ma sát tĩnh mà mô hình LQR thường bỏ qua. Phương pháp này giúp robot MiaPURE đạt vận tốc tới $2.3 m/s$ và phanh dừng trong 2 giây.
        \item Fuzzy PD: Cai et al. đề xuất bộ điều khiển Fuzzy PD tầng (Cascade Fuzzy PD) để xử lý các vùng phi tuyến ở góc nghiêng lớn, kết hợp với điều khiển vị trí và điều khiển góc xoay \cite{Cai2019}.
        \item Observer \& Feedback: Park sử dụng bộ bù bao gồm một bộ quan sát vòng kín và phản hồi trạng thái đầy đủ để giảm thiểu sai số mô hình \cite{Park2024}.
    \end{itemize}

    \item Ước lượng góc nghiêng: Việc sử dụng bộ lọc Kalman hoặc bộ lọc bù (Complementary Filter) để hợp nhất dữ liệu từ gia tốc kế và con quay hồi chuyển là tiêu chuẩn để có được góc nghiêng chính xác, loại bỏ nhiễu và trôi \cite{HYHYHan2014, Park2024}.
\end{itemize}

Sự ổn định động của Ballbot mở ra các ứng dụng tương tác vật lý thú vị:

\begin{itemize}
    \item Tương tác vật lý (pHRI): Ballbot có khả năng nhường đường khi bị đẩy, giúp nó an toàn hơn khi va chạm so với robot ổn định tĩnh. Một lực đẩy nhẹ có thể di chuyển robot, trong khi một cú đẩy mạnh có thể được hiểu là lệnh để robot tránh đi chỗ khác \cite{Nagarajan2013}.

\begin{figure}[!htbp]
    \centering
    \includegraphics[width=0.4\textwidth]{figures/fig_Xiao2023.pdf}
    \caption{Robot MiaPURE hỗ trợ di chuyển \cite{Xiao2023}}
    \label{fig:fig_Xiao2023}
\end{figure}

    \item Hỗ trợ di chuyển: Các nghiên cứu gần đây như của Xiao \cite{Xiao2023} và Song \cite{Song2023} tập trung vào việc sử dụng Ballbot làm phương tiện di chuyển cá nhân (như xe lăn thế hệ mới).
    \begin{itemize}
        \item Hệ thống MiaPURE có thể chở tải trọng 60kg và người lái có thể điều khiển bằng cách nghiêng người \cite{Xiao2023}.
        \item Nghiên cứu mô phỏng VR của Song giới thiệu giao diện điều khiển rảnh tay dựa trên chuyển động thân người, cho phép cả người dùng xe lăn và người bình thường điều khiển robot hiệu quả qua các môi trường phức tạp \cite{Song2023}.
    \end{itemize}
\end{itemize}

Ballbot đại diện cho một lớp robot di động hứa hẹn với khả năng vận động đa hướng và tương tác an toàn. Các thách thức hiện tại nằm ở việc xử lý ma sát phi tuyến, đảm bảo an toàn khi chở người, và tối ưu hóa cơ khí để giảm kích thước nhưng vẫn đảm bảo mô-men xoắn lớn \cite{Xiao2023, Pellegrini2011}. Xu hướng nghiên cứu đang chuyển dịch từ các mô hình thử nghiệm phòng thí nghiệm sang các ứng dụng thực tế như robot dịch vụ và phương tiện hỗ trợ di chuyển cá nhân.

\subsubsection{Robot dạng cầu (spherical robot)}

Robot dạng cầu là một loại robot di động độc đáo với toàn bộ hệ thống cơ khí, điện tử và nguồn năng lượng được bao bọc hoàn toàn bên trong một vỏ hình cầu kín. Đặc điểm cấu trúc này mang lại cho robot dạng cầu những ưu thế vượt trội so với robot bánh xe hoặc robot chân truyền thống, bao gồm khả năng tự cân bằng, di chuyển đa hướng, và khả năng chống chịu cao với các yếu tố môi trường như bụi bẩn, độ ẩm hoặc sự ăn mòn. Tuy nhiên, do đặc tính tiếp xúc điểm với mặt đất và cơ chế truyền động gián tiếp, robot dạng cầu là một hệ thống phi tuyến tính cao, thiếu cơ cấu chấp hành và chịu các ràng buộc không holonomic, đặt ra nhiều thách thức lớn trong việc thiết kế và điều khiển.

Nguyên lý di chuyển của robot dạng cầu chủ yếu dựa vào việc thay đổi trọng tâm (Barycentric method) hoặc bảo toàn mô-men động lượng. Các thiết kế có thể được phân loại thành ba nhóm chính:

\begin{itemize}
    \item Cơ chế Dịch chuyển Trọng tâm: Đây là phương pháp phổ biến nhất. Robot di chuyển bằng cách thay đổi vị trí của một khối lượng bên trong, làm lệch trọng tâm của cả hệ thống và tạo ra mô-men xoắn làm quay vỏ cầu.
    \begin{itemize}
        \item Hệ thống IDU (Internal Drive Unit): Alves và Dias \cite{Alves2003} đã giới thiệu robot "BYO" sử dụng một đơn vị dẫn động bên trong (IDU) nằm ở đáy quả cầu. Khi IDU di chuyển dọc theo các trục bên trong, nó thay đổi vị trí trọng tâm, khiến robot lăn theo hướng mong muốn.
        \item Con lắc: Li \cite{MLi2014} đề xuất mô hình sử dụng con lắc 2 bậc tự do. Một động cơ điều khiển góc nghiêng của con lắc để lái, trong khi động cơ khác tạo lực đẩy để di chuyển dọc. Tương tự, Yang \cite{Yang2020} cũng sử dụng cơ chế con lắc để điều khiển robot bám theo quỹ đạo, tận dụng tính chất đơn giản và hiệu quả năng lượng của nó.
        \item Thiết kế cơ khí tối ưu: Tholapu \cite{Tholapu2021} tập trung vào việc tối ưu hóa thiết kế cơ khí của hệ thống dịch chuyển trọng tâm để đảm bảo robot có thể di chuyển ổn định trên các địa hình phẳng. Ren \cite{Ren2023} cũng nghiên cứu trên cấu trúc này nhưng tập trung vào việc giải quyết vấn đề thiếu cơ cấu chấp hành của hệ thống con lắc-vỏ cầu.
    \end{itemize}

\begin{figure}[!htbp]
    \centering
    \includegraphics[width=0.4\textwidth]{figures/fig_Ren2023.pdf}
    \caption{Mô hình robot hình cầu \cite{Ren2023}}
    \label{fig:fig_Ren2023}
\end{figure}

    \item Cơ chế Bánh đà/Rotor: Khác với việc dịch chuyển khối lượng tịnh tiến, nhóm này sử dụng nguyên lý bảo toàn mô-men động lượng. Joshi \cite{Joshi2009} đã phân tích một thiết kế robot cầu phức tạp sử dụng 4 rotor (bánh đà) bố trí bên trong. Bằng cách tăng/giảm tốc độ quay của các rotor này, mô-men phản lực được tạo ra để điều khiển robot lăn theo bất kỳ hướng nào.
    
    \item Cơ chế Biến đổi Cấu trúc: Để khắc phục hạn chế về khả năng vượt chướng ngại vật của vỏ cầu cứng, Chen \cite{Chen2013} đã đề xuất một loại robot cầu có cấu trúc biến đổi. Robot này có thể mở rộng hoặc thu nhỏ đường kính vỏ cầu nhờ các cánh tay đòn bên trong, giúp nó leo qua các vật cản cao hơn bán kính của chính nó hoặc chui qua khe hẹp.
\end{itemize}

Việc xây dựng mô hình toán học chính xác là bước tiên quyết để điều khiển robot cầu.

\begin{itemize}
    \item Phương pháp Lagrange: Hầu hết các nghiên cứu đều sử dụng phương trình Euler-Lagrange để mô tả động lực học hệ thống. Sang \cite{Sang2010} đã xây dựng mô hình động lực học cho robot cầu dựa trên năng lượng hệ thống, xem xét các ràng buộc lăn không trượt. J. Chen \cite{JChen2016} mở rộng phân tích động lực học cho robot cầu hoạt động trong môi trường dưới nước, nơi các lực thủy động lực học (lực cản, lực nổi) đóng vai trò quan trọng bên cạnh động lực học vật rắn thông thường.
    \item Ràng buộc Không Holonomic: Đặc điểm lăn không trượt của robot cầu tạo ra các ràng buộc không holonomic (không thể tích phân trực tiếp để ra vị trí). Điều này được nhấn mạnh trong các nghiên cứu của Alves \cite{Alves2003} và Joshi \cite{Joshi2009}, làm cho việc lập kế hoạch quỹ đạo trở nên phức tạp hơn so với robot bánh xe thông thường.
\end{itemize}

Các phương pháp điều khiển đa dạng từ kinh điển đến hiện đại đã được áp dụng để giải quyết vấn đề ổn định và bám quỹ đạo:

\begin{itemize}
    \item Điều khiển bám quỹ đạo: M. Li \cite{MLi2014} đã phát triển bộ điều khiển để robot con lắc 2-DOF bám theo các quỹ đạo cong phức tạp, chứng minh tính ổn định thông qua mô phỏng số. Yang \cite{Yang2020} đề xuất một luật điều khiển phi tuyến tính giúp robot cầu bám quỹ đạo mong muốn với sai số nhỏ nhất, ngay cả khi có nhiễu động.
    \item Điều khiển bền vững: Đối mặt với các bất định của mô hình và nhiễu môi trường, Ren \cite{Ren2023} đã nghiên cứu các phương pháp điều khiển tiên tiến (như điều khiển trượt - Sliding Mode Control hoặc các biến thể) để đảm bảo robot hoạt động ổn định.
    \item Điều khiển chuyển động Holonomic: Joshi \cite{Joshi2009} đã phát triển thuật toán điều khiển cho hệ thống 4 rotor, cho phép robot thực hiện các chuyển động quay và tịnh tiến độc lập, tận dụng tối đa khả năng di chuyển đa hướng.
\end{itemize}

Ngoài khả năng di chuyển, các nhà nghiên cứu còn tích hợp thêm các chức năng để robot cầu trở nên hữu dụng hơn:

\begin{itemize}
    \item Ổn định Camera: Một nhược điểm lớn của robot cầu là vỏ quay liên tục khiến camera gắn trên đó bị chóng mặt. Chadil \cite{Chadil2011} đã giải quyết vấn đề này bằng cách thiết kế một cơ cấu ổn định bên trong, sử dụng đối trọng để giữ cho camera luôn hướng về một phía cố định hoặc điều khiển được góc nhìn độc lập với chuyển động lăn của vỏ cầu.
    \item Hoạt động đa môi trường: J. Chen \cite{JChen2016} đã nghiên cứu khả năng hoạt động của robot cầu trong môi trường nước, mở rộng phạm vi ứng dụng từ trên cạn xuống dưới nước.
\end{itemize}

Từ các bài báo, có thể thấy robot dạng cầu là một lĩnh vực nghiên cứu sôi động. Xu hướng phát triển đang chuyển dịch từ các thiết kế cơ khí đơn giản (con lắc đơn) sang các hệ thống phức tạp hơn (biến đổi cấu trúc, rotor đa hướng) và từ các thuật toán điều khiển cơ bản sang các phương pháp điều khiển phi tuyến, bền vững. Khả năng tích hợp các hệ thống quan sát ổn định và khả năng biến hình vượt chướng ngại vật là những hướng đi hứa hẹn để đưa robot cầu vào ứng dụng thực tế như thám hiểm, giám sát và cứu hộ.

\subsubsection{Robot đơn bánh vành lớn (monowheel)}

Robot đơn bánh vành lớn (monowheel) là một dạng phương tiện hoặc robot di động đặc biệt, trong đó toàn bộ các cơ cấu điều khiển, nguồn năng lượng, cảm biến và tải trọng đều được bố trí bên trong một bánh xe lớn \cite{Gim2024}. Cấu trúc này mang lại ưu điểm vượt trội về khả năng di chuyển trên nhiều loại địa hình, tính nhỏ gọn và hiệu quả năng lượng cao so với các hệ thống nhiều bánh hoặc chân \cite{Cieslak2011, Buratowski2012}.

Tuy nhiên, thách thức lớn nhất của loại robot này là tính không ổn định vốn có. Khác với robot xe đạp một bánh có trọng tâm cao hơn trục bánh xe và hoạt động như một con lắc ngược cần cân bằng liên tục, robot monowheel có trọng tâm thấp hơn trục quay (nhờ phân bố khối lượng bên trong), tạo ra sự ổn định tự nhiên theo phương dọc giống như nguyên lý con lắc. Mặc dù vậy, vấn đề nghiêm trọng nằm ở sự ổn định ngang \cite{Gim2024}. Khi robot đứng yên hoặc di chuyển chậm, hiệu ứng con quay hồi chuyển chưa đủ lớn để giữ thăng bằng, khiến robot dễ bị đổ sang hai bên nếu không có cơ chế điều khiển chủ động \cite{Cieslak2011, Buratowski2012}.

Có hai hướng tiếp cận chính trong thiết kế cơ khí cho robot monowheel:

\begin{enumerate}
    \item Cơ cấu cân bằng sử dụng cánh tay đòn: Đây là phương pháp được nhóm nghiên cứu từ ĐH AGH (Cieslak, Buratowski) đề xuất \cite{Cieslak2011, Buratowski2012}.
    \begin{itemize}
        \item Cấu tạo: Robot bao gồm một bánh xe lớn bên ngoài và một thân xe di chuyển tự do bên trong nhờ các con lăn. Cơ cấu cân bằng bao gồm một cánh tay đòn hình cung có gắn vật đối trọng, được điều khiển bởi động cơ servo để xoay sang trái hoặc phải.
        \item Nguyên lý: Việc xoay cánh tay đòn sẽ làm thay đổi trọng tâm (CoM) của toàn bộ robot theo phương ngang, tạo ra mô-men xoắn để giữ thăng bằng hoặc điều hướng. Động cơ chính sẽ truyền động qua hệ thống bánh răng để làm quay bánh xe lớn, tạo chuyển động tịnh tiến.
    \end{itemize}

    \item Cơ cấu Robot Lai Chân-Bánh (Ringbot): Đây là thiết kế tiên tiến được Gim và Kim (2024) giới thiệu \cite{Gim2024}.
    \begin{itemize}
        \item Cấu tạo: Thay vì một khối thân xe duy nhất, Ringbot sử dụng hai mô-đun lái riêng biệt di chuyển bên trong vành bánh xe. Đặc biệt, mỗi mô-đun được tích hợp một chân robot 3 bậc tự do (3-DoF).
        \item Nguyên lý: Hai mô-đun này có thể thay đổi vị trí dọc theo vành bánh để tạo mô-men xoắn cho chuyển động lăn (tương tự việc thay đổi trọng tâm dọc). Các chân robot không chỉ đóng vai trò cân bằng bằng cách dịch chuyển trọng tâm ngang mà còn có thể chạm đất để thực hiện các tác vụ phức tạp như tự đứng dậy, xoay tại chỗ, hoặc bước qua vật cản.
    \end{itemize}
\end{enumerate}

Các nghiên cứu đều khẳng định robot đơn bánh chịu sự ràng buộc không holonomic, nghĩa là robot không thể trượt ngang mà chỉ có thể lăn theo hướng của bánh xe.

\begin{itemize}
    \item Động lực học phương dọc: Chuyển động tịnh tiến được tạo ra do sự thay đổi trọng tâm. Khi các cơ cấu bên trong (thân xe hoặc mô-đun lái) tiến lên phía trước vành bánh xe, trọng lực sẽ tạo ra mô-men xoắn làm bánh xe lăn về phía trước để đưa trọng tâm về vị trí thấp nhất.
    \item Động lực học phương ngang: Đây là bài toán phức tạp nhất, được mô hình hóa như một con lắc ngược kép với khớp đầu tiên (điểm tiếp xúc đất) không có truyền động.
    \begin{itemize}
        \item Khi di chuyển nhanh: Hiệu ứng con quay hồi chuyển sinh ra từ mô-men động lượng của bánh xe giúp ổn định robot. Tốc độ tiến động tỉ lệ nghịch với mô-men động lượng, nghĩa là bánh xe quay càng nhanh thì khả năng rẽ càng chậm lại với cùng một lực tác động nghiêng.
        \item Khi di chuyển chậm hoặc đứng yên: Hiệu ứng con quay biến mất. Robot của nhóm AGH sử dụng mô hình tuyến tính hóa quanh điểm cân bằng để điều khiển cánh tay đòn giữ thăng bằng. Trong khi đó, Ringbot sử dụng chân để tương tác trực tiếp với mặt đất hoặc dịch chuyển trọng tâm chân để giữ thăng bằng tĩnh.
    \end{itemize}
\end{itemize}

Do tính chất phi tuyến tính và không ổn định, các chiến lược điều khiển phức tạp đã được áp dụng:

\begin{itemize}
    \item Ước lượng trạng thái: Cả hai hệ thống đều sử dụng bộ lọc Kalman để hợp nhất dữ liệu từ cảm biến gia tốc và con quay hồi chuyển nhằm ước lượng chính xác góc nghiêng và vận tốc góc, loại bỏ nhiễu và sai số trôi.
    \item Bộ điều khiển:
    \begin{itemize}
        \item Nhóm AGH (Cieslak/Buratowski): Đã thử nghiệm PID và Fuzzy Logic nhưng không thành công. Cuối cùng, họ phát triển một bộ điều khiển tùy chỉnh dựa trên công thức toán học có tính đến giá trị điều khiển trước đó và các yếu tố phi tuyến, giúp robot tự cân bằng khi đứng yên hoặc di chuyển chậm ($< 0.3 m/s$).
        \item Ringbot (Gim \& Kim): Sử dụng bộ điều khiển tách biệt cho tốc độ và góc lái.
        \begin{itemize}
            \item Điều khiển lái: Sử dụng cấu trúc vòng lặp: vòng ngoài điều khiển góc yaw, vòng trong điều khiển góc nghiêng thông qua việc dịch chuyển chân.
            \item Máy trạng thái hữu hạn (Finite State Machine - FSM): Ringbot sử dụng FSM để chuyển đổi giữa các chế độ: Lái, Hồi phục khi ngã, Tự đứng dậy, và Xoay tại chỗ.
        \end{itemize}
    \end{itemize}
\end{itemize}

Các thử nghiệm thực tế cho thấy robot của nhóm AGH có thể chịu được các lực tác động bên ngoài (như đẩy tay) và tự cân bằng lại khi đứng yên, tuy nhiên sẽ thất bại nếu lực vượt quá ngưỡng tới hạn. Ringbot thể hiện ưu thế vượt trội nhờ kích thước bánh xe lớn (đường kính > 500mm) và sự hỗ trợ của chân. Nó có thể vượt qua các vật cản cao tới 17.5\% đường kính bánh xe (tương đương gờ giảm tốc hoặc vỉa hè). Ringbot có khả năng tự đứng dậy sau khi ngã và xoay tròn tại chỗ nhờ chân, khắc phục nhược điểm không thể quay đầu khi đứng yên của các monowheel truyền thống.

Tổng hợp từ các bài báo cho thấy sự tiến hóa của robot đơn bánh từ các thiết kế sử dụng đối trọng bên trong đơn thuần (Cieslak/Buratowski) để giải quyết bài toán cân bằng tĩnh, sang các thiết kế lai ghép phức tạp (Ringbot) sử dụng chân robot để tăng cường khả năng cơ động và tương tác môi trường. Các nghiên cứu đều chỉ ra tiềm năng lớn của loại robot này trong các ứng dụng giao thông cá nhân hoặc giao hàng chặng cuối trong môi trường đô thị chật hẹp.

% ======================================================================

% ----- ROBOT HAI BÁNH -----

\subsection{Robot hai bánh}

Trong lĩnh vực robot di động, các robot hai bánh thường được phân loại thành hai nhóm cấu trúc chính dựa trên nguyên lý hoạt động và phương pháp giữ thăng bằng: Robot kiểu xe đạp (Bicycle-type robot) và Robot kiểu con lắc ngược (Inverted-pendulum-type robot).

\begin{figure}[!htbp]
    \centering
    \includegraphics[width=0.4\textwidth]{figures/fig_robot_hai_banh.pdf}
    \caption{Cấu trúc chung của (a) Robot kiểu xe đạp và (b) Robot kiểu con lắc ngược \cite{chung2008}}
    \label{fig:fig_robot_hai_banh}
\end{figure}

% ======================================================================

% ----- ROBOT KIỂU XE ĐẠP -----

\subsubsection{Robot kiểu xe đạp}

Robot kiểu xe đạp là một lớp đặc biệt trong các phương tiện di động hai bánh, thu hút sự quan tâm lớn của cộng đồng nghiên cứu nhờ khả năng di chuyển linh hoạt, tiết kiệm năng lượng và kích thước nhỏ gọn. Tuy nhiên, từ góc độ động lực học hệ thống, robot xe đạp thuộc loại phương tiện "con lắc ngược có bánh xe", vốn dĩ không ổn định về mặt tĩnh học theo phương ngang và thể hiện các hành vi động lực học phi tuyến tính phức tạp. Các nghiên cứu gần đây tập trung vào việc giải quyết bài toán ổn định hệ thống dưới các điều kiện bất định về trọng tâm, biến thiên tốc độ, nhiễu loạn từ môi trường và tối ưu hóa tính toán, cũng như mở rộng ứng dụng sang các môi trường công nghiệp đặc thù.

\begin{figure}[!htbp]
    \centering
    \includegraphics[width=0.3\textwidth]{figures/fig_Chiu2020.pdf}
    \caption{Mô hình tổng quát của robot kiểu xe đạp \cite{Chiu2020}}
    \label{fig:fig_Chiu2020}
\end{figure}

Phương pháp phổ biến nhất để duy trì sự cân bằng cho robot xe đạp khi di chuyển là điều chỉnh góc lái, tận dụng lực ly tâm để chống lại mô-men trọng trường gây đổ. Tuy nhiên, thách thức lớn nằm ở việc xử lý sự thay đổi vận tốc và các sai số mô hình.

Yeh và cộng sự \cite{Yeh2019} đã đề xuất một bộ điều khiển cân bằng mạnh mẽ dựa trên mô hình thay đổi tham số tuyến tính (Linear Parameter Varying - LPV). Thay vì giải một họ các bất đẳng thức ma trận tuyến tính (LMI) vô hạn, nhóm nghiên cứu đã phân rã mô hình động lực học thành một tổ hợp lồi của bốn hệ thống con tuyến tính với các hệ số thay đổi theo thời gian. Bộ điều khiển đề xuất cũng có cấu trúc tương tự như luật phản hồi trạng thái mờ cho các mô hình TSK. Kết quả mô phỏng và thực nghiệm cho thấy phương pháp này cho phép robot duy trì ổn định ngang ngay cả khi tăng tốc từ $1.5m/s$ lên $10m/s$, vượt trội hơn so với bộ điều khiển LQR truyền thống vốn dễ mất ổn định khi vận tốc thay đổi.

\begin{figure}[!htbp]
    \centering
    \includegraphics[width=0.25\textwidth]{figures/fig_Huang2017.pdf}
    \caption{Mô hình robot hình người đi xe đạp \cite{Huang2017}}
    \label{fig:fig_Huang2017}
\end{figure}

Trong một hướng tiếp cận khác, Huang và cộng sự \cite{Huang2017} tập trung vào vấn đề bất định của trọng tâm (Center of Gravity - COG) khi robot hình người cưỡi xe đạp. Sự mất cân bằng khối lượng (do lắp ráp hoặc tải trọng) tạo ra một sai số trong phép đo góc nghiêng, khiến các bộ điều khiển PD thông thường dẫn đến sai số xác lập, làm xe bị lệch hướng. Nhóm tác giả đã phát triển một khung điều khiển có khả năng ước lượng trực tuyến các sai lệch này và triệt tiêu chúng. Kết quả thực nghiệm với robot hình nhân nhỏ cho thấy hệ thống có thể tự động bù đắp cho việc lệch trọng tâm (ví dụ: khi gắn thêm vật nặng 50g vào tay lái) và duy trì đường đi thẳng ổn định.

Đối với các ứng dụng yêu cầu robot đứng yên hoặc di chuyển chậm, cơ chế lái không tạo đủ lực ly tâm để cân bằng. Do đó, việc sử dụng bánh đà hoặc con lắc phản lực để tạo mô-men cân bằng là giải pháp tối ưu.

Chiu và Wu \cite{Chiu2020} đã phát triển một hệ thống điều khiển bám sát backstepping thông minh và mạnh mẽ (Robust Intelligent Backstepping Tracking Control - RIBTC) cho robot xe đạp. Do mô hình toán học chính xác của hệ thống rất khó xác định do ma sát và độ rơ cơ khí, nhóm tác giả sử dụng bộ điều khiển thích nghi mạng nơ-ron CMAC loại TSK hồi quy đầu ra (AORTSKCMAC) để mô phỏng bộ điều khiển backstepping lý tưởng. Đồng thời, một bộ điều khiển bền vững $H^\infty$ được tích hợp để giảm thiểu ảnh hưởng của các sai số xấp xỉ và nhiễu loạn bên ngoài. Thực nghiệm trên robot xe đạp cho thấy phương pháp này có khả năng phục hồi cân bằng nhanh chóng (trong khoảng 2 giây) khi chịu tác động ngoại lực ngang, vượt trội so với các phương pháp PID, Fuzzy hay mạng nơ-ron Elman truyền thống.

Tuy nhiên, các bộ điều khiển cấp cao (Higher Order Controller - HOC) như $H^\infty$ thường dẫn đến độ phức tạp tính toán lớn, gây khó khăn cho việc triển khai trên phần cứng thời gian thực. Kumar và Suman \cite{Kumar2021} đã giải quyết vấn đề này bằng phương pháp mô hình hóa giảm bậc (Reduced Order Modelling - MOR). Họ áp dụng kỹ thuật xấp xỉ nhiễu loạn đơn lẻ cân bằng (Balanced Singular Perturbation Approximation - BSPA), là sự kết hợp giữa cắt cụt cân bằng và xấp xỉ nhiễu loạn đơn lẻ. Phương pháp này cho phép giảm bộ điều khiển từ bậc 6 xuống bậc 3 mà vẫn giữ lại các đặc tính quan trọng của hệ thống gốc, bao gồm cả độ lợi DC. Kết quả mô phỏng chứng minh rằng bộ điều khiển bậc 3 (Reduced Order Controller - ROC) hoạt động hiệu quả tương đương với bộ điều khiển bậc cao ngay cả khi có sự thay đổi tham số tải trọng và tốc độ bánh đà.

Vượt ra ngoài việc di chuyển trên mặt phẳng, cấu hình robot kiểu xe đạp còn được ứng dụng trong các nhiệm vụ kiểm tra kết cấu thép phức tạp. Nguyễn Sơn Thanh \cite{Nguyen2022} đã giới thiệu một robot leo trèo kiểu xe đạp linh hoạt (Agile Bicycle-like Robot) sử dụng hai bánh xe từ tính độc lập để bám dính vào các bề mặt sắt từ. Điểm đột phá trong thiết kế này là cơ chế tịnh tiến khứ hồi và các khớp động, cho phép robot thay đổi hình dạng (khoảng cách trục bánh xe) để vượt qua các chướng ngại vật khó như góc nhọn, cạnh mỏng hoặc dầm chữ T, chữ L. Robot được trang bị cảm biến siêu âm để đo độ dày thép và có khả năng di chuyển đa hướng trên các cấu trúc cầu và đường ống phức tạp mà các robot bánh xích hoặc robot leo trèo truyền thống khó tiếp cận.

\begin{figure}[!htbp]
    \centering
    \includegraphics[width=0.4\textwidth]{figures/fig_Nguyen2022.pdf}
    \caption{Mô hình robot kiểu xe đạp có thể bám bề mặt \cite{Nguyen2022}}
    \label{fig:fig_Nguyen2022}
\end{figure}

Tổng hợp từ các nghiên cứu trên cho thấy sự phát triển đa dạng của robot kiểu xe đạp. Về mặt điều khiển, xu hướng chuyển dịch từ các bộ điều khiển tuyến tính cổ điển sang các phương pháp phi tuyến thích nghi (như TSK-fuzzy, Neural Networks) và tối ưu hóa tính toán (Reduced Order Modelling) để đối phó với tính bất định và phi tuyến của hệ thống. Về mặt thiết kế cơ khí, sự kết hợp giữa cơ chế lái, bánh đà và bánh xe từ tính linh hoạt đã mở rộng phạm vi hoạt động của robot từ mặt đất phẳng sang các cấu trúc 3D phức tạp, phục vụ hiệu quả cho các nhiệm vụ kiểm tra và giám sát tự động.

% ======================================================================

% ----- ROBOT KIỂU CON LẮC NGƯỢC -----

\subsubsection{Robot kiểu con lắc ngược}















% ======================================================================

% ----- ROBOT BA BÁNH -----

\subsection{Robot ba bánh}

\begin{figure}[!htbp]
    \centering
    \includegraphics[width=0.5\textwidth]{figures/fig_robot_ba_banh.pdf}
    \caption{Năm cấu hình thiết kế robot ba bánh phổ biến: (a) Robot dẫn động vi sai, (b) Robot dẫn động đồng bộ, (c) Robot toàn hướng sử dụng bánh Mecanum, (d) Robot toàn hướng sử dụng bánh xe dẫn hướng chủ động, (e) Robot toàn hướng sử dụng cơ chế lái chủ động \cite{chung2008}}
    \label{fig:fig_robot_ba_banh}
\end{figure}

Trong lĩnh vực robot di động, cấu trúc ba bánh là một trong những nền tảng thiết kế phổ biến nhất nhờ vào hai ưu điểm cốt lõi: sự ổn định tĩnh và cấu trúc cơ khí đơn giản. Ưu điểm về mặt cân bằng giúp robot duy trì tư thế ổn định trên mặt phẳng mà không cần các thuật toán điều khiển phức tạp như robot hai bánh. Hơn nữa, sự linh hoạt của cấu trúc ba bánh cho phép tích hợp hầu hết các loại bánh xe đã được giới thiệu ở các phần trước để tạo ra các đặc tính di chuyển khác nhau. Dựa trên cách bố trí và loại bánh xe sử dụng, phần này sẽ tập trung phân tích 5 cấu hình thiết kế tiêu biểu nhất (được minh họa trong Hình \ref{fig:fig_robot_ba_banh}):

% ======================================================================

% ----- ROBOT DẪN ĐỘNG VI SAI -----

\subsubsection{Robot dẫn động vi sai}

\begin{figure}[!htbp]
    \centering
    \includegraphics[width=0.4\textwidth]{figures/fig6.pdf}
    \caption{Mô hình robot dẫn động vi sai \cite{siegwart2011}}
    \label{fig:fig6}
\end{figure}

% ======================================================================

% ----- ROBOT DẪN ĐỘNG ĐỒNG BỘ -----

\subsubsection{Robot dẫn động đồng bộ}

% ======================================================================

% ----- ROBOT TOÀN HƯỚNG SỬ DỤNG BÁNH MECANUM -----

\subsubsection{Robot toàn hướng sử dụng bánh Mecanum}

% ======================================================================

% ----- ROBOT TOÀN HƯỚNG SỬ DỤNG BÁNH XE DẪN HƯỚNG CHỦ ĐỘNG -----

\subsubsection{Robot toàn hướng sử dụng bánh xe dẫn hướng chủ động}

% ======================================================================

% ----- ROBOT TOÀN HƯỚNG SỬ DỤNG CƠ CHẾ LÁI CHỦ ĐỘNG -----

\subsubsection{Robot toàn hướng sử dụng cơ chế lái chủ động}

% ======================================================================

% ----- CẤU TRÚC LÁI BA BÁNH -----

\subsection{Cấu trúc lái ba bánh (\textit{Steered Tricycle})}

Cấu trúc lái ba bánh là nền tảng kinh điển trong robot di động mặt đất, đặc trưng bởi sự tách biệt giữa chức năng lái và dẫn động. Dựa trên phân tích tổng hợp từ các nghiên cứu hiện đại, cấu trúc này được phân loại và đánh giá qua ba khía cạnh: Cấu hình hình học cơ khí, Kiến trúc phần cứng nhúng và Cấu trúc hệ thống điều khiển.

\subsubsection{Biến thể cấu hình cơ khí}

\begin{figure}[!htbp]
    \centering
    \includegraphics[width=0.4\textwidth]{figures/fig_Widiawan2020.pdf}
    \caption{Sơ đồ động học của robot di động ba bánh cấu hình Delta. Trong đó, góc lái $\gamma$ điều khiển vận tốc góc quay $\dot{\psi}$ so với Hệ quy chiếu Toàn cục (GRF). \cite{Widiawan2020}.}
    \label{fig:fig_Widiawan2020}
\end{figure}

Sự ổn định động lực học của robot phụ thuộc hoàn toàn vào cách bố trí khung gầm. Hai biến thể chính hiện nay là:

\begin{itemize}
    \item Cấu hình Delta (1F/2R): Đây là thiết kế tiêu chuẩn với một bánh lái phía trước và hai bánh sau cố định. Mặc dù có ưu điểm về cơ cấu lái servo đơn giản \cite{Widiawan2020}, các nghiên cứu của Vázquez-Rodríguez và các cộng sự \cite{VazquezRodriguez2018} và Licea và các cộng sự \cite{Licea2018} chỉ ra rằng cấu trúc này có "tam giác hỗ trợ" hẹp về phía trước, khiến robot cực kỳ dễ bị lật (rollover) khi phanh gấp hoặc vào cua tốc độ cao. Để khắc phục, Jaskot và Posiadała \cite{jaskot2018} nhấn mạnh việc phải đưa các thành phần lực ma sát trượt vào mô hình cấu trúc động lực học thay vì chỉ dùng mô hình hình học cứng nhắc, giúp dự đoán sớm các trạng thái mất ổn định.
    \item Cấu hình Tadpole (2F/1R): Để giải quyết vấn đề ổn định của Delta, cấu hình Tadpole (hai bánh trước, một bánh sau) được ưu tiên trong các ứng dụng tốc độ cao \cite{VazquezRodriguez2018}. Cấu trúc này yêu cầu cơ cấu lái Ackermann phức tạp hơn để đồng bộ hóa hai bánh trước, nhưng đổi lại, trọng tâm robot luôn nằm trong vùng hỗ trợ rộng nhất khi phanh, giúp giảm thiểu chỉ số lật (Rollover Index - RI) \cite{Licea2018}.
\end{itemize}

\subsubsection{Kiến trúc Phần cứng và Tích hợp Cảm biến}

Cấu trúc robot hiện đại không chỉ là khung sắt mà là một hệ thống cơ điện tử tích hợp.

\begin{itemize}
    \item Kiến trúc Phân cấp (Hierarchical Architecture): Widiawan và các cộng sự \cite{Widiawan2020} đề xuất mô hình phần cứng phân cấp để khắc phục độ trễ truyền dẫn. Hệ thống sử dụng Raspberry Pi cho các tác vụ bậc cao (xử lý ảnh/IoT) và vi điều khiển ATMega328 để điều khiển động cơ thời gian thực.
    \item Tích hợp Đa cảm biến: Do cấu trúc ba bánh dễ bị trượt ngang, phép đo hành trình (odometry) từ encoder thường thiếu chính xác. Chand \cite{chand2022} đã đề xuất tích hợp La bàn điện tử vào cấu trúc phần cứng. Dữ liệu hướng từ la bàn được hợp nhất với encoder thông qua bộ lọc Kalman, giúp robot duy trì định vị chính xác trong môi trường trong nhà.
    \item Mô hình hóa Không gian An toàn: Trong môi trường công nghiệp, Subash và các cộng sự \cite{subash2024} đề xuất thay thế cấu trúc khung bao hình chữ nhật bằng các "hình tròn bao phủ" (covering circles). Cấu trúc ảo này cho phép robot tận dụng khả năng bẻ lái linh hoạt của bánh trước để di chuyển trong không gian hẹp mà không va chạm.
\end{itemize}

\subsubsection{Cấu trúc Hệ thống Điều khiển}

Để vận hành cấu trúc cơ khí trên, các kiến trúc điều khiển hiện đại đã chuyển dịch từ PID cổ điển sang các cấu trúc thông minh và thích nghi.

\begin{itemize}
    \item Cấu trúc Điều khiển Tầng (Cascaded Control): Kamga và Rachid \cite{kamga1996} đã đặt nền móng với cấu trúc điều khiển tách biệt cho vận tốc và góc lái. Phát triển trên nền tảng đó, Precup và các cộng sự \cite{Precup2006} đề xuất kiến trúc điều khiển Mờ (Fuzzy Control) kết hợp với trường thế năng nhân tạo, giúp robot xử lý các hành vi phi tuyến của cấu trúc ba bánh khi tránh vật cản.
    \item Cấu trúc Thích nghi và Tối ưu: Để đối phó với sự thay đổi trọng tâm và ma sát, Pătraşcu và Gheorghe \cite{patrascu2021} sử dụng kiến trúc điều khiển P-D được tinh chỉnh online bằng Giải thuật Di truyền (Genetic Algorithms). Đồng thời, Ketemaw và Seid [10] đề xuất kiến trúc điều khiển Trượt (Sliding Mode Control) tích hợp Logic mờ để giám sát chỉ số RI, tự động kích hoạt phanh vi sai cầu sau (rear differential braking) để chống lật chủ động.
\end{itemize}

Cấu trúc lái ba bánh hiện đại là sự hội tụ của thiết kế cơ khí tối ưu (ưu tiên Tadpole cho tốc độ cao), phần cứng nhúng phân cấp (Raspberry Pi/Vi điều khiển) và các thuật toán điều khiển thông minh. Sự kết hợp này cho phép khắc phục các nhược điểm vật lý vốn có (như dễ lật, dễ trượt), biến nó thành nền tảng tin cậy cho nông nghiệp thông minh và logistics tự động.

% ======================================================================
% ======================================================================

% ----- ROBOT BỐN BÁNH -----

\subsection{Robot bốn bánh}

% ======================================================================

% ----- CẤU TRÚC LÁI ACKERMANN -----

\subsubsection{Cấu trúc lái Ackermann}










% ======================================================================

% ----- CẤU TRÚC DẪN ĐỘNG TRƯỢT -----

\subsubsection{Cấu trúc dẫn động trượt}
Cấu trúc này sử dụng ma sát trượt để thay đổi hướng, tương tự như các phương tiện bánh xích.
Thường sử dụng bốn (hoặc sáu, tám) bánh xe cố định, được gắn chặt vào khung robot và không thể thay đổi hướng. Tất cả các bánh xe ở mỗi bên (trái và phải) thường được kết nối và chủ động đồng thời.

Robot rẽ bằng cách tạo ra chênh lệch vận tốc giữa hai phía của robot, giống hệt như Cấu trúc Dẫn động Vi sai. Ví dụ, để rẽ trái, các bánh xe bên phải quay nhanh hơn bên trái (hoặc bên trái quay lùi). Sự chênh lệch vận tốc này buộc các bánh xe phải trượt (skid) trên mặt đất để robot có thể xoay [..].

Về mặt điều khiển, nó được xem như một hệ thống phi đa hướng tương tự Dẫn động Vi sai, có khả năng xoay tại chỗ (zero turning radius). Tuy nhiên, mô hình động học của nó phức tạp hơn nhiều do sự trượt không thể đoán trước, phụ thuộc rất nhiều vào ma sát và bề mặt địa hình.

\begin{itemize}
    \item Ưu điểm: Cấu trúc cơ khí cực kỳ đơn giản và bền bỉ (không có cơ cấu lái). Rất phù hợp cho địa hình gồ ghề.
    \item Nhược điểm: Tiêu thụ năng lượng rất lớn do ma sát trượt. Gây mài mòn bánh xe và làm hỏng các bề mặt mềm. Việc ước tính vị trí (Odometry) rất không chính xác do độ trượt cao [..]. 
\end{itemize}

\begin{figure}[!htbp]
    \centering
    \includegraphics[width=0.4\textwidth]{figures/fig500.pdf}
    \caption{Mô hình động học trên mặt phẳng của cấu trúc dẫn động trượt bốn bánh, hiển thị các tâm quay tức thời (ICR) của thân xe ($ICR_v$) và hai bên bánh xe ($ICR_l$, $ICR_r$). Các tâm quay này nằm bên ngoài thân robot do hiện tượng trượt \cite{mandow2007}.}
    \label{fig:fig500}
\end{figure}

Mô hình Động học và Tâm quay tức thời: Khác với robot dẫn động vi sai lý tưởng (nơi bánh xe lăn không trượt), chuyển động của robot dẫn động trượt phụ thuộc vào sự tương tác phức tạp giữa lốp và mặt đất. Để mô tả động học của nó, các nghiên cứu hiện đại sử dụng khái niệm Tâm quay tức thời \cite{mandow2007}.

Mô hình động học được xấp xỉ tương đương với một robot vi sai mở rộng:

\begin{equation} \label{eq9}
    \begin{split}
        \begin{bmatrix}
            v_x \\
            \omega_z
        \end{bmatrix}
        =
        \begin{bmatrix}
            \dfrac{R}{2} & \dfrac{R}{2} \\
            -\dfrac{R}{2y_{ICR}} & \dfrac{R}{2y_{ICR}}
        \end{bmatrix}
        \begin{bmatrix}
            \omega_L \\
            \omega_R
        \end{bmatrix}
    \end{split}
\end{equation}

Trong đó:
\begin{itemize}
    \item $v_x$, $\omega_z$: Vận tốc dài và vận tốc góc của robot.
    \item $\omega_L$, $\omega_R$: Vận tốc góc của bánh trái và phải.
    \item $y_{ICR}$: Toạ độ ngang của tâm quay tức thời bánh xe (hay còn gọi là bán kính trục xe hiệu dụng).
\end{itemize}

\begin{figure}[!htbp]
    \centering
    \includegraphics[width=0.4\textwidth]{figures/fig499.pdf}
    \caption{Sơ đồ vùng tiếp xúc thể hiện sự tương đồng động học giữa các cấu trúc: (a) Robot vi sai tiêu chuẩn, (b) Robot bánh xích, (c) Robot dẫn động trượt 4 bánh, (d) Robot dẫn động trượt 6 bánh \cite{mandow2007}.}
    \label{fig:fig499}
\end{figure}

Quan sát Hình \ref{fig:fig499}, ta thấy mối liên hệ: Cấu trúc bánh xích (b) và cấu trúc dẫn động trượt (c, d) thực chất có thể được coi là các biến thể mở rộng của cấu trúc vi sai (a) với vùng tiếp xúc mặt đất được kéo dài.

\begin{itemize}
    \item Ở robot vi sai (a), vùng tiếp xúc là điểm, ma sát xoay là tối thiểu.
    \item Ở các cấu trúc còn lại (b, c, d), vùng tiếp xúc là một dải (continuous tread) hoặc một chuỗi các điểm rời rạc.
\end{itemize}

Chính sự "kéo dài" vùng tiếp xúc này mang lại khả năng bám địa hình vượt trội nhưng đồng thời sinh ra mô-men cản xoay lớn, buộc các bánh xe phải trượt để robot có thể đổi hướng \cite{reina2016}. Đây chính là nguyên nhân gốc rễ của sự phức tạp trong mô hình động học ICR đã phân tích ở trên.

Thách thức Hiện đại và Giới hạn Mô hình: Các nghiên cứu gần đây chỉ ra rằng việc áp dụng trực tiếp mô hình vi sai tiêu chuẩn cho robot trượt là không khả thi do sai số trượt quá lớn. Mặc dù mô hình dựa trên Tâm quay tức thời (ICR) là giải pháp thay thế phổ biến, nó vẫn tồn tại hai giới hạn lớn trong các ứng dụng thực tế:

\begin{enumerate}
    \item Giới hạn tốc độ: Mô hình ICR chỉ hợp lệ trong vùng động lực học thấp (tốc độ thấp). Ở tốc độ cao, hiệu suất bám quỹ đạo bị suy giảm mạnh do các yếu tố động lực học phi tuyến.
    \item Biến thiên tham số: Các tọa độ ICR ($y_{ICR}$) biến thiên liên tục theo địa hình và không thể đo trực tiếp. Do đó, xu hướng hiện đại là tích hợp việc ước lượng tham số ICR trực tuyến (Online Parameter Estimation) vào trong các bộ lọc Kalman mở rộng (EKF) cùng với dữ liệu IMU/GPS để đảm bảo độ chính xác định vị \cite{pentzer2014}.
    \item Tối ưu hóa Năng lượng trên Địa hình thực tế: Một thách thức lớn của robot dẫn động trượt là mức tiêu thụ năng lượng phi tuyến tính, đặc biệt trên địa hình đất mềm. Các mô hình cổ điển thường giả định mặt đường cứng và công suất tiêu thụ là hằng số. Tuy nhiên, nghiên cứu thực nghiệm gần đây của Fiset và các cộng sự (2021) \cite{fiset2021} đã chứng minh rằng trên nền đất xốp (như cát), công suất tiêu thụ tăng vọt tại các bán kính quay vòng nhỏ. Nguyên nhân là do hiện tượng "ủi đất": khi bánh xe phía trong quay ngược chiều để rẽ gắt, nó bị lún xuống và đẩy một lượng đất lớn phía trước, làm tăng mô-men cản lăn đột biến. Phát hiện này có ý nghĩa quan trọng trong việc lập kế hoạch quỹ đạo để tránh các bán kính quay gây tốn năng lượng này.
\end{enumerate}

% ======================================================================

% ----- ROBOT SÁU BÁNH VÀ CƠ CẤU ROCKER-BOGIE -----



% ======================================================================

% ----- CẤU TRÚC BÁNH XÍCH -----

\subsection{Cấu trúc bánh xích}
Cấu trúc này được tối ưu hóa cho khả năng vượt địa hình gồ ghề và bề mặt mềm.
Thay vì bánh xe, robot sử dụng hai dải xích song song.
Tương tự như cấu trúc Dẫn động trượt (Skid-Steer), robot được điều khiển bằng cách thay đổi vận tốc tương đối của hai dải xích \cite{wong2008}.
Đây là hệ thống phi đa hướng, có khả năng xoay tại chỗ, và phụ thuộc lớn vào sự trượt của dải xích trên mặt đất.
Ưu điểm: Khả năng bám địa hình (traction) vượt trội. Áp suất tác dụng lên mặt đất rất thấp (do diện tích tiếp xúc lớn), cho phép robot di chuyển trên cát, tuyết, hoặc bùn lầy. Khả năng vượt chướng ngại vật và leo dốc rất tốt \cite{murphy2014}.
Nhược điểm: Tốc độ di chuyển chậm, tiêu thụ năng lượng cao, phá hủy bề mặt di chuyển, và rất khó để ước tính vị trí chính xác [..].

Cấu trúc này đại diện cho giải pháp tối ưu nhất về khả năng vượt địa hình (traversability) trong các môi trường phi cấu trúc. Thay vì sử dụng các điểm tiếp xúc rời rạc như bánh xe, robot sử dụng hai (hoặc nhiều hơn) dải xích (tracks) tạo ra một vùng tiếp xúc liên tục (continuous footprint).

Nguyên lý hoạt động và Đặc tính cơ bản: Cơ chế lái của robot bánh xích tương tự như Cấu trúc Dẫn động trượt (Skid-Steer). Robot thay đổi hướng bằng cách tạo ra chênh lệch vận tốc giữa hai dải xích trái ($v_L$) và phải ($v_R$) [].

% ======================================================================

% ----- CẤU TRÚC LAI CƠ CẤU CHÂN - BÁNH -----

\subsection{Cấu trúc lai cơ cấu chân - bánh}
Cấu trúc lai chân - bánh đại diện cho sự hội tụ giữa hai cơ chế di chuyển cơ bản: bánh xe (hiệu quả năng lượng cao trên mặt phẳng) và chân (khả năng vượt địa hình phức tạp). Mục tiêu của thiết kế này là kết hợp ưu điểm của cả hai để tạo ra các robot đa năng vượt trội [Ref].

Nguyên lý hoạt động và Cấu hình: Thông thường, cấu trúc này bao gồm các bánh xe chủ động được gắn vào điểm cuối (end-effector) của các chân robot có nhiều bậc tự do (DOF). Robot có thể hoạt động ở hai chế độ chính:

Chế độ lăn (Driving Mode): Trên địa hình bằng phẳng, robot hạ thấp trọng tâm, khóa các khớp chân hoặc sử dụng chúng như hệ thống treo chủ động (active suspension), và di chuyển bằng bánh xe để đạt tốc độ cao và tiết kiệm năng lượng.

Chế độ bước (Walking/Stepping Mode): Khi gặp chướng ngại vật lớn hoặc cầu thang, robot khóa bánh xe và sử dụng chân để bước qua, thực hiện các chuyển động rời rạc [Ref].

Đặc tính Động học và Ưu điểm: Về mặt động học, đây là một hệ thống dư thừa dẫn động (redundant actuated system).

Khả năng thay đổi cấu hình (Reconfigurability): Khác với cấu trúc Ackermann hay Vi sai có hình học cố định, robot lai có thể thay đổi chiều rộng cơ sở (footprint), chiều cao trọng tâm (CoM height) và góc nghiêng thân xe (roll/pitch) một cách chủ động. Điều này cho phép robot duy trì thăng bằng trên các sườn dốc nghiêng mà các robot bánh xe thông thường sẽ bị lật.

Di chuyển Đa hướng (Omnidirectional): Bằng cách xoay trục của chân (yaw joint), các bánh xe có thể được định hướng theo bất kỳ góc nào, cho phép robot di chuyển holonomic tương tự như dùng bánh Mecanum nhưng trên địa hình gồ ghề.

Các vấn đề Hiện đại và Thách thức Điều khiển (Modern Issues): Đây là phần quan trọng nhất, làm nổi bật tính "hiện đại" của báo cáo:

Điều khiển Toàn thân (Whole-Body Control - WBC): Thách thức lớn nhất không nằm ở cơ khí mà ở thuật toán. Robot phải giải quyết bài toán tối ưu hóa phức tạp trong thời gian thực để phối hợp mô-men xoắn của động cơ bánh xe (để di chuyển) và động cơ chân (để giữ thăng bằng và giảm xóc). Các phương pháp hiện đại như Model Predictive Control (MPC) thường được sử dụng để dự đoán trạng thái tương lai và giữ robot ổn định khi di chuyển tốc độ cao trên địa hình lồi lõm [Ref].

Quản lý Chuyển đổi trạng thái (Locomotion Mode Switching): Việc quyết định khi nào nên lăn và khi nào nên bước là một bài toán khó về nhận thức môi trường. Robot cần sử dụng Vision/LiDAR để phân loại địa hình và chuyển đổi mượt mà giữa các mô hình động học khác nhau mà không bị gián đoạn hoặc mất thăng bằng.

Tương tác Động lực học: Khác với robot bánh xe truyền thống luôn bám đất, robot lai có những pha "bay" (flight phase) khi nhảy hoặc bước. Việc mô hình hóa các lực va chạm (impact forces) khi bánh xe tiếp đất sau một cú nhảy là rất quan trọng để bảo vệ phần cứng và duy trì quỹ đạo.

Ứng dụng thực tiễn: Các robot tiêu biểu cho xu hướng này bao gồm Boston Dynamics Handle (robot kho vận), ETH Zurich ANYmal on Wheels (robot thám hiểm), và Tencent Ollie (robot nhào lộn). Chúng đang mở ra hướng đi mới cho robot giao hàng chặng cuối (last-mile delivery) nơi robot phải leo lên bậc thềm nhà khách hàng.

% ======================================================================

% ----- ỨNG DỤNG -----

\section{Ứng dụng}
Sự đa dạng của các loại bánh xe và cấu trúc robot di động đã được phản ánh qua vô số ứng dụng thực tế. Việc lựa chọn một cấu hình di động cụ thể luôn là kết quả của bài toán tối ưu hóa giữa: khả năng cơ động, tải trọng, năng lượng tiêu thụ và đặc điểm môi trường hoạt động [].

\begin{enumerate}
    \item Logistics và Kho bãi thông minh: Đây là lĩnh vực ứng dụng lớn nhất của robot di động (AMR/AGV).
    \begin{itemize}
        \item Cấu trúc Dẫn động Vi sai: Chiếm ưu thế trong các robot vận chuyển kệ hàng (như Amazon Kiva/Robotics) nhờ chi phí thấp, độ tin cậy cao và khả năng xoay tại chỗ trong các lối đi vuông góc [].
        \item Cấu trúc Đa hướng (Mecanum/Omni): Đang dần phổ biến trong các nhà máy sản xuất linh kiện điện tử hoặc lắp ráp máy bay, nơi cần di chuyển chính xác các linh kiện lớn trong không gian chật hẹp mà không cần không gian quay đầu (zero turning radius) [].
    \end{itemize}

    \item Giao hàng chặng cuối: Một xu hướng "hiện đại" bùng nổ sau đại dịch.
    \begin{itemize}
        \item Cấu trúc 6 bánh (6-Wheel Differential/Rocker-bogie): Các robot như Starship Technologies sử dụng cấu trúc này để leo lề đường và di chuyển ổn định trên vỉa hè đô thị không bằng phẳng.
        \item Cấu trúc Ackermann: Được dùng cho các robot giao hàng tốc độ cao trên làn đường xe đạp hoặc khu dân cư rộng (như Nuro R2).
    \end{itemize}

    \item Thám hiểm và Nông nghiệp
    \begin{itemize}
        \item Cấu trúc Dẫn động trượt (Skid-steer) \& Rocker-bogie: Là tiêu chuẩn vàng cho các robot thám hiểm hành tinh (như Mars Rovers: Curiosity, Perseverance, Jackal \cite{jackal}) nhờ hệ thống cơ khí bền bỉ, không có các khớp lái phức tạp dễ hỏng hóc trong môi trường bụi bặm khắc nghiệt [].
        \item Cấu trúc Bánh xích: Vẫn là lựa chọn số một cho các robot cứu hỏa hoặc nông nghiệp hoạt động trên nền đất bùn lầy (soft soil) để giảm áp suất tiếp xúc.
    \end{itemize}

    \item Robot Dịch vụ và Tương tác người (Service \& HRI):
    \begin{itemize}
        \item Ballbot và Robot tự cân bằng: Nhờ diện tích đế nhỏ (small footprint) và chiều cao lớn, các cấu trúc này (như Rezero, Segway) rất phù hợp để di chuyển trong văn phòng đông người và tương tác ngang tầm mắt mà không gây cản trở giao thông [].
    \end{itemize}
\end{enumerate}

% ======================================================================

% ----- XU HƯỚNG NGHIÊN CỨU -----

\section{Xu hướng nghiên cứu}
Lĩnh vực thiết kế và điều khiển robot di động đang trải qua một sự chuyển dịch mô hình (paradigm shift) mạnh mẽ: từ các cơ cấu cứng nhắc, điều khiển dựa trên mô hình hình học cố định sang các hệ thống mềm dẻo, thích ứng và nhận thức môi trường. Dựa trên khảo sát các công bố khoa học từ 2019 đến 2025, các xu hướng nghiên cứu mũi nhọn có thể được phân thành bốn nhóm chính sau:

\subsubsection{Cơ cấu Bánh xe Biến hình và Vật liệu Mềm}

Giới hạn lớn nhất của bánh xe truyền thống là sự đánh đổi giữa tốc độ (trên đường bằng) và khả năng vượt chướng ngại vật. Xu hướng hiện đại tập trung vào các thiết kế có khả năng thay đổi hình học (geometry reconfiguration) trong thời gian thực.

\begin{itemize}
    \item Bánh xe Origami chịu tải trọng cao: Các nghiên cứu tiên phong của Lee và các cộng sự [48] đã giới thiệu bánh xe Origami có thể thay đổi đường kính. Tuy nhiên, thách thức lớn nhất là độ cứng vững (stiffness). Xu hướng mới nhất (2021-2023) tập trung vào cơ chế "khóa cấu trúc" (structural locking). Nghiên cứu [49] đã phát triển bánh xe Origami có khả năng chịu tải lên tới hàng trăm kg khi ở trạng thái khóa, nhưng vẫn linh hoạt biến hình khi cần thiết. Gần đây nhất, robot OriWheelBot (2023) [52] đã tích hợp khả năng này để tự động chuyển đổi giữa chế độ bánh xe tròn (di chuyển nhanh) và chế độ bánh răng/gai (leo dốc) dựa trên cảm nhận địa hình.
    \item Cấu trúc Lai Bánh-Chân (Omni-Wheg): Sự kết hợp giữa bánh xe và chân (Whegs) đang được cải tiến để đạt được tính năng toàn hướng. Robot OmniWheg [47] là một ví dụ điển hình, sử dụng cơ cấu biến hình để tách các phần của bánh xe Omni thành các "chân" riêng biệt, cho phép robot vừa đi ngang (holonomic) trên sàn phẳng, vừa bước qua bậc thang mà không cần hệ thống lái phức tạp.
\end{itemize}

\subsubsection{Động học dựa trên Thị giác và Ước lượng Trạng thái}

Đối với các robot dẫn động trượt (Skid-steer/Tracked) hoạt động ngoài trời, mô hình động học truyền thống (dựa trên Encoder) thường thất bại do hiện tượng trượt không xác định. Một xu hướng nghiên cứu đột phá trong năm 2024 là tích hợp thị giác máy vào vòng lặp động học.

\begin{itemize}
    \item Visual Kinematics Estimation: Thay vì chỉ dùng Camera/LiDAR để định vị (SLAM), nghiên cứu mới nhất của Zuo và các cộng sự (2024) [39] đề xuất sử dụng dữ liệu hình ảnh để ước lượng trực tiếp các tham số động học tức thời (như tâm quay ICR và vận tốc trượt). Bằng cách phân tích luồng quang học (optical flow) của kết cấu mặt đất, robot có thể "nhìn thấy" mình đang trượt như thế nào và tự động điều chỉnh lại mô hình động học (Self-calibrating Kinematics) theo thời gian thực, giúp giảm sai số vị trí xuống mức thấp kỷ lục mà không cần GPS độ chính xác cao.
\end{itemize}

\subsubsection{Điều khiển Phối hợp và Tối ưu Năng lượng}

Với sự phổ biến của các robot dẫn động 4 bánh độc lập (4WD/4WS), vấn đề không chỉ là "đi được" mà là "đi hiệu quả".

\begin{itemize}
    \item Điều khiển Phối hợp Mô-men xoắn (Torque Coordination): Trên các robot dẫn động trượt (Skid-steer) thừa dẫn động (over-actuated), các bánh xe thường sinh ra nội lực chống lại nhau (fighting forces). Các nghiên cứu của Liao và các cộng sự [40], [41] tập trung vào các thuật toán phân phối mô-men xoắn tối ưu (Optimal Torque Distribution). Mục tiêu là tìm ra tổ hợp lực kéo của 4 bánh sao cho robot vẫn bám quỹ đạo mong muốn nhưng tiêu tốn năng lượng ít nhất và giảm thiểu mài mòn lốp.
    \item Tích hợp Mô hình Tương tác Đất (Terramechanics-based Control): Việc đưa các mô hình cơ học đất phức tạp (như mô hình Bekker/Wong [14], [57]) vào bộ điều khiển thời gian thực đang trở thành xu hướng. Bằng cách ước lượng online các thông số của đất (độ lún, độ kết dính), robot có thể chủ động giới hạn gia tốc để tránh bị "đào hố" (digging) trên cát hoặc bùn lầy, như phân tích về tiêu thụ năng lượng của Fiset và các cộng sự (2021) [36].
\end{itemize}

\subsubsection{Điều khiển Tiên tiến cho Hệ thống Cân bằng}

Đối với cấu trúc Ballbot và robot tự cân bằng, nghiên cứu đã vượt qua giai đoạn PID/LQR cơ bản để hướng tới các thuật toán phi tuyến mạnh mẽ (Robust/Nonlinear Control).

\begin{itemize}
    \item Model Predictive Control (MPC): Xu hướng chủ đạo hiện nay là sử dụng MPC để xử lý các ràng buộc vật lý (như giới hạn góc nghiêng, điện áp động cơ). Nghiên cứu của Jespersen [73] và các công bố năm 2024-2025 [66], [75] cho thấy MPC giúp Ballbot thực hiện các quỹ đạo lượn tốc độ cao (high-speed maneuvering) và phục hồi sau va chạm tốt hơn hẳn các phương pháp cổ điển.
    \item Learning-based Control: Áp dụng AI để học đặc tính động lực học của robot. Ví dụ, sử dụng mạng nơ-ron mờ (Fuzzy Neural Networks) [6] để tự động tinh chỉnh tham số điều khiển khi robot mang vác các vật nặng có khối lượng thay đổi, giúp robot phục vụ trong gia đình trở nên an toàn và tin cậy hơn.
\end{itemize}

% ======================================================================

%----------------- KẾT LUẬN ----------------

\section{Kết luận}
Bài báo cáo này đã trình bày một cái nhìn tổng quan và hệ thống hóa về các loại bánh xe và cấu trúc cơ bản của robot di động, đồng thời phân tích các vấn đề hiện đại liên quan đến động lực học và điều khiển.

Qua phân tích, có thể rút ra các kết luận chính sau:

\begin{enumerate}
    \item Không có thiết kế vạn năng: Sự lựa chọn cấu trúc robot luôn là một sự đánh đổi (design trade-off). Cấu trúc Dẫn động vi sai tối ưu cho sự đơn giản và giá thành; Cấu trúc Đa hướng tối ưu cho sự linh hoạt trong không gian hẹp; trong khi Cấu trúc Skid-steer/Bánh xích tối ưu cho địa hình gồ ghề.
    \item Thách thức từ Thực tế: Các mô hình động học lý tưởng (lăn không trượt) là nền tảng cần thiết nhưng không đủ. Kỹ thuật robot hiện đại buộc phải giải quyết các vấn đề phi tuyến như ma sát xoay (scrubbing), trượt (slippage) và tương tác địa hình thông qua việc mở rộng mô hình và ghép nối cảm biến (Sensor Fusion).
    \item Tương lai là sự Thích ứng: Xu hướng phát triển đang hướng tới các cấu trúc lai (chân-bánh) và các vật liệu thông minh, cho phép robot không chỉ di chuyển trên địa hình mà còn thích ứng với địa hình đó.
\end{enumerate}

Hiểu rõ đặc tính của từng loại bánh xe và cấu trúc là bước đầu tiên và quan trọng nhất để thiết kế nên những hệ thống robot tự hành hiệu quả, bền bỉ và thông minh hơn trong tương lai.

% ======================================================================

\begin{thebibliography}{00}

% ----- TỔNG HỢP -----

\bibitem{mir2022} I. Mir và các cộng sự, "A survey of trajectory planning techniques for autonomous systems," Electronics, vol. 11, no. 18, p. 2801, Sep. 2022, doi: 10.3390/electronics11182801.
\bibitem{tagliavini2022} L. Tagliavini, G. Colucci, A. Botta, P. Cavallone, L. Baglieri, and G. Quaglia, "Wheeled Mobile robots: state of the art overview and kinematic comparison among three omnidirectional locomotion strategies," Journal of Intelligent \& Robotic Systems, vol. 106, no. 3, p. 57, Oct. 2022, doi: 10.1007/s10846-022-01745-7.

% ======================================================================

\bibitem{siegwart2011} R. Siegwart, I. R. Nourbakhsh, and D. Scaramuzza, \textit{Introduction to Autonomous Mobile Robots, second edition}. MIT Press, 2011.
\bibitem{fragapane2021} G. Fragapane, R. de Koster, F. Sgarbossa, and J. O. Strandhagen, "Planning and control of autonomous mobile robots for intralogistics: Literature review and research agenda," European Journal of Operational Research, vol. 294, no. 2, pp. 405-426, Jan. 2021, doi: https://doi.org/10.1016/j.ejor.2021.01.019.
\bibitem{chung2008} W. Chung and K. Iagnemma,  "Wheeled robots," \textit{Springer Handbook of Robotics}. 2008. doi: 10.1007/978-3-540-30301-5.
\bibitem{todd2012} D. J. Todd, \textit{Walking machines: An Introduction to Legged Robots}. Springer, 2012.
\bibitem{campion1993} G. Campion, G. Bastin and B. D'Andrea-Novel, "Structural properties and classification of kinematic and dynamic models of wheeled mobile robots," [1993] Proceedings IEEE International Conference on Robotics and Automation, Atlanta, GA, USA, 1993, pp. 462-469 vol.1, doi: 10.1109/ROBOT.1993.292023.
\bibitem{lynch2017} K. M. Lynch and F. C. Park, Modern robotics : mechanics, planning, and control. Cambridge: University Press, 2017.
\bibitem{rasam2016} H. R. Rasam, "Review on Land-Based Wheeled Robots," MATEC Web of Conferences, vol. 53, p. 01058, Jan. 2016, doi: 10.1051/matecconf/20165301058.
\bibitem{rubio2019} F. Rubio, F. Valero, and C. Llopis-Albert, "A review of mobile robots: Concepts, methods, theoretical framework, and applications," International Journal of Advanced Robotic Systems, vol. 16, no. 2, Mar. 2019, doi: 10.1177/1729881419839596.
\bibitem{mikova2016} Ľ. Miková and A. Gmiterko, "Kinematic model and control algorithm for the path tracking of nonholonomic mobile robots," Journal of Automation and Control, vol. 4, no. 2, pp. 26-29, Dec. 2016, doi: 10.12691/automation-4-2-4.
\bibitem{leong2022} J. S. Ling Leong, K. T. Kin Teo and H. P. Yoong, "Four Wheeled Mobile Robots: A Review," \textit{2022 IEEE International Conference on Artificial Intelligence in Engineering and Technology (IICAIET)}, pp. 1-6, Sep. 2022, doi: 10.1109/IICAIET55139.2022.9936855.
\bibitem{mikova2013} L. Miková, F. Trebuňa and M. Čurilla, "Model of mechatronic system's undercarriage created on the basis of its dynamics," 2013 International Conference on Process Control (PC), Strbske Pleso, Slovakia, 2013, pp. 231-234, doi: 10.1109/PC.2013.6581414.
\bibitem{shabalina2018} K. Shabalina, A. Sagitov and E. Magid, "Comparative Analysis of Mobile Robot Wheels Design," 2018 11th International Conference on Developments in eSystems Engineering (DeSE), Cambridge, UK, 2018, pp. 175-179, doi: 10.1109/DeSE.2018.00041.
\bibitem{wong2022} J. Y. Wong, "Mechanics of Vehicle-Terrain Interaction," Theory of ground vehicles. 2022. doi: 10.1002/9781119719984.
\bibitem{iagnemma2000} K. Iagnemma and S. Dubowsky, "Mobile Robot Rough-Terrain Control (RTC) for Planetary Exploration," Sep. 2000, doi: https://doi.org/10.1115/detc2000/mech-14211.
\bibitem{ward2008} C. C. Ward and K. Iagnemma, "A Dynamic-Model-Based Wheel Slip Detector for Mobile Robots on Outdoor Terrain," in IEEE Transactions on Robotics, vol. 24, no. 4, pp. 821-831, Aug. 2008, doi: 10.1109/TRO.2008.924945.
\bibitem{yi2009} J. Yi, H. Wang, J. Zhang, D. Song, S. Jayasuriya, and J. Liu, "Kinematic Modeling and Analysis of Skid-Steered Mobile Robots With Applications to Low-Cost Inertial-Measurement-Unit-Based Motion Estimation," vol. 25, no. 5, pp. 1087-1097, Oct. 2009, doi: https://doi.org/10.1109/tro.2009.2026506.
\bibitem{mandow2007} A. Mandow, J. Alfredo Martínez, J. Morales, J. M. Blanco, A. García-Cerezo, and J. Suarez Gonzalez, "Experimental kinematics for wheeled skid-steer mobile robots," Dec. 2007, doi: https://doi.org/10.1109/iros.2007.4399139.
\bibitem{thrun2010} S. Thrun, W. Burgard, and D. Fox, Probabilistic robotics. Cambridge, Mass.: Mit Press, 2010.
\bibitem{moore2014} T. Moore and D. Stouch, "A generalized extended kalman filter implementation for the robot operating system," in Proceedings of the 13th International Conference on Intelligent Autonomous Systems (IAS-13), 2014.
\bibitem{parhi2011} D. R. Parhi and B. B. V. L. Deepak, "Kinematic model of three wheeled mobile robot," Mechanical Engineering Research, vol. 3, no. 9, pp. 307-318, Sep. 2011, doi: 10.5897/jmer.9000032.
\bibitem{ueno2017} Y. Ueno, K. Watanabe and I. Nagai, "Design and development of steered active wheel casters and its application," 2017 IEEE International Conference on Mechatronics and Automation (ICMA), Takamatsu, Japan, 2017, pp. 507-512, doi: 10.1109/ICMA.2017.8015869.
\bibitem{qiu2018} Q. Qiu và các cộng sự, "Extended Ackermann Steering Principle for the coordinated movement control of a four wheel drive agricultural mobile robot," Computers and Electronics in Agriculture, vol. 152, pp. 40-50, Jul. 2018, doi: 10.1016/j.compag.2018.06.036.
\bibitem{gautam2021} P. Gautam, S. Sahai, S. S. Kelkar, P. S. Agrawal, and M. R. D, "Designing Variable Ackermann Steering geometry for Formula Student Race car," International Journal of Analytical Experimental and Finite Element Analysis (IJAEFEA), vol. 8, no. 1, Feb. 2021, doi: 10.26706/ijaefea.1.8.20210101.
\bibitem{patel2021} S. Patel, R. Rawat, N. Shantanu, A. Kumar, and N. Amardeep, "Study of steering system for an electric Trike-Ackermann steering," in Smart innovation, systems and technologies, 2021, pp. 9-18. doi: 10.1007/978-981-16-2857-3\_3.
\bibitem{low2005} K. H. Low and Y. P. Leow, "Kinematic modeling, mobility analysis and design of wheeled mobile robots," Advanced Robotics, vol. 19, no. 1, pp. 73-99, Jan. 2005, doi: 10.1163/1568553053020241.
\bibitem{gillespie1992} T. D. Gillespie, Fundamentals of Vehicle Dynamics. Warrendale, PA: SAE International, 1992.
\bibitem{deluca1998} A. De Luca, G. Oriolo, and C. Samson, "Feedback control of a nonholonomic car-like robot," in Robot Motion Planning and Control, J.-P. Laumond, Ed. Springer, 1998, pp. 171-253.
\bibitem{arrizabalaga2021} J. Arrizabalaga, N. Van Duijkeren, M. Ryll, and R. Lange, "A caster-wheel-aware MPC-based motion planner for mobile robotics," 2021 20th International Conference on Advanced Robotics (ICAR), pp. 613-618, Dec. 2021, doi: 10.1109/icar53236.2021.9659478.

\bibitem{mukherjee1999} R. Mukherjee, M. A. Minor and J. T. Pukrushpan, "Simple motion planning strategies for spherobot: a spherical mobile robot," Proceedings of the 38th IEEE Conference on Decision and Control (Cat. No.99CH36304), Phoenix, AZ, USA, 1999, pp. 2132-2137 vol.3, doi: 10.1109/CDC.1999.831235.
\bibitem{nagarajan2009} U. Nagarajan, G. Kantor, and R. L. Hollis, "Trajectory planning and control of an underactuated dynamically stable single spherical wheeled mobile robot," IEEE Xplore, May 01, 2009. https://ieeexplore.ieee.org/document/5152624.
\bibitem{lee2024} W. Lee, J. Kim, and T. Seo, "Design and analysis of a mobile robot with novel caster mechanism for high step-overcoming capability," Scientific Reports, vol. 14, no. 1, p. 13745, Jun. 2024, doi: 10.1038/s41598-024-63825-y.

% ======================================================================

% ----- ROBOT XE ĐẠP MỘT BÁNH -----

% 1
\bibitem{Han2015} S. I. Han and J. M. Lee, "Balancing and Velocity Control of a Unicycle Robot Based on the Dynamic Model," in IEEE Transactions on Industrial Electronics, vol. 62, no. 1, pp. 405-413, Jan. 2015, doi: 10.1109/TIE.2014.2327562.
% 2
\bibitem{Rosyidi2016} M. A. Rosyidi, E. H. Binugroho, S. E. Radin Charel, R. S. Dewanto and D. Pramadihanto, "Speed and balancing control for unicycle robot," 2016 International Electronics Symposium (IES), Denpasar, Indonesia, 2016, pp. 19-24, doi: 10.1109/ELECSYM.2016.7860969.
% 3
\bibitem{Lee2013} J. Lee, S. Han and J. Lee, "Decoupled Dynamic Control for Pitch and Roll Axes of the Unicycle Robot," in IEEE Transactions on Industrial Electronics, vol. 60, no. 9, pp. 3814-3822, Sept. 2013, doi: 10.1109/TIE.2012.2208431.
% 4
\bibitem{Shen2020} J. Shen and D. Hong, "OmBURo: A Novel Unicycle Robot with Active Omnidirectional Wheel," 2020 IEEE International Conference on Robotics and Automation (ICRA), Paris, France, 2020, pp. 8237-8243, doi: 10.1109/ICRA40945.2020.9196927.
% 5
\bibitem{Rizal2015} Y. Rizal, C. -T. Ke and M. -T. Ho, "Point-to-point motion control of a unicycle robot: Design, implementation, and validation," 2015 IEEE International Conference on Robotics and Automation (ICRA), Seattle, WA, USA, 2015, pp. 4379-4384, doi: 10.1109/ICRA.2015.7139804.
% 6
\bibitem{Suzuki2014} H. Suzuki, S. Moromugi, and T. Okura, "Development of robotic unicycles," Journal of Robotics and Mechatronics, vol. 26, no. 5, pp. 540-549, Oct. 2014, doi: 10.20965/jrm.2014.p0540.
% 7
\bibitem{Geist2022} A. R. Geist, J. Fiene, N. Tashiro, Z. Jia and S. Trimpe, "The Wheelbot: A Jumping Reaction Wheel Unicycle," in IEEE Robotics and Automation Letters, vol. 7, no. 4, pp. 9683-9690, Oct. 2022, doi: 10.1109/LRA.2022.3192654.
% 8
\bibitem{JHLee2010} J. H. Lee, H. J. Shin, S. J. Lee and S. Jung, "Novel air blowing control for balancing a unicycle robot," 2010 IEEE/RSJ International Conference on Intelligent Robots and Systems, Taipei, Taiwan, 2010, pp. 2529-2530, doi: 10.1109/IROS.2010.5649120.

% ======================================================================

% ----- ROBOT CÂN BẰNG TRÊN BÓNG -----

% 1
\bibitem{Lauwers2006} T. B. Lauwers, G. A. Kantor and R. L. Hollis, "A dynamically stable single-wheeled mobile robot with inverse mouse-ball drive," Proceedings 2006 IEEE International Conference on Robotics and Automation, 2006. ICRA 2006., Orlando, FL, USA, 2006, pp. 2884-2889, doi: 10.1109/ROBOT.2006.1642139.
% 2
\bibitem{Kumagai2008} M. Kumagai and T. Ochiai, "Development of a robot balancing on a ball," 2008 International Conference on Control, Automation and Systems, Seoul, Korea (South), 2008, pp. 433-438, doi: 10.1109/ICCAS.2008.4694680.
% 3
\bibitem{Lima2018} F. H. B. Lima, E. Poleze, G. P. das Neves and B. A. Angélico, "Ball Balancing Robot: construction, modeling and control design," 2018 13th IEEE International Conference on Industry Applications (INDUSCON), Sao Paulo, Brazil, 2018, pp. 1368-1372, doi: 10.1109/INDUSCON.2018.8627171.
% 4
\bibitem{Nagarajan2013} U. Nagarajan, G. Kantor, and R. Hollis, "The ballbot: An omnidirectional balancing mobile robot," The International Journal of Robotics Research, vol. 33, no. 6, pp. 917-930, Nov. 2013, doi: 10.1177/0278364913509126.
% 5
\bibitem{Cai2019} C. Cai, J. Lu and Z. Li, "Kinematic Analysis and Control Algorithm for the Ballbot," in IEEE Access, vol. 7, pp. 38314-38321, 2019, doi: 10.1109/ACCESS.2019.2902219.
% 6
\bibitem{Xiao2023} C. Xiao, M. Mansouri, D. Lam, J. Ramos and E. T. Hsiao-Wecksler, "Design and Control of a Ballbot Drivetrain with High Agility, Minimal Footprint, and High Payload," 2023 IEEE/RSJ International Conference on Intelligent Robots and Systems (IROS), Detroit, MI, USA, 2023, pp. 376-383, doi: 10.1109/IROS55552.2023.10342007.
% 7
\bibitem{Song2023} S. Y. Song, N. Marin, C. Xiao, R. Okubo, J. Ramos and E. T. Hsiao-Wecksler, "Hands-Free Physical Human-Robot Interaction and Testing for Navigating a Virtual Ballbot," 2023 32nd IEEE International Conference on Robot and Human Interactive Communication (RO-MAN), Busan, Korea, Republic of, 2023, pp. 556-563, doi: 10.1109/RO-MAN57019.2023.10309327.
% 8
\bibitem{Park2024} S. Park, "Design, Implementation, and Control of a Ball-Balancing Robot," in IEEE Access, vol. 12, pp. 127380-127389, 2024, doi: 10.1109/ACCESS.2024.3456238.
% 9
\bibitem{HYHYHan2014} H. Y. Han, T. Y. Han and H. S. Jo, "Development of omnidirectional self-balancing robot," 2014 IEEE International Symposium on Robotics and Manufacturing Automation (ROMA), Kuala Lumpur, Malaysia, 2014, pp. 57-62, doi: 10.1109/ROMA.2014.7295862.
% 10
\bibitem{Pellegrini2011} E. Pellegrini, K. J. Diepold, R. Dessort, and H. Panzer, "3D-Modeling of a Robot Balancing on a Ball," Technical Reports on Automatic Control, Jan. 2011, [Online]. Available: https://mediatum.ub.tum.de/1081938.
% 11
\bibitem{Abdelrahim2025} M. Abdelrahim, M. A. Thabet, H. S. Abbas, M. M. M. Hassan, M. H. Amin and A. Morsi, "Modeling and Control of a Ballbot: A Systematic Approach," in IEEE Access, vol. 13, pp. 141263-141280, 2025, doi: 10.1109/ACCESS.2025.3597797.

% ======================================================================

% ----- ROBOT DẠNG CẦU -----

% 1
\bibitem{MLi2014} M. Li, S. Guo, H. Hirata, and H. Ishihara, "Design and performance evaluation of an amphibious spherical robot," Robotics and Autonomous Systems, vol. 64, pp. 21-34, Nov. 2014, doi: 10.1016/j.robot.2014.11.007.
% 2
\bibitem{Sang2010} S. Sang, Z. Jichao, H. Wu, S. Chen, and Q. An, "Modeling and simulation of a spherical mobile robot," Computer Science and Information Systems, vol. 7, no. 1, pp. 51-62, Jan. 2010, doi: 10.2298/csis1001051s.
% 3
\bibitem{Joshi2009} V. A. Joshi, R. N. Banavar, and R. Hippalgaonkar, "Design and analysis of a spherical mobile robot," Mechanism and Machine Theory, vol. 45, no. 2, pp. 130-136, May 2009, doi: 10.1016/j.mechmachtheory.2009.04.003.
% 4
\bibitem{Chen2013} W.-H. Chen, C.-P. Chen, J.-S. Tsai, J. Yang, and P.-C. Lin, "Design and implementation of a ball-driven omnidirectional spherical robot," Mechanism and Machine Theory, vol. 68, pp. 35-48, May 2013, doi: 10.1016/j.mechmachtheory.2013.04.012.
% 5
\bibitem{Alves2003} J. Alves and J. Dias, "Design and control of a spherical mobile robot," Proceedings of the Institution of Mechanical Engineers Part I Journal of Systems and Control Engineering, vol. 217, no. 6, pp. 457-467, Sep. 2003, doi: 10.1177/095965180321700602.
% 6
\bibitem{Ren2023} W. Ren et al., "Spherical robot: A novel robot for exploration in harsh unknown environments," IET Cyber-Systems and Robotics, vol. 5, no. 4, Oct. 2023, doi: 10.1049/csy2.12099.
% 7
\bibitem{Chadil2011} N. Chadil, M. Phadoongsidhi, K. Suwannasit, P. Manoonpong and P. Laksanacharoen, "A reconfigurable spherical robot," 2011 IEEE International Conference on Robotics and Automation, Shanghai, China, 2011, pp. 2380-2385, doi: 10.1109/ICRA.2011.5979756.
% 8
\bibitem{Yang2020} M. Yang et al., "Design and Analysis of a Spherical Robot with Two Degrees of Freedom Swing," 2020 Chinese Control And Decision Conference (CCDC), Hefei, China, 2020, pp. 4913-4918, doi: 10.1109/CCDC49329.2020.9164196.
% 9
\bibitem{JChen2016} J. Chen, P. Ye, H. Sun and Q. Jia, "Design and motion control of a spherical robot with control moment gyroscope," 2016 3rd International Conference on Systems and Informatics (ICSAI), Shanghai, China, 2016, pp. 114-120, doi: 10.1109/ICSAI.2016.7810940.
% 10
\bibitem{Tholapu2021} S. Tholapu, A. P. Sudheer, and M. L. Joy, "Kinematic modelling and structural analysis of a spherical robot: BALL-E," IOP Conference Series Materials Science and Engineering, vol. 1132, no. 1, p. 012034, Apr. 2021, doi: 10.1088/1757-899x/1132/1/012034.

% ======================================================================

% ----- ROBOT ĐƠN BÁNH VÀNH LỚN -----

% 1
\bibitem{Cieslak2011} P. Cieslak, T. Buratowski, T. Uhl, and M. Giergiel, "The mono-wheel robot with dynamic stabilisation," Robotics and Autonomous Systems, vol. 59, no. 9, pp. 611-619, May 2011, doi: 10.1016/j.robot.2011.05.002.
% 2
\bibitem{Gim2024} K. G. Gim and J. Kim, "Ringbot: Monocycle Robot With Legs," in IEEE Transactions on Robotics, vol. 40, pp. 1890-1905, 2024, doi: 10.1109/TRO.2024.3362326.
% 3
\bibitem{Buratowski2012} T. Buratowski, P. Cieślak, M. Giergiel, and T. Uhl, "A self-stabilising multipurpose single-wheel robot," Journal of Theoretical and Applied Mechanics/Mechanika Teoretyczna I Stosowana, vol. 50, no. 1, pp. 99-118, Jan. 2012, [Online]. Available: http://www.ptmts.org.pl/jtam/index.php/jtam/article/download/v50n1p99/102

% ======================================================================

% ----- ROBOT KIỂU XE ĐẠP -----

% 1
\bibitem{Chiu2020} C. -H. Chiu and C. -Y. Wu, "Bicycle Robot Balance Control Based on a Robust Intelligent Controller," in IEEE Access, vol. 8, pp. 84837-84849, 2020, doi: 10.1109/ACCESS.2020.2992792.
% 2
\bibitem{Yeh2019} T. Yeh, H.-T. Lu, and P.-H. Tseng, "Balancing Control of a Self-driving Bicycle," Proceedings of the 16th International Conference on Informatics in Control, Automation and Robotics - Volume 2: ICINCO, pp. 34-41, Jan. 2019, doi: 10.5220/0007810600340041.
% 3
\bibitem{Huang2017} C. -F. Huang, Y. -C. Tung and T. -J. Yeh, "Balancing control of a robot bicycle with uncertain center of gravity," 2017 IEEE International Conference on Robotics and Automation (ICRA), Singapore, 2017, pp. 5858-5863, doi: 10.1109/ICRA.2017.7989689.
% 4
\bibitem{Kumar2021} S. Kumar and A. Kumar, "Reduced order modelling and balancing control of bicycle robot," FME Transaction, vol. 49, no. 4, pp. 919-934, Jan. 2021, doi: 10.5937/fme2104919k.
% 5
\bibitem{Nguyen2022} S. T. Nguyen, H. Nguyen, S. T. Bui, V. A. Ho, T. D. Ngo and H. M. La, "An Agile Bicycle-like Robot for Complex Steel Structure Inspection," 2022 International Conference on Robotics and Automation (ICRA), Philadelphia, PA, USA, 2022, pp. 157-163, doi: 10.1109/ICRA46639.2022.9812153.

% ======================================================================

% ----- ROBOT KIỂU CON LẮC NGƯỢC -----

% 1
\bibitem{Chan2013} R. P. M. Chan, K. A. Stol, and C. R. Halkyard, "Review of modelling and control of two-wheeled robots," Annual Reviews in Control, vol. 37, no. 1, pp. 89-103, Apr. 2013, doi: 10.1016/j.arcontrol.2013.03.004.
% 2
\bibitem{Juang2013} H. -S. Juang and K. -Y. Lum, "Design and control of a two-wheel self-balancing robot using the arduino microcontroller board," 2013 10th IEEE International Conference on Control and Automation (ICCA), Hangzhou, China, 2013, pp. 634-639, doi: 10.1109/ICCA.2013.6565146.
% 3
\bibitem{Klemm2019} V. Klemm et al., "Ascento: A Two-Wheeled Jumping Robot," 2019 International Conference on Robotics and Automation (ICRA), Montreal, QC, Canada, 2019, pp. 7515-7521, doi: 10.1109/ICRA.2019.8793792.
% 4
\bibitem{Jeong2007} SH. Jeong and T. Takahashi, "Wheeled inverted pendulum type assistant robot: inverted mobile, standing, and sitting motions," 2007 IEEE/RSJ International Conference on Intelligent Robots and Systems, San Diego, CA, USA, 2007, pp. 1932-1937, doi: 10.1109/IROS.2007.4398961.
% 5
\bibitem{Akesson2006} J. Akesson, A. Blomdell and R. Braun, "Design and Control of YAIP — an Inverted Pendulum on Two Wheels Robot," 2006 IEEE Conference on Computer Aided Control System Design, 2006 IEEE International Conference on Control Applications, 2006 IEEE International Symposium on Intelligent Control, Munich, Germany, 2006, pp. 2178-2183, doi: 10.1109/CACSD-CCA-ISIC.2006.4776978.
% 6
\bibitem{Zhang2018} Y. Zhang, L. Zhang, W. Wang, Y. Li and Q. Zhang, "Design and Implementation of a Two-Wheel and Hopping Robot With a Linkage Mechanism," in IEEE Access, vol. 6, pp. 42422-42430, 2018, doi: 10.1109/ACCESS.2018.2859840.
% 7
\bibitem{Hassan2023} I. A. Hassan, I. A. Abed, and W. A. Al-Hussaibi, "Path planning and trajectory tracking control for Two-Wheel mobile robot," Journal of Robotics and Control (JRC), vol. 5, no. 1, pp. 1-15, Dec. 2023, doi: 10.18196/jrc.v5i1.20489.
% 8
\bibitem{Esmaeili2017} N. Esmaeili, A. Alfi, and H. Khosravi, "Balancing and trajectory tracking of Two-Wheeled mobile robot using backstepping sliding mode control: design and experiments," Journal of Intelligent \& Robotic Systems, vol. 87, no. 3-4, pp. 601-613, Jan. 2017, doi: 10.1007/s10846-017-0486-9.
% 9
\bibitem{An2013} W. An and Y. Li, "Simulation and control of a two-wheeled self-balancing robot," 2013 IEEE International Conference on Robotics and Biomimetics (ROBIO), Shenzhen, China, 2013, pp. 456-461, doi: 10.1109/ROBIO.2013.6739501.
% 10
\bibitem{Kim2015} S. Kim and S. Kwon, “Dynamic modeling of a two-wheeled inverted pendulum balancing mobile robot,” International Journal of Control Automation and Systems, vol. 13, no. 4, pp. 926-933, May 2015, doi: 10.1007/s12555-014-0564-8.
% 11
\bibitem{Aouf2019} A. Aouf, L. Boussaid, and A. Sakly, "Same fuzzy logic controller for Two-Wheeled mobile robot navigation in strange environments," Journal of Robotics, vol. 2019, pp. 1-11, Jan. 2019, doi: 10.1155/2019/2465219.
% 12
\bibitem{Li2007} J. Li, X. Gao, Q. Huang, Q. Du and X. Duan, "Mechanical Design and Dynamic Modeling of a Two-Wheeled Inverted Pendulum Mobile Robot," 2007 IEEE International Conference on Automation and Logistics, Jinan, China, 2007, pp. 1614-1619, doi: 10.1109/ICAL.2007.4338830.
% 13
\bibitem{Ali2021} A. T. Ali, A. M. O. Mohamedy, A. S. A. Salimz, E. -A. O. M. El-Aminx and O. M. K. Ahmed, "Design and Implementation of Two-Wheeled Self-Balancing Robot Using PID Controller," 2020 International Conference on Computer, Control, Electrical, and Electronics Engineering (ICCCEEE), Khartoum, Sudan, 2021, pp. 1-5, doi: 10.1109/ICCCEEE49695.2021.9429579.
% 14
\bibitem{Bin2010} H. Bin, L. W. Zhen and L. H. Feng, "The Kinematics Model of a Two-Wheeled Self-Balancing Autonomous Mobile Robot and Its Simulation," 2010 Second International Conference on Computer Engineering and Applications, Bali, Indonesia, 2010, pp. 64-68, doi: 10.1109/ICCEA.2010.169.



% ======================================================================

% ----- ROBOT DẪN ĐỘNG VI SAI -----


% ======================================================================

% ----- ROBOT DẪN ĐỘNG ĐỒNG BỘ -----


% ======================================================================

% ----- ROBOT TOÀN HƯỚNG SỬ DỤNG BÁNH MECANUM -----


% ======================================================================

% ----- ROBOT TOÀN HƯỚNG SỬ DỤNG BÁNH XE DẪN HƯỚNG CHỦ ĐỘNG -----



% ======================================================================

% ----- ROBOT TOÀN HƯỚNG SỬ DỤNG CƠ CHẾ LÁI CHỦ ĐỘNG -----


% ======================================================================

% ----- CẤU TRÚC LÁI BA BÁNH -----


% ======================================================================

% ----- CẤU TRÚC DẪN ĐỘNG TRƯỢT -----

\bibitem{mandow2007} A. Mandow, J. L. Martinez, J. Morales, J. L. Blanco, A. Garcia-Cerezo and J. Gonzalez, "Experimental kinematics for wheeled skid-steer mobile robots," 2007 IEEE/RSJ International Conference on Intelligent Robots and Systems, San Diego, CA, USA, 2007, pp. 1222-1227, doi: 10.1109/IROS.2007.4399139.
\bibitem{reina2016} G. Reina and R. Galati, "Slip-based terrain estimation with a skid-steer vehicle," Vehicle System Dynamics, vol. 54, no. 10, pp. 1384-1404, Jun. 2016, doi: 10.1080/00423114.2016.1203961.
\bibitem{pentzer2014} J. Pentzer, S. Brennan, and K. Reichard, "Model-based Prediction of Skid-steer Robot Kinematics Using Online Estimation of Track Instantaneous Centers of Rotation," Journal of Field Robotics, vol. 31, no. 3, pp. 455-476, Mar. 2014, doi: https://doi.org/10.1002/rob.21509.
\bibitem{fiset2021} J.-S. Fiset, M. Effati, and K. Skonieczny, "Effects of turning radius on skid-steered wheeled robot power consumption on loose soil," pp. 115-129, 2021.
\bibitem{rabiee2019} S. Rabiee and J. Biswas, "A Friction-Based Kinematic Model for Skid-Steer Wheeled Mobile Robots," 2019 International Conference on Robotics and Automation (ICRA), Montreal, QC, Canada, 2019, pp. 8563-8569, doi: 10.1109/ICRA.2019.8794216.
\bibitem{ferreira2023} A. R. S. Ferreira, V. S. Medeiros, H. V. H. Ayala and M. A. Meggiolaro, "Hierarchical control in skid steer mobile robots with nonholonomics constraints," 2023 31st Mediterranean Conference on Control and Automation (MED), Limassol, Cyprus, 2023, pp. 299-304, doi: 10.1109/MED59994.2023.10185672.
\bibitem{zuo2024} X. Zuo và các cộng sự, "Visual-Based Kinematics and Pose Estimation for Skid-Steering Robots," in IEEE Transactions on Automation Science and Engineering, vol. 21, no. 1, pp. 91-105, Jan. 2024, doi: 10.1109/TASE.2022.3214984.
\bibitem{liao2019} J. Liao, Z. Chen and B. Yao, "Model-Based Coordinated Control of Four-Wheel Independently Driven Skid Steer Mobile Robot with Wheel-Ground Interaction and Wheel Dynamics," in IEEE Transactions on Industrial Informatics, vol. 15, no. 3, pp. 1742-1752, March 2019, doi: 10.1109/TII.2018.2869573.
\bibitem{liao2017} J. Liao, Z. Chen and B. Yao, "Performance-Oriented Coordinated Adaptive Robust Control for Four-Wheel Independently Driven Skid Steer Mobile Robot," in IEEE Access, vol. 5, pp. 19048-19057, 2017, doi: 10.1109/ACCESS.2017.2754647.

% ======================================================================

\bibitem{garcia2016} J. M. García, J. L. Martínez, A. Mandow, and A. García-Cerezo, "Caster-leg aided maneuver for negotiating surface discontinuities with a wheeled skid-steer mobile robot," Robotics and Autonomous Systems, vol. 91, pp. 25-37, Dec. 2016, doi: 10.1016/j.robot.2016.12.007.
\bibitem{hirpo2017} B. D. Hirpo , W. Zhongmin, "Design and Control for Differential Drive Mobile Robot", INTERNATIONAL JOURNAL OF ENGINEERING RESEARCH \& TECHNOLOGY (IJERT), vol. 6, no. 10, pp. 327-334, Oct. 2017.
\bibitem{rodriguez2017} A. A. Rodriguez và các cộng sự, "Modeling, design and control of low-cost differential-drive robotic ground vehicles: Part II — Multiple vehicle study," 2017 IEEE Conference on Control Technology and Applications (CCTA), Maui, HI, USA, 2017, pp. 161-166, doi: 10.1109/CCTA.2017.8062457.
\bibitem{cao2022} R. Cao, J. Gu, C. Yu and A. Rosendo, "OmniWheg: An Omnidirectional Wheel-Leg Transformable Robot," 2022 IEEE/RSJ International Conference on Intelligent Robots and Systems (IROS), Kyoto, Japan, 2022, pp. 5626-5631, doi: 10.1109/IROS47612.2022.9982030.
\bibitem{lee2017} D.-Y. Lee, S.-R. Kim, J.-S. Kim, J.-J. Park, and K.-J. Cho, "Origami Wheel Transformer: A Variable-Diameter wheel drive robot using an origami structure," Soft Robotics, vol. 4, no. 2, pp. 163-180, May 2017, doi: 10.1089/soro.2016.0038.
\bibitem{lee2021} D.-Y. Lee, J.-K. Kim, C.-Y. Sohn, J.-M. Heo, and K.-J. Cho, "High-load capacity origami transformable wheel," Science Robotics, vol. 6, no. 53, Apr. 2021, doi: 10.1126/scirobotics.abe0201.
\bibitem{rhoads2016} B. P. Rhoads and H.-J. Su, "The Design and Fabrication of a Deformable Origami Wheel," Volume 5B: 40th Mechanisms and Robotics Conference, Aug. 2016, doi: https://doi.org/10.1115/detc2016-60045.
\bibitem{berre2020} J. Berre, François Geiskopf, L. Rubbert, and P. Renaud, "Origami-Inspired Design of a Deployable Wheel," Mechanisms and machine science, pp. 114-126, Oct. 2020, doi: https://doi.org/10.1007/978-3-030-60076-1\_11.
\bibitem{liu2023} J. Liu và các cộng sự, "OriWheelBot: An origami-wheeled robot," arXiv (Cornell University), Jan. 2023, doi: https://doi.org/10.48550/arxiv.2310.00033.
\bibitem{liao2008} Ching-Wen Liao, Ching-Chih Tsai, Yi Yu Li and Cheng-Kai Chan, "Dynamic modeling and sliding-mode control of a Ball robot with inverse mouse-ball drive," 2008 SICE Annual Conference, Chofu, Japan, 2008, pp. 2951-2955, doi: 10.1109/SICE.2008.4655168.
\bibitem{trojnacki2019} M. Trojnacki and P. D\pl{ą}bek, "Mechanical properties of modern wheeled mobile robots," Journal of Automation Mobile Robotics \& Intelligent Systems, pp. 3-13, Jul. 2019, doi: 10.14313/jamris/3-2019/21.
\bibitem{nakata2022} Y. Nakata, S. Yagi, S. Yu, Y. Wang, N. Ise, Y. Nakamura, and H. Ishiguro, "Development of 'ibuki' an electrically actuated childlike Android with mobility and its potential in the future society," Robotica, vol. 40, no. 4, pp. 933-950, Apr. 2022.

\bibitem{bhatia2015} A. Bhatia, M. Kumagai, and R. Hollis, "Six-stator spherical induction motor for balancing mobile robots," in Proc. IEEE Int. Conf. Robot. Autom. (ICRA), May 2015, pp. 226-231.

% ======================================================================

% ----- BÁNH XE CỐ ĐỊNH -----

% ======================================================================

% ----- BÁNH XE LÁI -----

% ======================================================================

% ----- BÁNH XE XOAY TỰ DO -----

% ======================================================================

% ----- BÁNH XE OMNI -----

\bibitem{ferriere1996} L. Ferriere, B. Raucent and G. Campion, "Design of omnimobile robot wheels," Proceedings of IEEE International Conference on Robotics and Automation, Minneapolis, MN, USA, 1996, pp. 3664-3670 vol.4, doi: 10.1109/ROBOT.1996.509271.
\bibitem{kanjanawanishkul2015} K. Kanjanawanishkul, "Omnidirectional wheeled mobile robots: wheel types and practical applications," International Journal of Advanced Mechatronic Systems, vol. 6, no. 6, p. 289, Jan. 2015, doi: 10.1504/ijamechs.2015.074788.
\bibitem{patil2024} S. Patil, M. Madgule, and R. Patil, "Design and analysis of carbon reinforced composite plate for robotic Omni-Wheel: replacement of costly metals," Journal of the Institution of Engineers (India) Series D, Nov. 2024, doi: 10.1007/s40033-024-00831-4.
\bibitem{williams2002} R. L. Williams, B. E. Carter, P. Gallina and G. Rosati, "Dynamic model with slip for wheeled omnidirectional robots," in IEEE Transactions on Robotics and Automation, vol. 18, no. 3, pp. 285-293, June 2002, doi: 10.1109/TRA.2002.1019459.
\bibitem{komori2016} M. Komori, K. Matsuda, T. Terakawa, F. Takeoka, H. Nishihara, and H. Ohashi, "Active omni wheel capable of active motion in arbitrary direction and omnidirectional vehicle," Journal of Advanced Mechanical Design Systems and Manufacturing, vol. 10, no. 6, p. JAMDSM0086, Jan. 2016, doi: 10.1299/jamdsm.2016jamdsm0086.

% ======================================================================

% ----- BÁNH XE MECANUM -----

\bibitem{ilon1975} B. E. Ilon, "Wheels for a course stable selfpropelling vehicle movable in any desired direction on the ground or some other base," U.S. Patent 3 876 255, 1975.
\bibitem{wikipedia_mecanum_wheel} Wikipedia, "Mecanum wheel," Wikipedia, Sep. 08, 2025. https://en.wikipedia.org/wiki/Mecanum\_wheel.
\bibitem{barnett2003} D. H. Barnett, "Low vibration omni-directional wheel," U.S. Patent 6,547,340, Apr. 15, 2003.
\bibitem{potter2010} S. D. Potter, "Omni-directional wheel," U.S. Patent Application 2010/0187779 A1, published Jul. 29, 2010.
\bibitem{baker2010} A. R. Baker, "Omni-directional wheel design for construction cost reduction," U.S. Patent 7,641,288, Jan. 5, 2010.
\bibitem{scogna2011} J. Scogna, "Wheel for omni-directional vehicle," U.S. Patent 8,011,735, Sep. 6, 2011.
\bibitem{schlee2013} K. L. Schlee, "Mecanum wheel," U.S. Patent Application 2013/0292918 A1, published Nov. 7, 2013.
\bibitem{gunther2014} W. G\"unther, "Wheel," U.S. Patent 8,833,862, Sep. 16, 2014.
\bibitem{bae2016} J.-J. Bae and N. Kang, "Design optimization of a mecanum wheel to reduce vertical vibrations by the consideration of equivalent stiffness," Shock and Vibration, vol. 2016, pp. 1-8, Jan. 2016, doi: 10.1155/2016/5892784.
\bibitem{giurgiu2022} T. Giurgiu, G. Bârsan, I. Virca, and C. Pupăză, "Mecanum wheeled platforms for special applications," International Conference KNOWLEDGE-BASED ORGANIZATION, vol. 28, no. 3, pp. 44-51, Jun. 2022, doi: 10.2478/kbo-2022-0086.
\bibitem{dyakov2024} A. S. Dyakov and D. S. Fedorov, "GEOMETRY, KINEMATICS AND DYNAMICS OF THE MECANUM WHEEL. UNSTABLE ACCELERATED MOTION OF MECANUM VEHICLE," Izvestiya MGTU MAMI, May 2024, doi: 10.17816/2074-0530-629873.
\bibitem{dosoftei2020} C. Dosoftei, V. Horga, I. Doroftei, T. Popovici and Ş. Custura, "Simplified Mecanum Wheel Modelling using a Reduced Omni Wheel Model for Dynamic Simulation of an Omnidirectional Mobile Robot," 2020 International Conference and Exposition on Electrical And Power Engineering (EPE), Iasi, Romania, 2020, pp. 721-726, doi: 10.1109/EPE50722.2020.9305643.
\bibitem{hasana2021} S. Hasana and H. Alwan, "Modeling and control of wheeled mobile robot with four Mecanum wheels," Engineering and Technology Journal, vol. 39, no. 5A, pp. 779-789, May 2021, doi: 10.30684/etj.v39i5a.1926.
\bibitem{doroliat2022} A. P. Doroliat, M.-H. Ing, and C.-H. G. Li, "Optimization of mecanum wheels for mitigation of AGV vibration," The International Journal of Advanced Manufacturing Technology, vol. 121, no. 1-2, pp. 633-645, May 2022, doi: 10.1007/s00170-022-09298-7.
\bibitem{zimmermann2014} K. Zimmermann, I. Zeidis, and M. Abdelrahman, "Dynamics of Mechanical Systems with Mecanum Wheels," in Springer proceedings in mathematics \& statistics, 2014, pp. 269-279. doi: 10.1007/978-3-319-08266-0\_19.
\bibitem{zhewen2024} Z. Zhewen, Y. Hongliu, W. Chengjia, H. Pu, and W. Jiangui, "A comprehensive study on Mecanum wheel-based mobility and suspension solutions for intelligent nursing wheelchairs," Scientific Reports, vol. 14, no. 1, p. 20644, Sep. 2024, doi: 10.1038/s41598-024-71459-3.
\bibitem{yadav2022} N. P. S. Yadav, V. Agrawal, N. J. C. Mohanta, and N. Md. F. Ahmed, "A Theoretical Review of Mobile Robot Locomotion based on Mecanum Wheels," Evergreen, vol. 9, no. 2, pp. 396-403, Jun. 2022, doi: 10.5109/4794163.
\bibitem{salih2006} J. E. M. Salih, M. Rizon, and S. Yaacob, "Designing Omni-Directional Mobile Robot with Mecanum Wheel," American Journal of Applied Sciences, vol. 3, no. 5, pp. 1831-1835, May 2006, doi: 10.3844/ajassp.2006.1831.1835.
\bibitem{ignatiev2016} K. V. Ignatiev, M. M. Kopichev and A. V. Putov, "Autonomous omni-wheeled mobile robots," 2016 2nd International Conference on Industrial Engineering, Applications and Manufacturing (ICIEAM), Chelyabinsk, Russia, 2016, pp. 1-4, doi: 10.1109/ICIEAM.2016.7910957.
\bibitem{doroftei2011} I. Doroftei and B.-D. D. Mangeron, "PRACTICAL APPLICATIONS FOR MOBILE ROBOTS BASED ON MECANUM WHEELS - A SYSTEMATIC SURVEY," Romanian Review Precision Mechanics, Optics and Mechatronics, Jan. 2011, [Online]. Available: http://www.incdmtm.ro/mecahitech2011/articole/Pp112-123.pdf

% ======================================================================

% ----- BÁNH HÌNH CẦU -----



% ======================================================================

% ----- BÁNH ORIGAMI -----


% ======================================================================

% ----- CẤU TRÚC DẪN ĐỘNG VI SAI -----

% 1
\bibitem{Stefek2020} A. Stefek, T. V. Pham, V. Krivanek and K. L. Pham, "Energy Comparison of Controllers Used for a Differential Drive Wheeled Mobile Robot," in IEEE Access, vol. 8, pp. 170915-170927, 2020, doi: 10.1109/ACCESS.2020.3023345.
% 2
\bibitem{Fareh2019} R. Fareh, M. Baziyad, M. H. Rahman, T. Rabie, and M. Bettayeb, "Investigating reduced path planning strategy for differential wheeled mobile robot," Robotica, vol. 38, no. 2, pp. 235-255, May 2019, doi: 10.1017/s0263574719000572.


% ======================================================================

% ----- CẤU TRÚC LÁI Ackermann -----

% ======================================================================

% ----- CẤU TRÚC BA BÁNH -----

% 1
\bibitem{chand2022} P. Chand, "Integrating an electronic compass for position tracking on a wheeled tricycle mobile robot," Drone Systems and Applications, vol. 10, no. 1, pp. 179-199, Jan. 2022, doi: 10.1139/dsa-2021-0049.
% 2
\bibitem{jaskot2018} A. Jaskot and B. Posiadała, "Analysis of motion of the three wheeled mobile platform," MATEC Web of Conferences, vol. 157, p. 01008, Jan. 2018, doi: 10.1051/matecconf/201815701008.
% 3
\bibitem{patrascu2021} M. Pătrașcu and V. Gheorghe, "Robust Position Control for High Slip Risk Tricycle Robots with Real-coded Genetic Algorithms," 2021 60th IEEE Conference on Decision and Control (CDC), Austin, TX, USA, 2021, pp. 629-634, doi: 10.1109/CDC45484.2021.9683704.
% 4
\bibitem{subash2024} A. Subash, D. Kloeser, J. Frey, R. Reiter, M. Diehl and K. Bohlmann, "Model Predictive Control for Frenet-Cartesian Trajectory Tracking of a Tricycle Kinematic Automated Guided Vehicle," 2024 IEEE/RSJ International Conference on Intelligent Robots and Systems (IROS), Abu Dhabi, United Arab Emirates, 2024, pp. 11948-11953, doi: 10.1109/IROS58592.2024.10802822.
% 5
\bibitem{kamga1996} A. Kamga and A. Rachid, "Speed, steering angle and path tracking controls for a tricycle robot," Proceedings of Joint Conference on Control Applications Intelligent Control and Computer Aided Control System Design, Dearborn, MI, USA, 1996, pp. 56-61, doi: 10.1109/CACSD.1996.555197.
% 6
\bibitem{Precup2006} R. -e. Precup, S. Preitl and Z. Preitl, "Fuzzy Control Solution for a Class of Tricycle Mobile Robots," 2006 IEEE International Conference on Mechatronics, Budapest, Hungary, 2006, pp. 203-208, doi: 10.1109/ICMECH.2006.252525.
% 7
\bibitem{Widiawan2020} B. Widiawan, C. Triwidiarto, S. Kautsar, and R. Firgiyanto, "Wireless greenhouse monitoring system using Tricycle Mobile-Robot based on Rasberry PI," IOP Conference Series Earth and Environmental Science, vol. 411, no. 1, p. 012058, Jan. 2020, doi: 10.1088/1755-1315/411/1/012058.
% 8
\bibitem{VazquezRodriguez2018} E. A. Vázquez-Rodríguez, F. J. Perez-Pinal, P. Bautista-Camino and M. A. Rodríguez-Licea, "The Rollover Risk and its Mitigation in Rickshaws," 2018 IEEE International Autumn Meeting on Power, Electronics and Computing (ROPEC), Ixtapa, Mexico, 2018, pp. 1-6, doi: 10.1109/ROPEC.2018.8661423.
% 9
\bibitem{Licea2018} M. A. R. Licea, E. A. V. Rodríguez, F. J. P. Pinal, and J. P. Olivares, "The rollover risk in Delta tricycles: a new rollover index and its robust mitigation by rear differential braking," Mathematical Problems in Engineering, vol. 2018, pp. 1-14, Oct. 2018, doi: 10.1155/2018/4972419.
% 10
\bibitem{Ketemaw2022} D. Ketemaw and S. Seid, "Design of a rollover index-based sliding mode controller for roll stability of three wheeled vehicles using rear differential braking," Frontiers in Mechanical Engineering, vol. 8, Nov. 2022, doi: 10.3389/fmech.2022.1038289.

% ======================================================================

% ----- CẤU TRÚC DẪN ĐỘNG TRƯỢT -----

% ======================================================================

% ----- CẤU TRÚC ĐA HƯỚNG -----

% ======================================================================

% ----- CẤU TRÚC BÁNH XÍCH -----

\bibitem{wong2008} J. Y. Wong, Theory of Ground Vehicles, 4th ed. John Wiley \& Sons, 2008.
\bibitem{murphy2014} R. R. Murphy, Disaster robotics. 2014. doi: 10.7551/mitpress/9407.001.0001.
\bibitem{aljarrah2019} A. Al-Jarrah, M. Salah and F. Almomani, "Controlling a Skid-Steered Tracked Mobile Robot with Slippage Using Various Control Schemes," 2019 20th International Conference on Research and Education in Mechatronics (REM), Wels, Austria, 2019, pp. 1-7, doi: 10.1109/REM.2019.8744123.
\bibitem{bekker1969} M. G. Bekker, Introduction to Terrain-Vehicle Systems. University of Michigan Press, 1969

% ======================================================================

% ----- CẤU TRÚC TỰ CÂN BẰNG -----

% ======================================================================

% ----- CẤU TRÚC LAI CƠ CẤU CHÂN - BÁNH -----

% ======================================================================

% ----- ỨNG DỤNG -----

\bibitem{jackal} "Jackal UGV - Small Weatherproof Robot - Clearpath," Clearpath Robotics, Jun. 09, 2025. https://www.clearpathrobotics.com/jackal-small-unmanned-ground-vehicle/ (accessed Nov. 20, 2025).

% ====================

% ----- XU HƯỚNG NGHIÊN CỨU -----



% ===============================

% ----- KẾT LUẬN -----



% ====================

\end{thebibliography}

\end{document}