\documentclass[conference]{IEEEtran}
\IEEEoverridecommandlockouts

\usepackage{cite}
\usepackage{amsmath,amssymb,amsfonts}
\usepackage{algorithmic}
\usepackage{graphicx}
\usepackage{float}
\usepackage{textcomp}
\usepackage{xcolor}

\def\BibTeX{{\rm B\kern-.05em{\sc i\kern-.025em b}\kern-.08em
    T\kern-.1667em\lower.7ex\hbox{E}\kern-.125emX}}

\usepackage[utf8]{inputenc}
\usepackage[T1, T5]{fontenc}
\usepackage{babel}
\usepackage{booktabs}

\babelprovide[import,main]{vietnamese}

\renewcommand{\IEEEkeywordsname}{Từ khóa}

\newcommand{\pl}[1]{{\fontencoding{T1}\selectfont #1}}

\begin{document}

\title{Phân loại bánh xe và các cấu trúc của robot di động}

\author{\IEEEauthorblockN{Nguyễn Văn Diễn}
%\IEEEauthorblockA{\textit{22027541} \\
%\textit{QH-2022-I/CQ-E-RE}}
\and
\IEEEauthorblockN{Hoàng Văn Cường}
%\IEEEauthorblockA{\textit{22027549} \\
%\textit{QH-2022-I/CQ-E-RE}}
\and
\IEEEauthorblockN{Vũ Đức Hiếu}
%\IEEEauthorblockA{\textit{22027527} \\
%\textit{QH-2022-I/CQ-E-RE}}
}

\maketitle

\thispagestyle{plain} % Bật số trang cho trang tiêu đề
\pagestyle{plain}    % Bật số trang cho các trang còn lại

\begin{abstract}

\end{abstract}

\begin{IEEEkeywords}
Robot di động, Phân loại, Động học robot, Cơ cấu bánh xe, Robot đa hướng
\end{IEEEkeywords}

\section{Giới thiệu}
Trong thập kỷ qua, lĩnh vực robot di động đã chứng kiến sự phát triển vượt bậc, 
trở thành một thành phần không thể thiếu trong kỷ nguyên Công nghiệp 4.0.
Từ các robot tự hành (AGV) vận chuyển hàng hóa trong kho bãi, robot dịch vụ trong bệnh viện,
đến các phương tiện thám hiểm tự hành trên các hành tinh khác, robot di động đang giải phóng 
con người khỏi các công việc lặp lại, nặng nhọc và nguy hiểm \cite{siegwart2011, fragapane2021}.

Một trong những yếu tố nền tảng quyết định khả năng vận hành của robot di động là 
cơ chế di chuyển. Mặc dù có nhiều phương thức di chuyển khác nhau như sử dụng chân, 
bay, hoặc bơi, cơ cấu sử dụng bánh xe vẫn là giải pháp phổ biến và chiếm ưu thế nhất 
trong hầu hết các ứng dụng trên mặt đất \cite{chung2008}. Ưu điểm vượt trội của robot bánh xe 
nằm ở hiệu suất năng lượng cao, khả năng chịu tải lớn, cơ khí đơn giản và 
thuật toán điều khiển tương đối hoàn thiện so với các cơ chế khác \cite{todd2012}.

Tuy nhiên, sự đa dạng về yêu cầu ứng dụng — từ di chuyển tốc độ cao trên đường nhựa 
đến khả năng xoay sở linh hoạt trong không gian chật hẹp — đã dẫn đến sự ra đời 
của hàng loạt thiết kế bánh xe và cấu trúc khung khác nhau. 
Việc lựa chọn giữa bánh xe tiêu chuẩn (standard wheel), bánh xe đa hướng (omnidirectional wheel), 
hay các cấu trúc dẫn động vi sai (differential drive), cấu trúc lái Ackerman... 
không chỉ ảnh hưởng đến khả năng cơ động mà còn quyết định mô hình động học và độ phức tạp của hệ thống điều khiển \cite{campion1993, lynch2017}.

Mặc dù đã có nhiều nghiên cứu riêng lẻ về từng loại robot, 
nhưng việc tổng hợp và hệ thống hóa mối liên hệ giữa đặc tính linh kiện và cấu trúc động học trong một tài liệu duy nhất là rất cần thiết để có cái nhìn toàn diện.
Bài báo này nhằm mục đích cung cấp một cái nhìn toàn diện và phân loại chi tiết về các hệ thống bánh xe và cấu trúc robot di động hiện có.

Cấu trúc của bài báo được tổ chức như sau: 
Phần II trình bày phân loại chi tiết các loại bánh xe, từ bánh xe tiêu chuẩn đến các thiết kế bánh xe đa hướng và đặc biệt. 
Phần III phân tích các cấu trúc robot di động phổ biến dựa trên mô hình động học của chúng. 
Phần IV tổng hợp và so sánh các đặc tính kỹ thuật. 
Cuối cùng, Phần V và Phần VI thảo luận về các ứng dụng thực tế, xu hướng nghiên cứu tương lai và kết luận.

%---------------------------------

\section{Phân loại bánh xe}
Bánh xe là cơ chế di chuyển được sử dụng nhiều nhất trong robot di động và trong các phương tiện do con người tạo ra \cite{siegwart2011}.
Việc sử dụng bánh xe đơn giản và tiết kiệm chi phí hơn so với cơ chế di chuyển bằng chân, đồng thời cũng dễ dàng hơn trong việc thiết kế, chế tạo và lập trình khi di chuyển trên bề mặt phẳng, không gồ ghề \cite{rubio2019}.
Nó có hiệu suất làm việc cao và sử dụng ít năng lượng hơn so với các cơ chế di chuyển khác (\textit{Hình \ref{fig:fig1}}).
Bánh xe có thể chia thành 3 loại chính gồm \textit{bánh xe tiêu chuẩn}, \textit{bánh xe đa hướng} và \textit{bánh xe đặc biệt} \cite{chung2008}.

\begin{figure}[!htbp]
    \centering
    \includegraphics[width=0.3\textwidth]{figures/fig1.pdf}
    \caption{Biểu đồ giữa tốc độ và năng lượng của các cơ chế di chuyển của robot \cite{todd2012}}
    \label{fig:fig1}
\end{figure}

\subsection{Bánh xe tiêu chuẩn (standard wheel)}
Bánh xe tiêu chuẩn là nhóm bánh xe phổ biến và cơ bản nhất trong robot di động. 
Đặc điểm chung của nhóm này là chúng tạo ra các ràng buộc phi đa hướng (non-holonomic constraints) \cite{rasam2016, siegwart2011}, 
nghĩa là tại một thời điểm, 
bánh xe chỉ có thể lăn hiệu quả theo một hướng và không thể trượt theo phương của trục bánh xe trong điều kiện lý tưởng. 

\subsubsection{Bánh xe cố định (fixed wheel)}
Bánh xe cố định là loại bánh xe đơn giản nhất, được gắn chặt vào khung robot với một hướng duy nhất không thể thay đổi \cite{mikova2016, leong2022}.
Nó chỉ có một bậc tự do (degree of freedom) là quay quanh trục chính của nó \cite{chung2008, mikova2013}.
Nó có cấu trúc đơn giản, nhiều kích thước và có độ tin cậy cao \cite{shabalina2018}.

\begin{figure}[!htbp]
    \centering
    \includegraphics[width=0.4\textwidth]{figures/fig2.pdf}
    \caption{Mô hình động học tổng quát mô tả một bánh xe tiêu chuẩn được gắn trên khung robot \cite{chung2008}}
    \label{fig:fig2}
\end{figure}

Hình \ref{fig:fig2} trình bày sơ đồ động học tổng quát của một bánh xe tiêu chuẩn.
Trong sơ đồ, vị trí của tâm bánh xe $A$ được xác định trong hệ toạ độ cực
bởi các thông số khoảng cách $l$ và góc $\alpha$ so với hệ toạ độ gắn trên khung robot $\{X_R, Y_R\}$.
Hướng của bánh xe được xác định bởi góc $\beta$, là góc hợp bởi đường thẳng
nối tâm khung robot $P$ với tâm bánh xe $A$ và trục của bánh xe.

Đối với \textit{bánh xe cố định}, cả ba thông số ($l$, $\alpha$, và $\beta$)
đều là các hằng số thiết kế và không thay đổi trong quá trình robot di chuyển.
Bánh xe có bán kính $r$ và chuyển động của nó được định nghĩa bằng góc thay đổi theo thời gian $\varphi (t)$.

Về mặt toán học, \textit{bánh xe cố định} áp đặt hai ràng buốc động học chính 
lên robot trong điều kiện lý tưởng \cite{siegwart2011,mikova2016}:
\begin{itemize}
    \item[--] Ràng buộc lăn: Vận tốc tại điểm tiếp xúc mặt đất phải bằng không
    (lăn không trượt), điều này liên quan đến vận tốc quay của bánh xe:
    \begin{equation} \label{eq1}
        \begin{bmatrix}
            \sin(\alpha + \beta) & -cos(\alpha + \beta) & -l\cos{\beta}
        \end{bmatrix}
        R(\theta) \dot{\xi}_I - r\dot{\varphi} = 0
    \end{equation}
    \item[--] Ràng buộc trượt ngang: Vận tốc của tâm bánh xe (điểm $A$) theo phương
    vuông góc với trục bánh xe phải bằng không:
    \begin{equation} \label{eq2}
        \begin{bmatrix}
            \cos(\alpha + \beta) & \sin(\alpha + \beta) & l \sin{\beta}
        \end{bmatrix}
        R(\theta) \dot{\xi}_I = 0
    \end{equation}

    Với $\dot{\varphi}$ là vận tốc góc của bánh và  $\dot{\xi_I} = \begin{bmatrix} \dot{x} \\ \dot{y} \\ \dot{\psi} \end{bmatrix}$ là vectơ biểu diễn chuyển động của robot
    ($x$, $y$ và $\psi$ lần lượt là vị trí và hướng của robot). 
\end{itemize}

Từ phương trình ràng buộc (\ref{eq2}), ta có thể thấy rằng bánh xe cố định triệt tiêu hoàn toàn 
khả năng di chuyển tức thời theo phương ngang trục bánh.
Tính chất này biến robot sử dụng bánh xe cố định thành hệ thống phi đa hướng.
Về mặt điều khiển, điều này có nghĩa là không gian điều khiển (số lượng động cơ) nhỏ hơn không gian cấu hình (số bậc tự do vị trí),
buộc robot phải thực hiện các quỹ đạo phức tạp (như đỗ xe song song) để đến được vị trí
mong muốn thay vì di chuyển thẳng []. Tuy nhiên, chính ràng buộc này lại mang lại sự ổn định hướng
vượt trội khi di chuyển ở tốc độ cao, giúp robot giữ lộ trình tốt hơn so với các loại bánh đa hướng.

Các phương trình (\ref{eq1}) và (\ref{eq2}) dựa trên giả thuyết lý tưởng về "lăn không trượt" (pure rolling)
và tiếp xúc điểm cứng tuyệt đối (rigid point contact). 
Tuy nhiên, trong các ứng dụng thực tế, đặc biệt là với các robot vận tải (AGV) mang tải trọng lớn 
hoặc robot nông nghiệp di chuyển trên địa hình mềm, giả thuyết này thường bị vi phạm:

\begin{itemize}
    \item Hiện tượng trượt: Khi lực kéo vượt quá giới hạn ma sát nghỉ, hiện tượng trượt dọc sẽ xảy ra, dẫn đến sai số tích luỹ trong bài toán định vị. Ngoài ra, khi robot quay vòng, hiện tượng trượt ngang cũng xuất hiện do lực ly tâm, tạo ra góc trượt làm lệch hướng di chuyển thực tế so với hướng bánh xe.
    \item Biến dạng lốp: Diện tích tiếp xúc thực tế là một vùng thay vì một điểm, tạo ra mô-men cản lăn làm tiêu tốn năng lượng \cite{wong2022, iagnemma2000}.
\end{itemize}

Do đó, để giải quyết các thách thức này, kỹ thuật robot hiện đại thường tiếp cận theo hai hướng song song:
\begin{enumerate}
    \item Các mô hình động học được bổ sung thêm các tham số về độ trượt và độ cứng lốp để phản ánh chính xác hơn tương tác vật lý \cite{ward2008, yi2009, mandow2007}.
    \item Vì mô hình hóa không thể loại bỏ hoàn toàn sai số tích lũy, dữ liệu từ bánh xe (Odometry) bắt buộc phải được kết hợp với các cảm biến ngoại vi (như IMU, LiDAR) thông qua các bộ lọc ước lượng trạng thái (như EKF) để đảm bảo độ chính xác vị trí trong thời gian dài \cite{thrun2010, moore2014}.
\end{enumerate}

Nhờ sự kết hợp giữa độ tin cậy cơ khí cao và các giải pháp điều khiển/định vị tiên tiến này, bánh xe cố định vẫn giữ vững vai trò là nền tảng cốt lõi cho các cấu trúc robot phổ biến như dẫn động vi sai và dẫn động trượt \cite{parhi2011}.

\subsubsection{Bánh xe lái (steered standard wheel)}
Bánh xe lái phức tạp hơn bánh cố định. Ngoài bậc tự do quay quanh trục của bánh xe, 
nó còn có thêm một bậc tự do thứ hai là khả năng quay quanh trục thẳng đứng (trục lái) đi qua tâm của bánh xe và điểm tiếp xúc với mặt đất \cite{parhi2011}.
Một động cơ lái thường được sử dụng để chủ động thay đổi góc của bánh xe so với khung của robot \cite{ueno2017}.
Mặc dù có thể thay đổi hướng, tại bất kỳ thời điểm nào, 
nó vẫn là một bánh xe tiêu chuẩn và phải tuân thủ ràng buộc lăn không trượt theo phương mà nó đang hướng theo.
Đây là thành phần cốt lõi trong cấu trúc lái Ackerman \cite{qiu2018, gautam2021} (như bánh trước xe ô tô) hoặc trong cấu trúc 3 bánh \cite{patel2021}.

\begin{figure}[!htbp]
    \centering
    \includegraphics[width=0.4\textwidth]{figures/fig3.pdf}
    \caption{Mô hình động học tổng quát mô tả một bánh xe lái được gắn trên khung robot \cite{chung2008}}
    \label{fig:fig3}
\end{figure}

Ràng buộc lăn của bánh (yêu cầu điều khiển vận tốc) \cite{low2005}:
\begin{equation} \label{eq:eq3}
    \begin{split}
        & \begin{bmatrix}
            \sin(\alpha + \beta(t)) & -\cos(\alpha + \beta(t)) & -l \cos(\beta(t))
        \end{bmatrix} R(\theta)\dot{\xi}_I \\
        & \hspace{6.5cm} - r\dot{\varphi} = 0
    \end{split}
\end{equation}

Ràng buộc trượt của bánh (yêu cầu điều khiển vị trí):
\begin{equation} \label{eq:eq4}
    \begin{split}
        \begin{bmatrix}
            \cos(\alpha + \beta(t)) & \sin(\alpha + \beta(t)) & l \sin(\beta(t))
        \end{bmatrix} R(\theta)\dot{\xi}_I = 0
    \end{split}
\end{equation}

So với bánh xe cố định, việc biến góc $\beta$ thành một biến số theo thời gian $\beta(t)$ mang lại sự linh hoạt vượt trội trong việc định hình quỹ đạo. Tuy nhiên, điều này dẫn đến những thách thức kỹ thuật đặc thù mà các hệ thống robot hiện đại phải giải quyết:

\begin{enumerate}
    \item Ma sát xoay và Hiện tượng "Cày lốp" (Scrubbing): Một vấn đề vật lý nghiêm trọng của bánh xe lái là hiện tượng ma sát xoay khi robot cần đổi hướng mà vận tốc lăn thấp hoặc bằng không. Để thay đổi góc $\beta$, động cơ lái phải thắng được mô-men ma sát trượt xoay (twisting friction) rất lớn giữa bề mặt lốp và mặt đường \cite{gillespie1992}. Điều này không chỉ gây mài mòn lốp nhanh chóng (tire wear) mà còn đòi hỏi động cơ lái phải có mô-men xoắn cực đại lớn, làm tăng kích thước và tiêu thụ năng lượng của hệ thống.
    \item Giới hạn tốc độ lái và Phối hợp điều khiển: Trong lý thuyết, góc lái $\beta$ có thể thay đổi tức thời. Tuy nhiên, trong thực tế, tốc độ thay đổi góc lái ($\beta$) bị giới hạn bởi động lực học của cơ cấu chấp hành. Nếu bộ điều khiển yêu cầu thay đổi hướng quá nhanh so với vận tốc lăn của robot, bánh xe sẽ bị trượt ngang (lateral slip), vi phạm ràng buộc (4). Do đó, các thuật toán điều khiển hiện đại cho robot dùng bánh lái (như Ackerman hoặc Tricycle) đòi hỏi sự phối hợp đồng bộ (coordinated control) chặt chẽ giữa vận tốc lăn $\varphi$ và vận tốc lái $\beta$ để đảm bảo quỹ đạo mượt mà \cite{deluca1998}.
    \item Sai số cơ khí (Backlash): Hệ thống truyền động lái (thường dùng bánh răng hoặc trục vít) luôn tồn tại độ rơ (backlash). Sai số này làm cho góc lái thực tế lệch so với góc lái đo được từ encoder, dẫn đến sai số trong bài toán định vị (Odometry) lớn hơn so với bánh xe cố định.
\end{enumerate}

Ứng dụng điển hình của bánh xe lái là trong các robot nông nghiệp cỡ lớn, xe tự hành ngoài trời hoặc các robot giao hàng tốc độ cao, nơi cần kết hợp giữa lực kéo mạnh và khả năng điều hướng ổn định theo kiểu ô tô.

%------- BÁNH XE XOAY TỰ DO -------------------

\subsubsection{Bánh xe xoay tự do (caster wheel)}
Bánh xe xoay tự do (bánh xe con lăn) cũng có hai bậc tự do tương tự như bánh lái (lăn và xoay quanh trục đứng).
Tuy nhiên, điểm khác biệt cơ bản là nó là một cơ cấu bị động.
Khi robot di chuyển, lực ma sát từ mặt đất sẽ tạo ra một mô-men xoắn, 
khiến bánh xe tự động xoay và căn chỉnh theo hướng di chuyển của robot để giảm thiểu lực cản.
Chúng không cung cấp bất kỳ lực đẩy chủ động nào.
Chúng không dùng để lái hay đẩy, mà chỉ dùng để hỗ trợ và giữ thăng bằng cho robot.
Chúng cực kỳ phổ biến và thường được dùng làm bánh xe phụ trong cấu trúc dẫn động vi sai \cite{arrizabalaga2021}, 
hoặc có thể giúp robot di chuyển lên bậc hoặc bề mặt không bằng phẳng \cite{lee2024, garcia2016}.

\begin{figure}[!htbp]
    \centering
    \includegraphics[width=0.4\textwidth]{figures/fig4.pdf}
    \caption{Mô hình động học tổng quát mô tả một bánh xe xoay tự do được gắn trên khung robot \cite{chung2008}}
    \label{fig:fig4}
\end{figure}

Ràng buộc lăn của bánh xoay tự do:
\begin{equation}
    \begin{split}
        & \begin{bmatrix}
            \sin(\alpha + \beta(t))  & -\cos(\alpha + \beta(t)) & -l \cos{\beta(t)}
        \end{bmatrix}
        R(\theta) \dot{\xi}_I \\
        & \hspace{6.5cm} - r\dot{\varphi} = 0
    \end{split}
\end{equation}

Ràng buộc trượt của bánh xoay tự do:
\begin{equation}
    \begin{split}
        \begin{bmatrix}
            \cos(\alpha + \beta) & \sin(\alpha + \beta) & d + l\sin{\beta}
        \end{bmatrix}
        R(\theta)\dot{\xi}_I + d\dot{\beta} = 0
    \end{split}
\end{equation}

Sự khác biệt cốt lõi giữa bánh Caster và bánh lái tiêu chuẩn nằm ở tham số $d$ (khoảng cách lệch tâm giữa trục lái và điểm tiếp xúc) trong phương trình (6). Sự xuất hiện của thành phần $d\dot{\beta}$ mang ý nghĩa vật lý quan trọng: mọi chuyển động ngang (trượt) của khung robot tại điểm gắn kết đều có thể được triệt tiêu bằng một tốc độ xoay $\dot{\beta}$ phù hợp của bánh xe.
Điều này dẫn đến một kết luận quan trọng: Nếu khớp xoay là tự do (ma sát thấp), bánh xe Caster không áp đặt bất kỳ ràng buộc động học nào lên chuyển động tổng thể của robot [17]. Robot có thể di chuyển theo bất kỳ quỹ đạo nào, và bánh Caster sẽ tự động xoay để "đi theo" (follow) chuyển động đó.

Các vấn đề Hiện đại và Thách thức Động lực học: Mặc dù về mặt hình học (kinematics), bánh Caster rất hoàn hảo để làm bánh phụ trợ, nhưng về mặt động lực học (dynamics), chúng gây ra nhiều vấn đề phức tạp cho các robot hiện đại:

\begin{enumerate}
    \item Hiện tượng Rung lắc (Caster Flutter/Shimmy):Đây là một vấn đề kinh điển nhưng vẫn là thách thức trong thiết kế robot tốc độ cao. Khi di chuyển nhanh, bánh Caster thường xảy ra hiện tượng dao động tự kích (self-excited oscillation) quanh trục lái, gây rung lắc dữ dội cho thân robot. Điều này xuất phát từ sự tương tác phức tạp giữa độ đàn hồi của lốp, độ lệch tâm $d$ và ma sát, điều mà các mô hình động học (5) và (6) không mô tả được [18]. Các robot hiện đại thường phải sử dụng bộ giảm chấn (damper) hoặc thiết kế tối ưu hóa độ cứng lốp để triệt tiêu hiện tượng này.
    \item Hiệu ứng "Đảo chiều" (Turn-around Effect): Khi robot đổi hướng di chuyển đột ngột (ví dụ: từ tiến sang lùi), bánh Caster phải thực hiện một hành trình xoay 180 độ quanh trục đứng để căn chỉnh lại. Quá trình này tạo ra một khoảng "trễ" và một lực phản hồi bất định lên khung robot, gây ra sai số vị trí lớn (Odometry error) và làm quỹ đạo thực tế bị giật cục. Trong các thuật toán định vị hiện đại (SLAM), hành vi khó đoán này của bánh Caster thường được coi là nhiễu (noise).
    \item Xu hướng mới: Active Caster (Bánh Caster chủ động): Để khắc phục tính bị động, một xu hướng nghiên cứu hiện đại là Active Caster (hay Powered Caster). Thay vì để bánh tự xoay do ma sát, người ta gắn động cơ vào cả trục lăn và trục lái. Khi đó, hệ thống trở thành một mô đun truyền động đa hướng (omnidirectional drive) mạnh mẽ, cho phép robot di chuyển holonomic mà không cần dùng bánh Mecanum đắt tiền [19].
\end{enumerate}

Tóm lại, bánh Caster là giải pháp rẻ tiền và hiệu quả để duy trì cân bằng tĩnh, nhưng sự phức tạp về động lực học của nó đòi hỏi sự cân nhắc kỹ lưỡng trong thiết kế robot chính xác cao.

%-----


% -----------------------

\subsection{Bánh xe đa hướng (omnidirectional wheel)}
Bánh xe đa hướng, hay còn gọi là bánh xe toàn hướng, là một bước tiến quan trọng so với bánh xe tiêu chuẩn.
Chúng được thiết kế để khắc phục các ràng buộc phi đa hướng của bánh xe tiêu chuẩn nên chúng ngày càng phổ biến
trong robot di động vì robot có thể đi thẳng từ điểm này đến điểm khác \cite{ignatiev2016}.
Một robot sử dụng các bánh xe này có thể di chuyển theo bất kỳ hướng nào (tiến/lùi, sang ngang, chéo) 
và xoay tại chỗ một cách đồng thời mà không cần phải xoay định hướng thân robot trước.
Khả năng này đạt được bằng cách gắn các con lăn bị động vào chu vi của bánh xe chính.

\subsubsection{Bánh xe omni (omni wheel)}
Bánh xe Omni có cấu tạo gồm một vành bánh xe chính, trên đó gắn các con lăn nhỏ bị động.
Điểm mấu chốt trong thiết kế của bánh Omni là các trục quay của con lăn này được đặt vuông góc (90 độ) 
so với trục quay chính của bánh xe [..].
Khi bánh xe chính quay (do động cơ tác động), nó tạo ra lực đẩy theo hướng tiến hoặc lùi. 
Tuy nhiên, nhờ các con lăn bị động, bánh xe có thể trượt tự do theo phương ngang (phương của trục bánh xe) với ma sát rất nhỏ.
Ưu điểm của bánh Omni là cấu tạo cơ khí tương đối đơn giản hơn so với bánh Mecanum. 
Tuy nhiên, nhược điểm là các con lăn bố trí rời rạc có thể gây ra hiện tượng rung, xóc khi di chuyển trên các bề mặt không hoàn toàn bằng phẳng [..].

\subsubsection{Bánh xe Mecanum (Mecanum wheel)}
Bánh xe Mecanum là bánh có thiết kế đặc biệt được phát minh bởi Bengt Ilon tại công ty Mecanum AB của Thụy Điển \cite{wikipedia_mecanum_wheel}.
Tương tự bánh omni, nó cũng có các con lăn bị động gắn trên vành bánh chính. Tuy nhiên, điểm khác biệt cốt lõi là các con lăn này được đặt nghiêng một góc (thường là 45 độ) so với trục quay chính của bánh xe.
Chính góc nghiêng 45 độ này là yếu tố quyết định. Khi bánh xe quay, lực đẩy tạo ra không chỉ có thành phần tiến/lùi mà còn có cả thành phần lực theo phương ngang.
Bánh Mecanum cung cấp khả năng cơ động linh hoạt bậc nhất.
Tuy nhiên, chúng có nhược điểm là độ phức tạp cao trong cả chế tạo và điều khiển (đòi hỏi mô hình động học ngược phức tạp để "trộn" vận tốc). Chúng cũng rất nhạy cảm với bề mặt, yêu cầu bề mặt phải rất bằng phẳng và sạch sẽ để đạt hiệu suất tối ưu [..].

%-----

\subsection{Bánh xe đặc biệt}
Nhóm này bao gồm các thiết kế bánh xe mang tính thử nghiệm, đột phá hoặc mới lạ, 
thường tập trung vào các lĩnh vực nghiên cứu cụ thể như robot tự cân bằng hoặc robot có khả năng thích ứng địa hình.

\subsubsection{Bánh hình cầu}
Bánh hình cầu là một cơ cấu bánh xe sử dụng một quả cầu để tiếp xúc với mặt đất. 
Thiết kế này về cơ bản cho phép chuyển động đa hướng.
Chúng ta có thể phân biệt hai loại chính: bánh cầu bị động và bánh cầu chủ động.

\begin{itemize}
    \item Bánh cầu bị động: Nó hoạt động như một bánh xe xoay tự do đa hướng, bị động.
          Thường được dùng làm bánh xe hỗ trợ (thứ ba hoặc thứ tư) để giữ thăng bằng cho các robot có cấu trúc dẫn động vi sai.
    \item Bánh cầu chủ động: Đây là một cơ cấu cơ khí cực kỳ phức tạp, trong đó quả cầu được chủ động điều khiển để tạo lực đẩy.
          Quả cầu được giữ trong một hốc (socket) và tiếp xúc trực tiếp với các con lăn (rollers) 
          hoặc động cơ được đặt bên trong cơ cấu. Bằng cách điều khiển các con lăn này 
          (ví dụ, hai con lăn đặt vuông góc với nhau), cơ cấu có thể làm quả cầu quay theo bất kỳ trục nào trên mặt phẳng \cite{lauwers2006, mukherjee1999}.
          Một bánh xe cầu chủ động là một cơ cấu truyền động đa hướng (holonomic) hoàn chỉnh. 
          Một robot được trang bị 3 hoặc 4 bánh xe này có thể di chuyển theo mọi hướng và xoay đồng thời, 
          tương tự như bánh Mecanum nhưng tiềm năng về độ mượt mà và khả năng vượt địa hình gồ ghề (nhẹ) tốt hơn \cite{nagarajan2009}.
          Ưu điểm là khả năng cơ động đa hướng tuyệt đối. 
          Nhược điểm là độ phức tạp cơ khí cực cao, chi phí lớn, và khó khăn trong việc duy trì lực bám và chống mài mòn \cite{kumagai2008}.
\end{itemize}

\begin{figure}[!htbp]
    \centering
    \includegraphics[width=0.4\textwidth]{figures/fig5.pdf}
    \caption{Mô hình động học tổng quát mô tả một bánh xe hình cầu được gắn trên khung robot \cite{chung2008}}
    \label{fig:fig5}
\end{figure}

Quan sát mô hình động học (\textit{Hình \ref{fig:fig5}}), ta thấy các tham số hình học ($l$, $\alpha$, $\beta$) tương tự như \textit{bánh xe cố định}. Tuy nhiên, ý nghĩa vật lý của các phương trình ràng buộc lại hoàn toàn khác biệt \cite{siegwart2011}:

Phương trình lăn:
    \begin{equation} \label{eq7}
        \begin{bmatrix}
            \sin(\alpha + \beta) & -cos(\alpha + \beta) & -l\cos{\beta}
        \end{bmatrix}
        R(\theta) \dot{\xi}_I - r\dot{\varphi} = 0
    \end{equation}

Phương trình trượt:
    \begin{equation} \label{eq8}
        \begin{bmatrix}
            \cos(\alpha + \beta) & \sin(\alpha + \beta) & l \sin{\beta}
        \end{bmatrix}
        R(\theta) \dot{\xi}_I = 0
    \end{equation}

Đối với \textit{bánh xe cố định}, góc $\beta$ là hằng số, nên phương trình (\ref{eq8}) là một ràng buộc ngăn cản chuyển động ngang. Ngược lại, đối với \textit{bánh hình cầu}, góc $\beta$ là một biến tự do. Do đó, phương trình (\ref{eq8}) không hạn chế chuyển động của robot ($\dot{\xi}_I$), mà nó dùng để xác định hướng quay tức thời của quả cầu sao cho phù hợp với hướng di chuyển của robot. Nói cách khác, \textit{bánh hình cầu} không áp đặt bất kỳ ràng buộc động học nào lên khung robot, cho phép robot di chuyển toàn hướng.

Thách thức hiện đại: Mặc dù lý thuyết động học rất lý tưởng, việc hiện thực hóa \textit{bánh hình cầu} chủ động gặp thách thức lớn về hiệu suất truyền động. Lực đẩy phụ thuộc hoàn toàn vào ma sát tiếp xúc điểm giữa con lăn dẫn động và quả cầu. Nếu ma sát không đủ sẽ gây trượt, nếu quá lớn sẽ gây mài mòn và tổn hao năng lượng. Đây là bài toán khó về vật liệu và cơ khí chính xác mà các nghiên cứu hiện nay đang tập trung giải quyết \cite{kumagai2008}.

%-----

\subsubsection{Bánh origami}
Đây là một hướng nghiên cứu mới, ứng dụng các nguyên lý của nghệ thuật gấp giấy Nhật Bản (Origami) để tạo ra các bánh xe có thể thay đổi hình dạng.
Ưu điểm lớn nhất là khả năng thích ứng địa hình (terrain adaptability).
Nhược điểm là độ phức tạp về cơ khí, độ bền của các khớp gấp, và sự phức tạp trong hệ thống điều khiển để quyết định khi nào cần thay đổi hình dạng [..].

Cơ chế và Nguyên lý Biến hình: Hầu hết các bánh xe Origami hiện đại dựa trên các mẫu gấp (tessellation patterns) nổi tiếng như mẫu "Bóng ma thuật" (Magic Ball pattern) hoặc "Waterbomb base" [].

\begin{itemize}
    \item Cấu trúc: Bánh xe được cấu tạo từ một màng mỏng (thường là composite hoặc polymer) được gấp nếp theo quy luật hình học xác định.
    
    \item Hoạt động: Một cơ cấu chấp hành tuyến tính (linear actuator) ở trục bánh xe sẽ đẩy hai đĩa ốp hai bên lại gần hoặc ra xa nhau. Nhờ cấu trúc gấp nếp, chuyển động tuyến tính này sẽ được chuyển đổi thành chuyển động hướng kính, làm thay đổi đường kính của bánh xe.
    \begin{itemize}
        \item Trạng thái thu gọn: Đường kính nhỏ, bề mặt trơn nhẵn → Di chuyển nhanh, hiệu suất cao trên mặt phẳng.
        \item Trạng thái mở rộng: Đường kính lớn, bề mặt xuất hiện các "gai" hoặc mấu bám → Tăng mô-men vượt chướng ngại vật, leo cầu thang hoặc đi trên đất mềm.
    \end{itemize}
\end{itemize}

Đặc tính Động học Biến thiên (Variable Kinematics): Điểm khác biệt "hiện đại" nhất so với bánh xe truyền thống là bán kính bánh xe r không còn là hằng số mà là một hàm số phụ thuộc vào trạng thái biến hình u(t). Phương trình ràng buộc lăn (1) trở thành phi tuyến:

Điều này đặt ra thách thức lớn cho bài toán Odometry và điều khiển PID, vì hệ số khuếch đại của hệ thống thay đổi liên tục khi bánh xe biến hình. Bộ điều khiển cần phải thích nghi (Adaptive Control) để cập nhật lại mô hình động học theo thời gian thực.

Các vấn đề Hiện đại và Thách thức Kỹ thuật: Nghiên cứu về bánh xe Origami đang đối mặt với 3 bài toán lớn:

\begin{itemize}
    \item Mâu thuẫn giữa Độ cứng và Độ linh hoạt (Stiffness vs. Flexibility Trade-off): Để biến hình, vật liệu cần mềm dẻo (flexible). Nhưng để chịu tải trọng của robot (Payload), bánh xe cần phải cứng (stiff).
    \begin{itemize}
        \item Giải pháp hiện đại: Các nghiên cứu mới nhất (như của ĐH Quốc gia Seoul) đề xuất cơ chế khóa cấu trúc (structural locking). Khi ở trạng thái mong muốn, bánh xe tự khóa cứng lại để chịu tải lên tới hàng trăm kg, nhưng khi cần biến hình thì mở khóa để trở nên mềm dẻo [Ref].
    \end{itemize}

    \item Hiệu ứng Đa giác (Polygon Effect): Khi mở rộng, bánh xe Origami thường không giữ được hình tròn hoàn hảo mà tạo thành hình đa giác. Điều này gây ra rung động chu kỳ (vibration) khi robot di chuyển, ảnh hưởng đến độ ổn định của camera hoặc cảm biến trên thân robot.
    
    \item Độ bền mỏi (Fatigue Durability): Các nếp gấp (hinges) là nơi tập trung ứng suất. Sau hàng nghìn chu kỳ gấp/mở, vật liệu rất dễ bị nứt gãy. Công nghệ vật liệu hiện đại đang hướng tới các loại vải gia cường (fabric-reinforced composites) để tăng tuổi thọ cho bánh xe.
\end{itemize}

Ứng dụng tiềm năng nhất của công nghệ này là các robot giao hàng (last-mile delivery) có khả năng tự leo vỉa hè hoặc robot thám hiểm bề mặt hành tinh với địa hình không xác định.

% ----------

\subsection{Tổng hơp, so sánh các loại bánh xe}

\begin{table*}[t]
  \centering
  \caption{TỔNG HỢP VÀ SO SÁNH CÁC LOẠI BÁNH XE}
  \label{tab:comparison}
  % Bỏ các dấu | thẳng đứng đi, nhìn sẽ thoáng hơn
  \begin{tabular}{llllll} 
    \toprule % Đường kẻ đậm trên cùng
    \textbf{Loại bánh xe} & \textbf{Ràng buộc} & \textbf{Cơ khí} & \textbf{Điều khiển} & \textbf{Địa hình} & \textbf{Năng lượng} \\
    \midrule % Đường kẻ mảnh ở giữa
    Bánh cố định & Non-holonomic & Thấp & Thấp & Cao & Cao \\
    Bánh lái & Non-holonomic & TB & TB & Cao & Cao \\
    Bánh Caster & Bị động & Thấp & N/A & TB & TB \\
    Bánh Omni & Holonomic & TB & Cao & Thấp & TB \\
    Bánh Mecanum & Holonomic & Cao & Rất cao & Thấp & Thấp \\
    Bánh Hình cầu & Holonomic & Rất cao & Rất cao & Thấp & Thấp \\
    Bánh Origami & Biến thiên & Rất cao & Rất cao & Rất cao & TB \\
    \bottomrule % Đường kẻ đậm dưới cùng
  \end{tabular}
\end{table*}

%---------------------------------

\section{Phân loại cấu trúc robot di động}
Sau khi phân tích các loại bánh xe (linh kiện) ở phần II, phần này sẽ đi sâu vào việc phân loại các cấu trúc robot di động (hệ thống). 
Một cấu trúc robot được định nghĩa bởi cách các bánh xe được sắp xếp, loại bánh xe được sử dụng, 
và phương pháp truyền động [..]. Các yếu tố này kết hợp lại sẽ quyết định mô hình động học (kinematic model), 
các ràng buộc chuyển động (constraints), và từ đó, ảnh hưởng trực tiếp đến khả năng cơ động, độ ổn định, và môi trường hoạt động của robot.

\subsection{Cấu trúc dẫn động vi sai}
Đây là cấu trúc phổ biến, đơn giản và được sử dụng rộng rãi nhất cho robot di động trong nhà [].
Cấu trúc này bao gồm hai bánh xe chủ động (driven wheels) được lắp cố định trên cùng một trục và được điều khiển vận tốc một cách độc lập. Để giữ thăng bằng, robot thường sử dụng thêm một hoặc hai bánh xe bị động, thường là bánh xe xoay tự do (Caster Wheel) hoặc bánh bi (Ball Caster) đặt ở phía trước hoặc/và sau [..].
Cấu trúc dẫn động vi sai là một hệ thống phi đa hướng (non-holonomic). Mặc dù nó rất linh hoạt, nó vẫn bị một ràng buộc: robot không thể di chuyển tức thời theo phương ngang (vuông góc với trục bánh xe) [..]. Mô hình động học (cả động học thuận và ngược) của nó tương đối đơn giản và đã được nghiên cứu kỹ, khiến nó trở thành lựa chọn hàng đầu cho các robot cơ bản và trong giáo dục.

\begin{figure}[!htbp]
    \centering
    \includegraphics[width=0.4\textwidth]{figures/fig6.pdf}
    \caption{Mô hình cấu trúc dẫn động vi sai \cite{chung2008}}
    \label{fig:fig6}
\end{figure}

%-----

\subsection{Cấu trúc lái Ackerman}
Cấu trúc này mô phỏng cơ chế lái của các phương tiện giao thông đường bộ phổ biến như ô tô [].
Một cấu hình Ackerman điển hình có bốn bánh xe, bao gồm hai bánh sau chủ động được gắn cố định (Fixed Wheels) và hai bánh trước có thể lái được (Steered Wheels).
Điểm then chốt của cấu trúc này là Hình học lái Ackerman (Ackerman Steering Geometry). Khi robot rẽ, hai bánh xe phía trước không quay song song với nhau. Thay vào đó, bánh xe ở phía bên trong của vòng cua sẽ quay với một góc lớn hơn bánh xe ở phía bên ngoài [..]. Cơ chế này đảm bảo rằng tất cả bốn bánh xe đều lăn quanh một điểm quay tức thời chung (Instantaneous Center of Rotation - ICR), giúp giảm thiểu sự trượt ngang của lốp xe.
Đây là một hệ thống phi đa hướng (non-holonomic) điển hình. Nó không thể di chuyển ngang và không thể xoay tại chỗ (bán kính quay luôn lớn hơn 0).
Ưu điểm: Rất ổn định khi vận hành ở tốc độ cao, giảm mài mòn bánh xe do loại bỏ được phần lớn lực trượt khi rẽ.
Nhược điểm: Bán kính quay lớn, hạn chế sự linh hoạt trong không gian hẹp. Cơ cấu cơ khí (thanh liên kết) để thực hiện hình học Ackerman tương đối phức tạp [..].

%-----

\subsection{Cấu trúc 3 bánh}
Đây là một cấu hình phổ biến khác, cân bằng giữa sự đơn giản và khả năng cơ động.
Cấu trúc này, như tên gọi, sử dụng ba bánh xe. Cấu hình phổ biến nhất là một bánh lái (steered wheel) ở phía trước và hai bánh cố định (fixed wheels) ở phía sau. Bánh lái có thể đồng thời là bánh chủ động (driven) hoặc chỉ có chức năng lái, trong khi hai bánh sau là chủ động hoặc bị động [..].
Hướng di chuyển của robot được quyết định hoàn toàn bởi góc quay của bánh lái phía trước.
Tương tự như cấu trúc Ackerman, đây là một hệ thống phi đa hướng. Tuy nhiên, nếu bánh lái có thể quay được một góc lớn (ví dụ: ±90 độ), robot có thể đạt được bán kính quay rất nhỏ, tiệm cận không.
Ưu điểm: Thiết kế cơ khí đơn giản hơn đáng kể so với Ackerman.
Nhược điểm: Độ ổn định (stability) kém hơn so với cấu trúc 4 bánh, đặc biệt là khi phanh gấp hoặc rẽ ở tốc độ cao, có nguy cơ bị lật [..].

%-----

\subsection{Cấu trúc dẫn động trượt}
Cấu trúc này sử dụng ma sát trượt để thay đổi hướng, tương tự như các phương tiện bánh xích.
Thường sử dụng bốn (hoặc sáu, tám) bánh xe cố định, được gắn chặt vào khung robot và không thể thay đổi hướng. Tất cả các bánh xe ở mỗi bên (trái và phải) thường được kết nối và chủ động đồng thời.
Robot rẽ bằng cách tạo ra chênh lệch vận tốc giữa hai phía của robot, giống hệt như Cấu trúc Dẫn động Vi sai. Ví dụ, để rẽ trái, các bánh xe bên phải quay nhanh hơn bên trái (hoặc bên trái quay lùi). Sự chênh lệch vận tốc này buộc các bánh xe phải trượt (skid) trên mặt đất để robot có thể xoay [..].
Về mặt điều khiển, nó được xem như một hệ thống phi đa hướng tương tự Dẫn động Vi sai, có khả năng xoay tại chỗ (zero turning radius). Tuy nhiên, mô hình động học của nó phức tạp hơn nhiều do sự trượt không thể đoán trước, phụ thuộc rất nhiều vào ma sát và bề mặt địa hình.
Ưu điểm: Cấu trúc cơ khí cực kỳ đơn giản và bền bỉ (không có cơ cấu lái). Rất phù hợp cho địa hình gồ ghề.
Nhược điểm: Tiêu thụ năng lượng rất lớn do ma sát trượt. Gây mài mòn bánh xe và làm hỏng các bề mặt mềm. Việc ước tính vị trí (Odometry) rất không chính xác do độ trượt cao [..].

%-----

\subsection{Cấu trúc đa hướng}
Cấu trúc này cung cấp mức độ cơ động cao nhất trên mặt phẳng 2D.
Cấu trúc này sử dụng các loại bánh xe đặc biệt đã được đề cập ở Phần II.B, phổ biến nhất là 3 bánh Omni (đặt cách nhau 120 độ) hoặc 4 bánh Mecanum (đặt ở 4 góc) [..].
Bằng cách kết hợp vận tốc quay của từng bánh xe đặc biệt, robot có thể tạo ra một véc-tơ vận tốc tổng hợp theo bất kỳ hướng nào (X, Y) và đồng thời xoay quanh trục Z.
Đây là một hệ thống đa hướng (holonomic). Nó có 3 bậc tự do (DoF) trên mặt phẳng, cho phép robot di chuyển sang ngang, đi chéo, và xoay đồng thời mà không cần thay đổi hướng của thân robot. Mô hình động học ngược (từ vận tốc robot mong muốn sang vận tốc từng bánh) là một ma trận "trộn" vận tốc (mixing matrix) [..].
Ưu điểm: Khả năng cơ động tuyệt đối, cực kỳ lý tưởng cho các không gian chật hẹp, phức tạp.
Nhược điểm: Các bánh xe (đặc biệt là Mecanum) đắt tiền và phức tạp. Hệ thống điều khiển phức tạp. Rất nhạy cảm với bề mặt (yêu cầu mặt phẳng), hiệu suất truyền động thấp và dễ bị trượt [..].

%-----

\subsection{Cấu trúc bánh xích}
Cấu trúc này được tối ưu hóa cho khả năng vượt địa hình gồ ghề và bề mặt mềm.
Thay vì bánh xe, robot sử dụng hai dải xích (tracks) song song.
Tương tự như cấu trúc Dẫn động trượt (Skid-Steer), robot được điều khiển bằng cách thay đổi vận tốc tương đối của hai dải xích [..].
Đây là hệ thống phi đa hướng, có khả năng xoay tại chỗ, và phụ thuộc lớn vào sự trượt của dải xích trên mặt đất.
Ưu điểm: Khả năng bám địa hình (traction) vượt trội. Áp suất tác dụng lên mặt đất rất thấp (do diện tích tiếp xúc lớn), cho phép robot di chuyển trên cát, tuyết, hoặc bùn lầy. Khả năng vượt chướng ngại vật và leo dốc rất tốt [..].
Nhược điểm: Tốc độ di chuyển chậm, tiêu thụ năng lượng cao, phá hủy bề mặt di chuyển, và rất khó để ước tính vị trí chính xác [..].

\subsection{Cấu trúc tự cân bằng}
Đây là một nhóm cấu trúc đặc biệt, dựa vào nguyên lý điều khiển chủ động để duy trì sự ổn định.
Các robot này về bản chất là không ổn định. Cấu hình phổ biến nhất là robot 2 bánh (two-wheeled) hoạt động như một con lắc ngược (inverted pendulum), hoặc các robot 1 bánh (Ball-bot) tự cân bằng trên một quả cầu [..].
Robot liên tục sử dụng các cảm biến (thường là IMU - Cảm biến đo quán tính) để đo góc nghiêng của thân. Một vòng lặp điều khiển tốc độ cao (như PID, LQR) sẽ tính toán và ra lệnh cho động cơ (ở bánh xe hoặc quả cầu) di chuyển "về phía" robot đang ngã, nhằm đưa trọng tâm trở lại vị trí cân bằng [..].
Đây là các hệ thống không ổn định, phi tuyến (nonlinear) và thiếu truyền động (underactuated). Chuyển động (tiến/lùi) là hệ quả của việc chủ động "ngã" về phía trước hoặc sau.
Ưu điểm: Cực kỳ nhỏ gọn, diện tích chiếm dụng mặt bằng (footprint) bằng không, rất linh hoạt trong môi trường có con người.
Nhược điểm: Rất phức tạp về mặt điều khiển. Hoàn toàn phụ thuộc vào cảm biến và nguồn điện (mất điện là ngã). Khả năng chịu tải và vượt địa hình bị hạn chế [..].

\subsection{Cấu trúc lai cơ cấu chân - bánh}
Cấu trúc lai chân - bánh đại diện cho sự hội tụ giữa hai cơ chế di chuyển cơ bản: bánh xe (hiệu quả năng lượng cao trên mặt phẳng) và chân (khả năng vượt địa hình phức tạp). Mục tiêu của thiết kế này là kết hợp ưu điểm của cả hai để tạo ra các robot đa năng vượt trội [Ref].

Nguyên lý hoạt động và Cấu hình: Thông thường, cấu trúc này bao gồm các bánh xe chủ động được gắn vào điểm cuối (end-effector) của các chân robot có nhiều bậc tự do (DOF). Robot có thể hoạt động ở hai chế độ chính:

Chế độ lăn (Driving Mode): Trên địa hình bằng phẳng, robot hạ thấp trọng tâm, khóa các khớp chân hoặc sử dụng chúng như hệ thống treo chủ động (active suspension), và di chuyển bằng bánh xe để đạt tốc độ cao và tiết kiệm năng lượng.

Chế độ bước (Walking/Stepping Mode): Khi gặp chướng ngại vật lớn hoặc cầu thang, robot khóa bánh xe và sử dụng chân để bước qua, thực hiện các chuyển động rời rạc [Ref].

Đặc tính Động học và Ưu điểm: Về mặt động học, đây là một hệ thống dư thừa dẫn động (redundant actuated system).

Khả năng thay đổi cấu hình (Reconfigurability): Khác với cấu trúc Ackermann hay Vi sai có hình học cố định, robot lai có thể thay đổi chiều rộng cơ sở (footprint), chiều cao trọng tâm (CoM height) và góc nghiêng thân xe (roll/pitch) một cách chủ động. Điều này cho phép robot duy trì thăng bằng trên các sườn dốc nghiêng mà các robot bánh xe thông thường sẽ bị lật.

Di chuyển Đa hướng (Omnidirectional): Bằng cách xoay trục của chân (yaw joint), các bánh xe có thể được định hướng theo bất kỳ góc nào, cho phép robot di chuyển holonomic tương tự như dùng bánh Mecanum nhưng trên địa hình gồ ghề.

Các vấn đề Hiện đại và Thách thức Điều khiển (Modern Issues): Đây là phần quan trọng nhất, làm nổi bật tính "hiện đại" của báo cáo:

Điều khiển Toàn thân (Whole-Body Control - WBC): Thách thức lớn nhất không nằm ở cơ khí mà ở thuật toán. Robot phải giải quyết bài toán tối ưu hóa phức tạp trong thời gian thực để phối hợp mô-men xoắn của động cơ bánh xe (để di chuyển) và động cơ chân (để giữ thăng bằng và giảm xóc). Các phương pháp hiện đại như Model Predictive Control (MPC) thường được sử dụng để dự đoán trạng thái tương lai và giữ robot ổn định khi di chuyển tốc độ cao trên địa hình lồi lõm [Ref].

Quản lý Chuyển đổi trạng thái (Locomotion Mode Switching): Việc quyết định khi nào nên lăn và khi nào nên bước là một bài toán khó về nhận thức môi trường (perception). Robot cần sử dụng Vision/LiDAR để phân loại địa hình và chuyển đổi mượt mà giữa các mô hình động học khác nhau mà không bị gián đoạn hoặc mất thăng bằng.

Tương tác Động lực học (Hybrid Dynamics): Khác với robot bánh xe truyền thống luôn bám đất, robot lai có những pha "bay" (flight phase) khi nhảy hoặc bước. Việc mô hình hóa các lực va chạm (impact forces) khi bánh xe tiếp đất sau một cú nhảy là rất quan trọng để bảo vệ phần cứng và duy trì quỹ đạo.

Ứng dụng thực tiễn: Các robot tiêu biểu cho xu hướng này bao gồm Boston Dynamics Handle (robot kho vận), ETH Zurich ANYmal on Wheels (robot thám hiểm), và Tencent Ollie (robot nhào lộn). Chúng đang mở ra hướng đi mới cho robot giao hàng chặng cuối (last-mile delivery) nơi robot phải leo lên bậc thềm nhà khách hàng.

%------------- CẤU TRÚC DẪN ĐỘNG BÁNH CẦU ---------------------

\subsection{Cấu trúc dẫn động bánh cầu}
Cấu trúc dẫn động bánh cầu đại diện cho một lớp robot di động đặc biệt, di chuyển và giữ thăng bằng trên một quả cầu duy nhất thay vì nhiều bánh xe. Đây là ví dụ điển hình nhất cho hệ thống robot cân bằng động (dynamically balancing) và thiếu dẫn động (underactuated) trong kỹ thuật robot hiện đại [].

\begin{figure}[!htbp]
    \centering
    \includegraphics[width=0.4\textwidth]{figures/fig1000.pdf}
    \caption{Mô hình cấu trúc dẫn động bánh cầu trong mặt phẳng 2D (\textit{bên trái}) và không gian 3D (\textit{bên phải}) \cite{abdelrahim2025}}
    \label{fig:fig1000}
\end{figure}

Khác với bánh xe thông thường, quả cầu cần quay tự do theo mọi hướng, do đó không thể gắn trục động cơ cố định vào tâm cầu. Thay vào đó, hệ thống sử dụng cơ chế truyền động gián tiếp, thường được gọi là "cơ cấu chuột bi ngược" (inverse mouse-ball drive) \cite{lauwers2006, liao2008}.

\begin{figure}[!htbp]
    \centering
    \includegraphics[width=0.3\textwidth]{figures/fig997.pdf}
    \caption{Mô hình cấu trúc dẫn động bánh cầu trong mặt phẳng 2D (\textit{bên trái}) và không gian 3D (\textit{bên phải}) \cite{abdelrahim2025}}
    \label{fig:fig997}
\end{figure}

\begin{figure}[!htbp]
    \centering
    \includegraphics[width=0.2\textwidth]{figures/fig998.pdf}
    \caption{Robot inverse mouse-ball \cite{kumagai2008}}
    \label{fig:fig998}
\end{figure}

\begin{figure}[!htbp]
    \centering
    \includegraphics[width=0.2\textwidth]{figures/fig999.pdf}
    \caption{Robot inverse mouse-ball \cite{lauwers2006}}
    \label{fig:fig999}
\end{figure}

%--------------- ỨNG DỤNG ------------------

\section{Ứng dụng}
Sự đa dạng của các loại bánh xe và cấu trúc robot di động đã được phản ánh qua vô số ứng dụng thực tế. Việc lựa chọn một cấu hình di động cụ thể luôn là kết quả của bài toán tối ưu hóa giữa: khả năng cơ động, tải trọng, năng lượng tiêu thụ và đặc điểm môi trường hoạt động [].

\begin{enumerate}
    \item Logistics và Kho bãi thông minh: Đây là lĩnh vực ứng dụng lớn nhất của robot di động (AMR/AGV).
    \begin{itemize}
        \item Cấu trúc Dẫn động Vi sai: Chiếm ưu thế trong các robot vận chuyển kệ hàng (như Amazon Kiva/Robotics) nhờ chi phí thấp, độ tin cậy cao và khả năng xoay tại chỗ trong các lối đi vuông góc [].
        \item Cấu trúc Đa hướng (Mecanum/Omni): Đang dần phổ biến trong các nhà máy sản xuất linh kiện điện tử hoặc lắp ráp máy bay, nơi cần di chuyển chính xác các linh kiện lớn trong không gian chật hẹp mà không cần không gian quay đầu (zero turning radius) [].
    \end{itemize}

    \item Giao hàng chặng cuối: Một xu hướng "hiện đại" bùng nổ sau đại dịch.
    \begin{itemize}
        \item Cấu trúc 6 bánh (6-Wheel Differential/Rocker-bogie): Các robot như Starship Technologies sử dụng cấu trúc này để leo lề đường và di chuyển ổn định trên vỉa hè đô thị không bằng phẳng.
        \item Cấu trúc Ackerman: Được dùng cho các robot giao hàng tốc độ cao trên làn đường xe đạp hoặc khu dân cư rộng (như Nuro R2).
    \end{itemize}

    \item Thám hiểm và Nông nghiệp
    \begin{itemize}
        \item Cấu trúc Dẫn động trượt (Skid-steer) \& Rocker-bogie: Là tiêu chuẩn vàng cho các robot thám hiểm hành tinh (như Mars Rovers: Curiosity, Perseverance) nhờ hệ thống cơ khí bền bỉ, không có các khớp lái phức tạp dễ hỏng hóc trong môi trường bụi bặm khắc nghiệt [].
        \item Cấu trúc Bánh xích: Vẫn là lựa chọn số một cho các robot cứu hỏa hoặc nông nghiệp hoạt động trên nền đất bùn lầy (soft soil) để giảm áp suất tiếp xúc.
    \end{itemize}

    \item Robot Dịch vụ và Tương tác người (Service \& HRI):
    \begin{itemize}
        \item Ballbot và Robot tự cân bằng: Nhờ diện tích đế nhỏ (small footprint) và chiều cao lớn, các cấu trúc này (như Rezero, Segway) rất phù hợp để di chuyển trong văn phòng đông người và tương tác ngang tầm mắt mà không gây cản trở giao thông [].
    \end{itemize}
\end{enumerate}

%---------------- XU HƯỚNG NGHIÊN CỨU -----------------

\section{Xu hướng nghiên cứu}
Lĩnh vực cơ cấu di chuyển (locomotion) đang chuyển dịch từ việc thiết kế các cơ cấu cứng nhắc sang các hệ thống thông minh, linh hoạt và thích ứng hơn.

\begin{enumerate}
    \item Cấu trúc Lai và Biến hình (Hybrid \& Reconfigurable Locomotion): Để giải quyết sự đánh đổi giữa tốc độ (bánh xe) và khả năng vượt địa hình (chân), xu hướng Robot lai Chân-Bánh (Wheel-Legged Robots) đang dẫn đầu nghiên cứu hiện đại (ví dụ: ETH Zurich ANYmal on Wheels). Ngoài ra, các nghiên cứu về bánh xe biến hình (Origami wheels) cho phép thay đổi đường kính hoặc hình dạng bánh xe theo thời gian thực để thích ứng với địa hình cát, sỏi hoặc cầu thang đang rất được quan tâm [].
    
    \item Bánh xe Chủ động và Thông minh (Active \& Smart Wheels): Thay vì chỉ là một cơ cấu chấp hành thụ động, bánh xe đang trở thành một mô-đun thông minh.
    \begin{itemize}
        \item Active Caster: Biến bánh xe xoay tự do thành bánh xe dẫn động toàn hướng có điều khiển, loại bỏ nhược điểm rung lắc (shimmy) [].
        \item In-wheel Motors \& Sensors: Tích hợp động cơ và cảm biến mô-men xoắn trực tiếp vào trong lòng bánh xe để thực hiện các thuật toán điều khiển bám đường (traction control) và ước lượng độ trượt (slip estimation) ngay tại điểm tiếp xúc.
    \end{itemize}

    \item Terramechanics và Điều khiển dựa trên Mô hình (Model-based Control): Như đã phân tích ở Phần II, các mô hình động học hình học cổ điển không còn đủ chính xác. Xu hướng hiện nay là tích hợp Cơ học đất (Terramechanics) vào vòng điều khiển. Các bộ điều khiển hiện đại (như MPC - Model Predictive Control) sử dụng dữ liệu thời gian thực để dự đoán sự lún và trượt của đất, từ đó điều phối lực kéo tối ưu cho từng bánh xe, giúp robot không bị sa lầy trên địa hình mềm [].
\end{enumerate}

%----------------- KẾT LUẬN ----------------

\section{Kết luận}
Bài báo cáo này đã trình bày một cái nhìn tổng quan và hệ thống hóa về các loại bánh xe và cấu trúc cơ bản của robot di động, đồng thời phân tích các vấn đề hiện đại liên quan đến động lực học và điều khiển.

Qua phân tích, có thể rút ra các kết luận chính sau:

\begin{enumerate}
    \item Không có thiết kế vạn năng: Sự lựa chọn cấu trúc robot luôn là một sự đánh đổi (design trade-off). Cấu trúc Dẫn động vi sai tối ưu cho sự đơn giản và giá thành; Cấu trúc Đa hướng tối ưu cho sự linh hoạt trong không gian hẹp; trong khi Cấu trúc Skid-steer/Bánh xích tối ưu cho địa hình gồ ghề.
    \item Thách thức từ Thực tế: Các mô hình động học lý tưởng (lăn không trượt) là nền tảng cần thiết nhưng không đủ. Kỹ thuật robot hiện đại buộc phải giải quyết các vấn đề phi tuyến như ma sát xoay (scrubbing), trượt (slippage) và tương tác địa hình thông qua việc mở rộng mô hình và ghép nối cảm biến (Sensor Fusion).
    \item Tương lai là sự Thích ứng: Xu hướng phát triển đang hướng tới các cấu trúc lai (chân-bánh) và các vật liệu thông minh, cho phép robot không chỉ di chuyển trên địa hình mà còn thích ứng với địa hình đó.
\end{enumerate}

Hiểu rõ đặc tính của từng loại bánh xe và cấu trúc là bước đầu tiên và quan trọng nhất để thiết kế nên những hệ thống robot tự hành hiệu quả, bền bỉ và thông minh hơn trong tương lai.

%---------------------------------

\begin{thebibliography}{00}
\bibitem{siegwart2011} R. Siegwart, I. R. Nourbakhsh, and D. Scaramuzza, \textit{Introduction to Autonomous Mobile Robots, second edition}. MIT Press, 2011.
\bibitem{fragapane2021} G. Fragapane, R. de Koster, F. Sgarbossa, and J. O. Strandhagen, “Planning and control of autonomous mobile robots for intralogistics: Literature review and research agenda,” European Journal of Operational Research, vol. 294, no. 2, pp. 405-426, Jan. 2021, doi: https://doi.org/10.1016/j.ejor.2021.01.019.
\bibitem{chung2008} W. Chung and K. Iagnemma,  "Wheeled robots," \textit{Springer Handbook of Robotics}. 2008. doi: 10.1007/978-3-540-30301-5.
\bibitem{todd2012} D. J. Todd, \textit{Walking machines: An Introduction to Legged Robots}. Springer, 2012.
\bibitem{campion1993} G. Campion, G. Bastin and B. D'Andrea-Novel, "Structural properties and classification of kinematic and dynamic models of wheeled mobile robots," [1993] Proceedings IEEE International Conference on Robotics and Automation, Atlanta, GA, USA, 1993, pp. 462-469 vol.1, doi: 10.1109/ROBOT.1993.292023.
\bibitem{lynch2017} K. M. Lynch and F. C. Park, Modern robotics : mechanics, planning, and control. Cambridge: University Press, 2017.
\bibitem{rasam2016} H. R. Rasam, “Review on Land-Based Wheeled Robots,” MATEC Web of Conferences, vol. 53, p. 01058, Jan. 2016, doi: 10.1051/matecconf/20165301058.
\bibitem{rubio2019} F. Rubio, F. Valero, and C. Llopis-Albert, “A review of mobile robots: Concepts, methods, theoretical framework, and applications,” International Journal of Advanced Robotic Systems, vol. 16, no. 2, Mar. 2019, doi: 10.1177/1729881419839596.
\bibitem{mikova2016} Ľ. Miková and A. Gmiterko, “Kinematic model and control algorithm for the path tracking of nonholonomic mobile robots,” Journal of Automation and Control, vol. 4, no. 2, pp. 26–29, Dec. 2016, doi: 10.12691/automation-4-2-4.
\bibitem{leong2022} J. S. Ling Leong, K. T. Kin Teo and H. P. Yoong, "Four Wheeled Mobile Robots: A Review," \textit{2022 IEEE International Conference on Artificial Intelligence in Engineering and Technology (IICAIET)}, pp. 1-6, Sep. 2022, doi: 10.1109/IICAIET55139.2022.9936855.
\bibitem{mikova2013} L. Miková, F. Trebuňa and M. Čurilla, "Model of mechatronic system's undercarriage created on the basis of its dynamics," 2013 International Conference on Process Control (PC), Strbske Pleso, Slovakia, 2013, pp. 231-234, doi: 10.1109/PC.2013.6581414.
\bibitem{shabalina2018} K. Shabalina, A. Sagitov and E. Magid, "Comparative Analysis of Mobile Robot Wheels Design," 2018 11th International Conference on Developments in eSystems Engineering (DeSE), Cambridge, UK, 2018, pp. 175-179, doi: 10.1109/DeSE.2018.00041.
\bibitem{wong2022} J. Y. Wong, "Mechanics of Vehicle-Terrain Interaction," Theory of ground vehicles. 2022. doi: 10.1002/9781119719984.
\bibitem{iagnemma2000} K. Iagnemma and S. Dubowsky, “Mobile Robot Rough-Terrain Control (RTC) for Planetary Exploration,” Sep. 2000, doi: https://doi.org/10.1115/detc2000/mech-14211.
\bibitem{ward2008} C. C. Ward and K. Iagnemma, "A Dynamic-Model-Based Wheel Slip Detector for Mobile Robots on Outdoor Terrain," in IEEE Transactions on Robotics, vol. 24, no. 4, pp. 821-831, Aug. 2008, doi: 10.1109/TRO.2008.924945.
\bibitem{yi2009} J. Yi, H. Wang, J. Zhang, D. Song, S. Jayasuriya, and J. Liu, “Kinematic Modeling and Analysis of Skid-Steered Mobile Robots With Applications to Low-Cost Inertial-Measurement-Unit-Based Motion Estimation,” vol. 25, no. 5, pp. 1087-1097, Oct. 2009, doi: https://doi.org/10.1109/tro.2009.2026506.
\bibitem{mandow2007} A. Mandow, J. Alfredo Martínez, J. Morales, J. M. Blanco, A. García-Cerezo, and J. Suarez Gonzalez, “Experimental kinematics for wheeled skid-steer mobile robots,” Dec. 2007, doi: https://doi.org/10.1109/iros.2007.4399139.
\bibitem{thrun2010} S. Thrun, W. Burgard, and D. Fox, Probabilistic robotics. Cambridge, Mass.: Mit Press, 2010.
\bibitem{moore2014} T. Moore and D. Stouch, "A generalized extended kalman filter implementation for the robot operating system," in Proceedings of the 13th International Conference on Intelligent Autonomous Systems (IAS-13), 2014.
\bibitem{parhi2011} D. R. Parhi and B. B. V. L. Deepak, “Kinematic model of three wheeled mobile robot,” Mechanical Engineering Research, vol. 3, no. 9, pp. 307-318, Sep. 2011, doi: 10.5897/jmer.9000032.
\bibitem{ueno2017} Y. Ueno, K. Watanabe and I. Nagai, "Design and development of steered active wheel casters and its application," 2017 IEEE International Conference on Mechatronics and Automation (ICMA), Takamatsu, Japan, 2017, pp. 507-512, doi: 10.1109/ICMA.2017.8015869.
\bibitem{qiu2018} Q. Qiu et al., “Extended Ackerman Steering Principle for the coordinated movement control of a four wheel drive agricultural mobile robot,” Computers and Electronics in Agriculture, vol. 152, pp. 40-50, Jul. 2018, doi: 10.1016/j.compag.2018.06.036.
\bibitem{gautam2021} P. Gautam, S. Sahai, S. S. Kelkar, P. S. Agrawal, and M. R. D, “Designing Variable Ackerman Steering geometry for Formula Student Race car,” International Journal of Analytical Experimental and Finite Element Analysis (IJAEFEA), vol. 8, no. 1, Feb. 2021, doi: 10.26706/ijaefea.1.8.20210101.
\bibitem{patel2021} S. Patel, R. Rawat, N. Shantanu, A. Kumar, and N. Amardeep, “Study of steering system for an electric Trike-Ackerman steering,” in Smart innovation, systems and technologies, 2021, pp. 9-18. doi: 10.1007/978-981-16-2857-3\_3.
\bibitem{low2005} K. H. Low and Y. P. Leow, “Kinematic modeling, mobility analysis and design of wheeled mobile robots,” Advanced Robotics, vol. 19, no. 1, pp. 73-99, Jan. 2005, doi: 10.1163/1568553053020241.
\bibitem{gillespie1992} T. D. Gillespie, Fundamentals of Vehicle Dynamics. Warrendale, PA: SAE International, 1992.
\bibitem{deluca1998} A. De Luca, G. Oriolo, and C. Samson, "Feedback control of a nonholonomic car-like robot," in Robot Motion Planning and Control, J.-P. Laumond, Ed. Springer, 1998, pp. 171-253.
\bibitem{arrizabalaga2021} J. Arrizabalaga, N. Van Duijkeren, M. Ryll, and R. Lange, “A caster-wheel-aware MPC-based motion planner for mobile robotics,” 2021 20th International Conference on Advanced Robotics (ICAR), pp. 613-618, Dec. 2021, doi: 10.1109/icar53236.2021.9659478.
\bibitem{lauwers2006} T. B. Lauwers, G. A. Kantor, and R. L. Hollis, “A dynamically stable single-wheeled mobile robot with inverse mouse-ball drive,” in Proc. IEEE Int. Conf. Robot. Automat., 2006, pp. 2884-2889.
\bibitem{mukherjee1999} R. Mukherjee, M. A. Minor and J. T. Pukrushpan, "Simple motion planning strategies for spherobot: a spherical mobile robot," Proceedings of the 38th IEEE Conference on Decision and Control (Cat. No.99CH36304), Phoenix, AZ, USA, 1999, pp. 2132-2137 vol.3, doi: 10.1109/CDC.1999.831235.
\bibitem{nagarajan2009} U. Nagarajan, G. Kantor, and R. L. Hollis, “Trajectory planning and control of an underactuated dynamically stable single spherical wheeled mobile robot,” IEEE Xplore, May 01, 2009. https://ieeexplore.ieee.org/document/5152624.
\bibitem{lee2024} W. Lee, J. Kim, and T. Seo, “Design and analysis of a mobile robot with novel caster mechanism for high step-overcoming capability,” Scientific Reports, vol. 14, no. 1, p. 13745, Jun. 2024, doi: 10.1038/s41598-024-63825-y.
\bibitem{garcia2016} J. M. García, J. L. Martínez, A. Mandow, and A. García-Cerezo, “Caster-leg aided maneuver for negotiating surface discontinuities with a wheeled skid-steer mobile robot,” Robotics and Autonomous Systems, vol. 91, pp. 25-37, Dec. 2016, doi: 10.1016/j.robot.2016.12.007.
\bibitem{ignatiev2016} K. V. Ignatiev, M. M. Kopichev and A. V. Putov, "Autonomous omni-wheeled mobile robots," 2016 2nd International Conference on Industrial Engineering, Applications and Manufacturing (ICIEAM), Chelyabinsk, Russia, 2016, pp. 1-4, doi: 10.1109/ICIEAM.2016.7910957.
\bibitem{wikipedia_mecanum_wheel} Wikipedia contributors, “Mecanum wheel,” Wikipedia, Sep. 08, 2025. https://en.wikipedia.org/wiki/Mecanum\_wheel
\bibitem{hirpo2017} B. D. Hirpo , W. Zhongmin, “Design and Control for Differential Drive Mobile Robot”, INTERNATIONAL JOURNAL OF ENGINEERING RESEARCH \& TECHNOLOGY (IJERT), vol. 6, no. 10, pp. 327-334, Oct. 2017.
\bibitem{rodriguez2017} A. A. Rodriguez et al., "Modeling, design and control of low-cost differential-drive robotic ground vehicles: Part II — Multiple vehicle study," 2017 IEEE Conference on Control Technology and Applications (CCTA), Maui, HI, USA, 2017, pp. 161-166, doi: 10.1109/CCTA.2017.8062457.
\bibitem{cao2022} R. Cao, J. Gu, C. Yu and A. Rosendo, "OmniWheg: An Omnidirectional Wheel-Leg Transformable Robot," 2022 IEEE/RSJ International Conference on Intelligent Robots and Systems (IROS), Kyoto, Japan, 2022, pp. 5626-5631, doi: 10.1109/IROS47612.2022.9982030.
\bibitem{lee2017} D.-Y. Lee, S.-R. Kim, J.-S. Kim, J.-J. Park, and K.-J. Cho, “Origami Wheel Transformer: A Variable-Diameter wheel drive robot using an origami structure,” Soft Robotics, vol. 4, no. 2, pp. 163–180, May 2017, doi: 10.1089/soro.2016.0038.
\bibitem{lee2021} D.-Y. Lee, J.-K. Kim, C.-Y. Sohn, J.-M. Heo, and K.-J. Cho, “High–load capacity origami transformable wheel,” Science Robotics, vol. 6, no. 53, Apr. 2021, doi: 10.1126/scirobotics.abe0201.
\bibitem{rhoads2016} B. P. Rhoads and H.-J. Su, “The Design and Fabrication of a Deformable Origami Wheel,” Volume 5B: 40th Mechanisms and Robotics Conference, Aug. 2016, doi: https://doi.org/10.1115/detc2016-60045.
\bibitem{berre2020} J. Berre, François Geiskopf, L. Rubbert, and P. Renaud, “Origami-Inspired Design of a Deployable Wheel,” Mechanisms and machine science, pp. 114-126, Oct. 2020, doi: https://doi.org/10.1007/978-3-030-60076-1\_11.
\bibitem{liu2023} J. Liu et al., “OriWheelBot: An origami-wheeled robot,” arXiv (Cornell University), Jan. 2023, doi: https://doi.org/10.48550/arxiv.2310.00033.
\bibitem{lauwers2006} T. B. Lauwers, G. A. Kantor and R. L. Hollis, "A dynamically stable single-wheeled mobile robot with inverse mouse-ball drive," Proceedings 2006 IEEE International Conference on Robotics and Automation, 2006. ICRA 2006., Orlando, FL, USA, 2006, pp. 2884-2889, doi: 10.1109/ROBOT.2006.1642139.
\bibitem{liao2008} Ching-Wen Liao, Ching-Chih Tsai, Yi Yu Li and Cheng-Kai Chan, "Dynamic modeling and sliding-mode control of a Ball robot with inverse mouse-ball drive," 2008 SICE Annual Conference, Chofu, Japan, 2008, pp. 2951-2955, doi: 10.1109/SICE.2008.4655168.
\bibitem{trojnacki2019} M. Trojnacki and P. D\pl{ą}bek, “Mechanical properties of modern wheeled mobile robots,” Journal of Automation Mobile Robotics \& Intelligent Systems, pp. 3-13, Jul. 2019, doi: 10.14313/jamris/3-2019/21.
\bibitem{nakata2022} Y. Nakata, S. Yagi, S. Yu, Y. Wang, N. Ise, Y. Nakamura, and H. Ishiguro, "Development of 'ibuki' an electrically actuated childlike Android with mobility and its potential in the future society," Robotica, vol. 40, no. 4, pp. 933-950, Apr. 2022.
\bibitem{kumagai2008} M. Kumagai and T. Ochiai, "Development of a robot balancing on a ball," in Proc. Int. Conf. Control, Autom. Syst., Hong Kong, Oct. 2008, pp. 433-438.
\bibitem{bhatia2015} A. Bhatia, M. Kumagai, and R. Hollis, "Six-stator spherical induction motor for balancing mobile robots," in Proc. IEEE Int. Conf. Robot. Autom. (ICRA), May 2015, pp. 226-231.
\bibitem{abdelrahim2025} M. Abdelrahim, M. A. Thabet, H. S. Abbas, M. M. M. Hassan, M. H. Amin and A. Morsi, "Modeling and Control of a Ballbot: A Systematic Approach," in IEEE Access, vol. 13, pp. 141263-141280, 2025, doi: 10.1109/ACCESS.2025.3597797.


\end{thebibliography}

\end{document}